%%
%% This is file `sample-lualatex.tex',
%% generated with the docstrip utility.
%%
%% The original source files were:
%%
%% samples.dtx  (with options: `sigconf')
%% 
%% IMPORTANT NOTICE:
%% 
%% For the copyright see the source file.
%% 
%% Any modified versions of this file must be renamed
%% with new filenames distinct from sample-lualatex.tex.
%% 
%% For distribution of the original source see the terms
%% for copying and modification in the file samples.dtx.
%% 
%% This generated file may be distributed as long as the
%% original source files, as listed above, are part of the
%% same distribution. (The sources need not necessarily be
%% in the same archive or directory.)
%%
%% The first command in your LaTeX source must be the \documentclass command.
\documentclass[sigconf]{acmart}

%%
%% \BibTeX command to typeset BibTeX logo in the docs
\AtBeginDocument{%
  \providecommand\BibTeX{{%
    \normalfont B\kern-0.5em{\scshape i\kern-0.25em b}\kern-0.8em\TeX}}}

%% Rights management information.  This information is sent to you
%% when you complete the rights form.  These commands have SAMPLE
%% values in them; it is your responsibility as an author to replace
%% the commands and values with those provided to you when you
%% complete the rights form.
\setcopyright{acmcopyright}
\copyrightyear{2018}
\acmYear{2018}
\acmDOI{10.1145/1122445.1122456}

%% These commands are for a PROCEEDINGS abstract or paper.
\acmConference[Woodstock '18]{Woodstock '18: ACM Symposium on Neural
  Gaze Detection}{June 03--05, 2018}{Woodstock, NY}
\acmBooktitle{Woodstock '18: ACM Symposium on Neural Gaze Detection,
  June 03--05, 2018, Woodstock, NY}
\acmPrice{15.00}
\acmISBN{978-1-4503-XXXX-X/18/06}


%%
%% Submission ID.
%% Use this when submitting an article to a sponsored event. You'll
%% receive a unique submission ID from the organizers
%% of the event, and this ID should be used as the parameter to this command.
%%\acmSubmissionID{123-A56-BU3}

%%
%% The majority of ACM publications use numbered citations and
%% references.  The command \citestyle{authoryear} switches to the
%% "author year" style.
%%
%% If you are preparing content for an event
%% sponsored by ACM SIGGRAPH, you must use the "author year" style of
%% citations and references.
%% Uncommenting
%% the next command will enable that style.
%%\citestyle{acmauthoryear}

%%
%% end of the preamble, start of the body of the document source.
\begin{document}

%%
%% The "title" command has an optional parameter,
%% allowing the author to define a "short title" to be used in page headers.
\title{MockDetector: A technique to identify mock objects created in unit tests}

%%
%% The "author" command and its associated commands are used to define
%% the authors and their affiliations.
%% Of note is the shared affiliation of the first two authors, and the
%% "authornote" and "authornotemark" commands
%% used to denote shared contribution to the research.
\author{Qian Liang}
\email{q8liang@uwaterloo.ca}
\author{Patrick Lam}
\email{patrick.lam@uwaterloo.ca}
\affiliation{%
  \institution{University of Waterloo}
  \city{Waterloo}
  \state{Ontario}
  \country{Canada}
}

%%
%% The abstract is a short summary of the work to be presented in the
%% article.
\begin{abstract}
Unit testing is a widely used tool in modern software development processes. A well-known issue in writing tests is handling dependencies: creating usable objects for dependencies is often complicated. Developers must therefore often introduce mock objects to stand in for dependencies during testing. 

We believe that the static analysis of test suites can enable developers to better understand and maintain existing test suites. However, because mock objects are created using reflection, they confound existing static analysis techniques. At present, it is impossible to statically distinguish methods invoked on mock objects from methods invoked on real objects. 

Researchers have started to build static analyses and developer tools for manipulating test cases. However, such tools currently cannot determine which dependencies' methods are actually tested, versus mock methods being called. As a specific example, removing confounding mock invocations from consideration as focal methods can improve the precision of analyses to detect focal methods under test, which is useful in itself and also a key prerequisite to further analysis of test cases.

In this paper, we introduce MockDetector, a technique to identify mock objects and pinpoint method invocations on mock objects. MockDetector locates common Java mock libraries' APIs for creating mock objects and propagates this information through test cases. Following our observations of tests in the wild, we have added special-case support for arrays and collections holding mock objects. We have built two implementations of MockDetector: a Soot-based imperative dataflow analysis implementation, as well as a Doop-based declarative analysis. On our suite of 8 open-source benchmarks, our imperative intraprocedural approach reported 2,095 invocations on mock objects, whereas our declarative interprocedural approach reported 5,315 invocations on mock objects (under context-insensitve base analyses), out of a total number of 63,017 method invocations in test suites; across benchmarks, mock invocations accounted for a range from 0.086\% to 31.8\% of the total invocations in tests.
	
\end{abstract}


%%
%% The code below is generated by the tool at http://dl.acm.org/ccs.cfm.
%% Please copy and paste the code instead of the example below.
%%
\begin{CCSXML}
<ccs2012>
 <concept>
  <concept_id>10010520.10010553.10010562</concept_id>
  <concept_desc>Computer systems organization~Embedded systems</concept_desc>
  <concept_significance>500</concept_significance>
 </concept>
 <concept>
  <concept_id>10010520.10010575.10010755</concept_id>
  <concept_desc>Computer systems organization~Redundancy</concept_desc>
  <concept_significance>300</concept_significance>
 </concept>
 <concept>
  <concept_id>10010520.10010553.10010554</concept_id>
  <concept_desc>Computer systems organization~Robotics</concept_desc>
  <concept_significance>100</concept_significance>
 </concept>
 <concept>
  <concept_id>10003033.10003083.10003095</concept_id>
  <concept_desc>Networks~Network reliability</concept_desc>
  <concept_significance>100</concept_significance>
 </concept>
</ccs2012>
\end{CCSXML}

\ccsdesc[500]{Computer systems organization~Embedded systems}
\ccsdesc[300]{Computer systems organization~Redundancy}
\ccsdesc{Computer systems organization~Robotics}
\ccsdesc[100]{Networks~Network reliability}

%%
%% Keywords. The author(s) should pick words that accurately describe
%% the work being presented. Separate the keywords with commas.
\keywords{mock analysis, unit testing}

%%
%% This command processes the author and affiliation and title
%% information and builds the first part of the formatted document.
\maketitle

\section{Introduction}
\label{sec:introduction}

Mock objects~\cite{beck02:_test_driven_devel} are commonly used in
unit tests for object-oriented systems.  They allow developers to test objects that 
rely on other objects, particularly ones that are hard 
to build (e.g. databases; or that come from different components).

While mock objects are an invaluable tool for developers, their use
complicates the static analysis and manipulation of test case source code. 
Such static analyses can help IDEs provide better
support to test case writers; enable better static estimation of test coverage
(avoiding mocks); and detect focal methods in test cases. While researchers have
proposed techniques for automatically generating mocks~\cite{alshahwan10:_autom,fazzini20:_framew_autom_test_mockin_mobil_apps}, our goal here is the opposite:
we detect mocks that already exist in test cases.

Ghafari et al discussed the notion of a focal method~\cite{ghafari15:_autom} for a test case---the method
whose behaviour is being tested---and presented a heuristic for determining focal methods.
By definition, the focal method's receiver object cannot be a mock object.
Ruling out mock invocations can thus improve the accuracy of focal method detection and
enable better understanding of a test case's behaviour.

Mock objects are difficult to analyze statically because, at the bytecode level,
a call to a mock object statically resembles a call to the real object (as
intended by the designers of mock libraries).
A naive static analysis attempting to be sound would have to include all of 
the possible behaviours of the actual object (rather than the mock) when analyzing such code. 
Such potential but unrealizable behaviours obscure the true behaviour 
of the test case.

We have designed a static analysis, \textsc{MockDetector}, which identifies
mock objects in JUnit\footnote{\url{https://junit.org}} test cases. It starts from a list of mock object creation sites
(our analyses include hardcoded APIs for common mocking libraries EasyMock\footnote{\url{https://easymock.org/}}, Mockito\footnote{\url{https://site.mockito.org/}}, and PowerMock\footnote{\url{https://github.com/powermock/powermock}}). 
It then propagates mockness
through the test and identifies invocation sites as (possibly) mock.
Given this analysis result, a subsequent analysis
can ask whether a given variable in a test case contains a mock or not, and
whether a given invocation site is a call to a mock object or not. We have
evaluated \textsc{MockDetector} on a suite of 8 benchmarks plus a microbenchmark. 
We have cross-checked results across the two implementations and manually inspected
the results on our microbenchmark, to ensure that the results are as expected.

Taking a broader view, we believe that helper static analyses like \textsc{MockDetector} 
can aid
in the development of more useful static analyses. These analyses can
encode useful domain properties; for instance, in our case, properties
of test cases. By taking a domain-specific approach, analyses can extract
useful facts about programs that would otherwise be difficult to establish.

We make the following contributions in this paper:
\begin{itemize}
	\item We designed and implemented two variants of a static mock detection algorithm, one as a dataflow analysis implemented imperatively (using Soot) and the other declaratively (using Doop).
	%\item We evaluate both the relative ease-of-implementation and precision of the imperative and declarative approaches, both intraprocedurally and interprocedurally (for Doop). % potentially intraprocedural as well
	\item We characterize our benchmark suite (8 open-source benchmarks, 383kLOC, 184 kLOC tests) with respect to their use of mock objects, finding that 1,084 out of 6,310 unit tests use intraprocedurally-detectable mocks, and that there are a total of 2,095 method invocations on mocks. %We further identify how powerful an analysis is required to identify mock object use---adding fields and collections adds X mock objects, while interprocedural techniques add Y mock objects.
	\item We present potential applications of mock analysis: detecting focal methods, helping to understand test cases, automated refactoring, and API usage extraction.
\end{itemize}
%% At a higher level, we see this paper as making both a contribution and a meta-contribution to
%% problems in source code analysis. The contribution, mock detection, enables more accurate analyses
%% of test cases, which account for a signficant fraction of modern codebases. The meta-contribution,
%% comparing analysis approaches, will help future researchers decide how to best solve their
%% source code analysis problems. In brief, the declarative approach allows users to quickly prototype, stating their properties
%% concisely, while the imperative approach is more amenable to use in program transformation; we return
%% to this question in Section~\ref{sec:discussion}.

We will make all of our artifacts and data publicly available.
%For the purposes of the not-very-well-defined
%required replication package for this submission, see: \url{https://anonymous.4open.science/r/MockAbstraction-B0F0/} or \url{https://figshare.com/s/285073ed5bf5b1e5e7a9}.
% also add somewhere:
% it's notoriously difficult to check static analyses but this is basically N-version programming and we can cross-check results.
% "The N-Version Approach to Fault-Tolerant Software"



\section{Approach}

\section{Experiment Setup}

\section{Evaluation}
\label{sec:evaluation}

We now quantitatively characterize the performance of our mock analysis implementations on a suite of popular open-source applications. We also provide evidence about the importance of mock invocations to test suites.

\begin{table*}
	\centering
	\caption{Our suite of 8 open-source benchmarks (8,000--117,000 LOC) plus our microbenchmark. Soot and Doop analysis run-times.}
	%	\begin{adjustbox}{width=0.1\textwidth}
	\resizebox{.95\textwidth}{!}{\begin{tabular}{lrrrrrrr}
			\toprule
			Benchmark & Total LOC & Test LOC & \thead{Soot intraproc \\ total (s)} & \thead{Doop intraproc \\ total (s)} & \thead{Soot intraproc \\ mock analysis (s)}  & \thead{Doop intraproc \\ mock analysis (s)} & \thead{Doop interproc \\ mock analysis (s)} \\
			\midrule
			bootique-2.0.B1-bootique         &  15530   & 8595   & 58  & 2810  &  0.276   & 19.93  & 24.90    \\
			commons-collections4-4.4         &  65273   & 36318  & 114 & 694   &  0.386   & 14.20  & 16.64     \\
			flink-core-1.13.0-rc1            &  117310  & 49730  & 341 & 1847  &  0.415   & 27.21  & 62.12      \\
			jsonschema2pojo-core-1.1.1       &  8233    & 2885   & 313 & 1005   &  0.282   & 29.33 & 41.05      \\
			maven-core-3.8.1   		         &  38866   & 11104  & 183 & 588   &  0.276   & 19.49  & 23.42     \\
			micro-benchmark         		 &  954     & 883	& 47  & 387   &  0.130   & 11.73   & 12.92     \\
			mybatis-3.5.6         		  	 &  68268   & 46334  & 500 & 4477  &  0.662   & 59.83  & 192.16      \\
			quartz-core-2.3.1        	  	 &  35355   & 8423   & 155 & 736   &  0.231   & 21.06  & 21.92   \\
			vraptor-core-3.5.5         	  	 &  34244   & 20133  & 371 & 1469  &  0.455   & 34.95  & 149.38    \\
			\bottomrule
			Total         	  				 &  384033  & 184405 & 2082 & 14013 &  3.123  & 237.73  & 544.51    \\
	\end{tabular}}
	%	\end{adjustbox}
	\label{tab:run-times}
\end{table*}

\begin{table*}
	\centering
	\caption{Counts of Test-Related (Test/Before/After) methods in public concrete test classes; counts of mocks, mock-containing arrays, mock-containing collections; and total number of field mock objects reported by Soot intraprocedural analysis.}
	%	\begin{adjustbox}{width=0.1\textwidth}
	\begin{tabular}{lrrrrr}
		\toprule
		Benchmark & \thead{\# of Test-Related \\ Methods} & \thead{\# of Test-Related \\ Methods with \\ mocks (intra)}  & \thead{\# of Test-Related \\ Methods with \\ mock-containing\\ arrays (intra)} & \thead{\# of Test-Related \\ Methods with \\ mock-containing\\ collections (intra)} & \thead{\# of Field \\ Mock Objects} \\
		\midrule
		bootique-2.0.B1-bootique           		&  420        &  32  & 7 & 0     &   8   \\
		commons-collections4-4.4          		&  1152       &  3   & 1 & 1     &   0   \\
		flink-core-1.13.0-rc1           		&  1091       &  4   & 0 & 0     &   0   \\
		jsonschema2pojo-core-1.1.1           	&  145        &  76  & 1 & 0     &   152  \\
		maven-core-3.8.1	           			&  337        &  24  & 0 & 0     &   8    \\
		micro-benchmark         		  		&  59         &  43  & 7 & 25    &   31   \\
		mybatis-3.5.6         		  			&  1769       &  330 & 3 & 0     &   41   \\	
		quartz-core-2.3.1         	  			&  218     	  &  7   & 0 & 0     &   0    \\
		vraptor-core-3.5.5         	  			&  1119       &  565 & 15 & 0    &   474  \\
		\bottomrule
		Total        	  						&  6310       &  1084  & 34 & 26  &  714   \\
	\end{tabular}
	%	\end{adjustbox}
	\label{tab:mocks}
\end{table*}

\begin{table*}
	\centering
	\caption{Total \# of InstanceInvokeExprs, and \# of InstanceInvokeExprs with Mock receivers found by Soot and Doop.}
	%	\begin{adjustbox}{width=0.1\textwidth}
	\begin{tabular}{lrrrr}
		\toprule
		Benchmark & \thead{Total Number \\ of Invocations} & \thead{Mock Invokes \\ intraproc (Soot)}  & \thead{Context-insensitive, \\ intraproc (Doop)} &\thead{Context-insensitive, \\ interproc (Doop)} \\
		\midrule
		bootique-2.0.B1-bootique           		&  3366     &  99  & 99   & 122    \\
		commons-collections4-4.4       			&  12753    &  11   &  3   & 23   \\
		flink-core-1.13.0-rc1           		&  11923    &  40   & 40   & 1389   \\
		jsonschema2pojo-core-1.1.1      	     	&  1896     &  276  & 282  & 604   \\
		maven-core-3.8.1           			&  4072     &  23   & 23   & 39  \\
		microbenchmark         		  		&  471      &  108  & 123  & 132   \\
		mybatis-3.5.6         		  		&  19232    &  575  & 577  & 1345     \\
		quartz-core-2.3.1       	  		&  3436     &  21   & 21   & 31    \\
		vraptor-core-3.5.51        	  		&  5868     &  942  & 962  & 1630   \\
		\bottomrule
		Total        	  				&  63017    & 2095  & 2130  & 5315   \\
	\end{tabular}
	%	\end{adjustbox}
	\label{tab:invokes}
\end{table*}

%\begin{table*}
%	\centering
%	\caption{Counts of Field Mock Objects defined via \protect \texttt{@Mock} annotation, \hspace{\textwidth} in the constructors, and in \texttt{@Before} methods, in each benchmark's test suite.}
%	%	\begin{adjustbox}{width=0.1\textwidth}
%	\vspace*{.5em}
%	\resizebox{1.3\columnwidth}{!}{
%		\begin{tabular}{lrrrr}
%			\toprule
%			Benchmark & \thead{\# of Annotated \\ Field Mock Objects} & \thead{\# of Field Mock Objects \\ defined in the \texttt{<init>} constructor}  & \thead{\# of Field Mock Objects \\ defined in @Before methods} \\
%			\midrule
%			bootique-2.0.B1-bootique           		&  0        &  0    & 8        \\
%			commons-collections4-4.4          		&  0        &  0    & 0        \\
%			flink-core-1.13.0-rc1           		&  0        &  0    & 0        \\
%			jsonschema2pojo-core-1.1.1           	&  26       &  126  & 0        \\
%			maven-core-3.8.1	           			&  7        &  0    & 1        \\
%			micro-benchmark         		  		&  2        &  0    & 29        \\
%			mybatis-3.5.6         		  			&  41       &  0    & 0        \\
%			quartz-core-2.3.1         	  			&  0     	&  0    & 0      \\
%			vraptor-core-3.5.5         	  			&  263      &  128  & 83       \\
%			\bottomrule
%		\end{tabular}
%	}
%	%	\end{adjustbox}
%	\label{tab:field-mocks}
%\end{table*}

%\begin{table*}
%	\centering
%	\caption{Doop analysis-only run-time after basic-only and context-insensitive base analyses. N/A = timed out after 90 minutes.}
%	%	\begin{adjustbox}{width=0.1\textwidth}
%	\begin{tabular}{lrrrrrr}
%		\toprule
%		Benchmark & \thead{Basic-only, \\ intraproc (s)} & \thead{Context-insensitive, \\ intraproc (s)} & \thead{Basic-only, \\ interproc (s)}  & \thead{Context-insensitive, \\ interproc (s)}  \\
%		\midrule
%		bootique-2.0.B1-bootique           		& 15.71  & 16.81 &  24.26    &  20.20     \\
%		commons-collections4-4.4           		& 17.42  & 12.26 &  21.79    &  15.36        \\
%		flink-core-1.13.0-rc1           		& 24.67  & 25.30 &  71.67    &  66.10         \\
%		jsonschema2pojo-core-1.1.1         		& 25.98  & 26.27 &  42.14    &  39.21         \\
%		maven-core-3.8.1   		        	& 18.01  & 16.34 &  25.49    &  22.09          \\
%		micro-benchmark         			& 10.97  & 10.50 &  12.51    &  12.53        \\
%		mybatis-3.5.6         		  		&  N/A   & 51.25 &   N/A     & 183.86          \\
%		quartz-core-2.3.1        	  		& 17.72  & 19.83 &  22.99    &  21.14        \\
%		vraptor-core-3.5.5         	  		& 22.10  & 23.81 &  66.73    & 146.09       \\
%		\bottomrule
%	\end{tabular}
%	%	\end{adjustbox}
%	\label{tab:doop-runtimes}
%\end{table*}



%\begin{table*}
%	\centering
%	\caption{Counts of mock invocations for Doop in basic-only and context-insensitive options, and for interprocedural and intraprocedural .}
%	%	\begin{adjustbox}{width=0.1\textwidth}
%	\begin{tabular}{lrrrr}
%		\toprule
%		Benchmark & \thead{Basic-only, \\ intraproc} & \thead{Context-insensitive, \\ intraproc} & \thead{Basic-only, \\ interproc} & \thead{Context-insensitive, \\ interproc} \\
%		\midrule
%		bootique-2.0.B1-bootique           		&    &    &   &       \\
%		commons-collections4-4.4           		&    &    &   &        \\
%		flink-core-1.13.0-rc1           		&    &    &   &         \\
%		jsonschema2pojo-core-1.1.1         		&    &    &   &          \\
%		maven-core-3.8.1   		           		&    &    &   &           \\
%		micro-benchmark         		  		&    & 	  &   &           \\
%		mybatis-3.5.6         		  			&    &    &   &          \\
%		quartz-core-2.3.1        	  			&    &    &   &         \\
%		vraptor-core-3.5.5         	  			&    &    &   &         \\
%		\bottomrule
%		Total         	  						&    &    &   &         \\
%	\end{tabular}
%	%	\end{adjustbox}
%	\label{tab:doop-mock-invokes}
%\end{table*}

%% The goal of our study is to correctly identify and trace mock objects as well as the method invocations in the test suite. To this end, we conduct quantitative and qualitative research focusing on two research questions:

%% \begin{quote}
%% 	\emph{RQ 1: Are the mocks correctly identified and traced for each test method?}
%% \end{quote}

%% \begin{quote}
%% 	\emph{RQ 2: Would this be helpful for existing static analysis tools?}
%% \end{quote}


\paragraph{Benchmark suite} We evaluated \textsc{MockDetector} on 8 open-source benchmarks, plus a micro-benchmark we developed to test our tool. We ran our experiments on a 32-core Intel(R) Xeon(R) CPU E5-4620 v2 at 2.60GHz with 128GB of RAM running Ubuntu 16.04.7 LTS.

Table~\ref{tab:run-times} presents summary information about our benchmarks and run-times---the LOC and Soot and Doop analysis run-times for each benchmark. The 9 benchmarks include over 383 kLOC, with 184 kLOC in the test suites, per SLOCCount\footnote{\url{https://dwheeler.com/sloccount/}}.  Our benchmarks are from different domains and created by different groups of developers. The Soot total time is the amount of time that it takes for Soot to analyze the benchmark and test suite in whole-program mode, including our analyses. The Soot intraprocedural analysis time is the sum of run-times for the main analysis plus two pre-analyses (Section~\ref{subsec:soot}). The reported Doop run-time is from the context-insensitive analysis, while the Doop analysis time for intraprocedural and interprocedural mock invocation analyses are for running the analysis alone based on recorded facts from the benchmark. The total Doop run-time is much slower than the total Soot run-time because Doop always computes a callgraph---an expensive operation. The Doop analysis-only time is also slower than the Soot time; Doop computes a solution over the entire program, while Soot works one method at a time.
%The major difference between the Doop's total run-time and the actual time spent on mock invocation analysis comes from the build of the complete graph. %Add reference for SLOCCount.

\paragraph{Mock usage in suite} Table~\ref{tab:mocks} presents the number of test-related (Test/Before/After) methods that contain local variables or that access fields that are mocks, mock-containing arrays, or mock-containing collections, as reported by our Soot-based intraprocedural analysis. Table~\ref{tab:mocks} also presents the total number of field mock objects created in each benchmark; 5 out of the 8 open-source benchmarks use field mock objects for ease of testing. Instead of creating the same mock objects in each test case requiring them, these benchmarks choose to create the field mock objects once and use them in all the test cases. The data suggest that our pre-analysis for field mocks is necessary to consequently analyzing for mock objects and mock invocations.

\begin{mdframed}[
	leftmargin=\parindent,
	rightmargin=\parindent,
	skipabove=\topsep,
	skipbelow=\topsep
	]
	{\bf Finding 1:} Across the 8 benchmarks, test-related methods containing local/field mocks or mock-containing containers accounted for 0.35\% to 51.8\% of the total number of test-related methods found in public concrete test classes.
\end{mdframed}

Of our benchmarks, 3 show almost no mock use, 2 show extensive use, and the remainder are in between. Benchmarks \textsc{vraptor-core} and \textsc{jsonschema2pojo-core} have more than half of their test-related methods containing mock objects (and mock-containing arrays); in both of these, most field mocks are created via annotations and reused in multiple test cases. The difference in mock usage reflects their different philosophies and constraints regarding the creation and usage of mock objects in tests.

\subsection{Mock Analysis Results}
Table~\ref{tab:invokes} is the core result about our mock analyses. It presents the detected number of method invocations on mocks. We include numbers from the imperative intraprocedural Soot implementation, as well as intraprocedural and interprocedural versions of the declarative Doop implementation.

\begin{mdframed}[
	leftmargin=\parindent,
	rightmargin=\parindent,
	skipabove=\topsep,
	skipbelow=\topsep
	]
	{\bf Finding 2:} Our intraprocedural analysis finds that method calls on mock objects account for 0.086\% to 16.4\% of the total number of method calls in tests. 
\end{mdframed}

%--- discuss the numbers for the interprocedural analysis.

In Section~\ref{sec:common} we discussed the implementation of our intraprocedural and interprocedural analyses. We can now discuss the effects of these implementation choices on the experimental results. Recall that we chose, unsoundly, to not propagate any information across method calls in the intraprocedural analysis. Thus, the intraproc columns in Table~\ref{tab:invokes} show smaller numbers than the interproc columns, as expected.

There is a sometimes drastic increase from the intraprocedural to the interprocedural result, e.g. from 40 to 1300 for \textsc{flink}. This is because mocks can propagate from tests to the methods that they call and further. Most of the increase is in main code.

To understand better, we manually investigated the 23 additional interprocedural mocks in \textsc{bootique}. Here, our intraprocedural analysis does not miss any mock invocations in tests. Listing~\ref{lis:bootiqueMockCall} illustrates. \texttt{mockParsed.valueOf()} is a mock invocation. \texttt{opts.optionStrings()} is not; there is a real \texttt{opts} object. But in \texttt{optionStrings()} there is a call to method \texttt{optionSet.values()}. Our interprocedural analysis marks \texttt{optionSet} as a mock field when it analyzed the constructor, and then flagged the call to \texttt{optionSet.values()} (in main code) as a mock, while the intraprocedural analysis assumed that the constructor got no mock parameters, and hence does not mark the field as mock.

\begin{lstlisting}[basicstyle=\ttfamily, caption={An interprocedural mock invocation from boutique's \texttt{JoptCliTest}.},
basicstyle=\scriptsize\ttfamily,language = Java, framesep=4.5mm, escapechar=|,
framexleftmargin=1.0mm, captionpos=b, label=lis:bootiqueMockCall, morekeywords={@Test}]
// JoptCliTest, modified to use @Mock vs @Before
@Mock private OptionSet mockParsed;

@Test public void testStringsFor_Missing() {
  when(mockParsed.valueOf(anyString())).
      thenReturn(Collections.emptyList());

  JoptCli opts = new JoptCli(mockParsed, "aname");
  assertNotNull(opts.optionStrings("no_such_opt"));
  /* ... */ }

// JoptCli
private OptionSet optionSet;
private String commandName;

public JoptCli(OptionSet parsed, String commandName) {
  this.optionSet = parsed; this.commandName = commandName; }

@Override public List<String> optionStrings(String name) {
  return /* mock */ optionSet.valuesOf(name).stream()
      .map(String::valueOf).collect(toList()); }
\end{lstlisting}

One false negative in the intraprocedural analysis that we specifically point out is due to lambda expressions. Mockness should propagate to lambdas, but that requires an interprocedural analysis. The Doop interprocedural implementation finds these mocks.

% TODO transitive mocks

\begin{mdframed}[
	leftmargin=\parindent,
	rightmargin=\parindent,
	skipabove=\topsep,
	skipbelow=\topsep
	]
	{\bf Finding 3:} Interprocedural analysis finds from 1.07$\times$ to 34$\times$ more mock invocations than intraprocedural analysis.
\end{mdframed}

We have successfully executed our Doop analysis with different base (pointer) analyses. We reported numbers for the context-insensitive base analysis here as it matches our own mock analysis. (It would also be possible, with much more effort, to adapt our analysis to carry around context.)

\paragraph{Imprecision and Unsoundness in Practice.} In Section~\ref{subsec:analysis-design} we discussed possible sources of imprecision for our static analysis; our analysis could be subject to false positives (loss of precision due to reported mocks that aren't) or false negatives (unsoundness due to unreported mocks). Cross-checking the results revealed that the two approaches have similar precision in detecting intraprocedural mock invokes. We observed some false negatives in the Soot implementation (fewer than 1\%, mostly due to missing support for \texttt{PriorityQueue} and \texttt{TreeSet}). The implementations also differ because Soot does not identify mock invocations in abstract test classes, constructors, or methods not labelled as tests. False negatives common to the two implementations are harder to detect.

After a manual inspection of our \textsc{bootique} benchmark showed zero false positives on that benchmark, we realized that dataflow analysis on test code is likely to have far fewer false positives than on arbitrary code, because it does not have many control-flow merges. We thus instrumented our Soot analysis implementations to count the number of merges where our static analysis must approximate. Among the 1,084 test-related methods in our benchmark suite that contain intraprocedural mocks, only 23 of them have merges (conditionals, loops or try-catch blocks) with different information on the two incoming branches.  {\bf An exhaustive manual inspection of the 23 methods showed that our analysis reports \emph{zero} false positive mock objects or mock invocations due to dataflow merges on our benchmark set.}

We have not quantified imprecision due to arrays, fields, and collections; however, we would be surprised if it were important to a client of our analysis that an object from an array, field, or collection could potentially be either mock or non-mock. Most of the time, such an object should be treated as a mock.


%\paragraph{Field Mocks Results} We perform an evaluation on the necessity of our pre-analyses finding field mocks. 
%Table~\ref{tab:field-mocks} displays the number of field mock objects that are defined via \texttt{@Mock} annotations, in the constructors, and in the \texttt{@Before}/\texttt{setUp()} methods, respectively. 
%
%We focus on the 5 benchmarks that have defined field mock objects. 
%%They are \textsc{bootique}, \textsc{maven-core}, \textsc{jsonschema2pojo-core}, \textsc{mybatis}, and \textsc{vraptor-core}. 
%Among these benchmarks, \textsc{jsonschema2pojo-core}, \textsc{mybatis}, and \textsc{vraptor-core} have a high number (565) or a high percentage (over 50\%) of test-related methods containing mock objects, with many intraprocedural mock invokes. From the results in Table~\ref{tab:field-mocks}, we can tell these benchmarks also define field mock objects instead of repetitively creating the same mock objects in each test, reducing the need for code maintenance. In addition, although \textsc{bootique} and \textsc{maven-core} have lower number of tests using mock objects, these benchmarks still define field mocks. 
%
%\begin{mdframed}[
%	leftmargin=\parindent,
%	rightmargin=\parindent,
%	skipabove=\topsep,
%	skipbelow=\topsep
%	]
%	{\bf Finding 4:} More than half (5/8) of our benchmarks define field mock objects in their tests.
%\end{mdframed}
%We can deduce that our pre-analysis for field mocks described in Section~\ref{subsubsec:pre-analysis} is required for analyzing mocks in  test suites.

\subsection{Application: Mocks contribute to coverage} Since our mock analysis identifies mock objects, we are in a position to empirically evaluate the importance of mock objects in our benchmarks' test suites. One measure of their importance is how much they contribute to code coverage. Using jacoco with the surefire Maven plugin, we measured branch and statement coverage for 4 of our benchmarks (the ones with nontrivial mock usage, excluding one benchmark with parametrized test cases), both with and without test cases that contain intraprocedural mock invocations. We excluded intraprocedural mocks (on a per-test-case granularity) by generating custom Maven test execution commandlines.

\begin{table*}
	\centering
	\caption{Comparison of Statement Coverage and Branch Coverage with all test cases, \hspace{\textwidth} and with only test cases that do not contain intraprocedural mock invocations.}
	\vspace*{.5em}
	\begin{tabular}{lrrrrr} \toprule
		& \multicolumn{2}{c}{Statement Coverage} & & \multicolumn{2}{c}{Branch Coverage} \\
		\cmidrule{2-3} \cmidrule{5-6}
		\thead{Benchmark} & \thead{All Test Cases} & \thead{Test Cases without \\ Intraproc Mocks} & & \thead{All Test Cases} & \thead{Test Cases without \\ Intraproc Mocks} \\ 
		\midrule
		
		jsonschema2pojo-core-1.1.1  & 37\%  & 24\% & & 33\%    &  19\%     \\
		maven-core-3.8.1   		    & 48\%  & 48\% & & 39\%    &  38\%       \\
		mybatis-3.5.6   		    & 85\%  & 81\% & &  82\%    &  76\%        \\
		vraptor-core-3.5.5         	& 87\%  & 59\% & & 81\%   &  56\%    \\
		\bottomrule
	\end{tabular}
	\label{tab:test-coverages}
\end{table*}

Table~\ref{tab:test-coverages} presents our results. We can see that for benchmarks \textsc{jsonschema2pojo-core} and \textsc{vraptor-core}, which have about 15\% mock invocations, statement coverage drops by over 13 to 28 percentage points when excluding intraprocedural mocks. The fact that a relationship exists may be unsurprising, but the magnitude of the 28 point gap of statement coverage for \textsc{vraptor-core} is striking, and points out that the mock objects are indispensable in achieving coverage. We conclude that mock objects play a key role in test suites. Correctly tracing mock objects and invocations could provide better insights about benchmarks relying on mock objects for testing. For instance, with mock objects correctly identified, developers could easily find out which lines of their code are only executed in mock-using test cases. Note that our analysis, or a dynamic version thereof, is required to collect these numbers.

\begin{mdframed}[
	leftmargin=\parindent,
	rightmargin=\parindent,
	skipabove=\topsep,
	skipbelow=\topsep
	]
	{\bf Finding 4:} Test suites use mocks to increase statement coverage by 13--28\% versus not using mocks.
\end{mdframed}

%% \begin{mdframed}[
%% 	leftmargin=\parindent,
%% 	rightmargin=\parindent,
%% 	skipabove=\topsep,
%% 	skipbelow=\topsep
%% 	]
%% 	{\bf Finding 5:} About 2\% of intraprocedural mock-containing test related methods uses branches. % ... to be added
%% \end{mdframed}

%--- , indicates the removal of mock invocations from call graph would improve the call graph's accuracy on method coverage for the benchmarks on the high end of the mock invocation percentage. 

%We explored the performance of our 4 declarative analysis variants based on recorded program facts, and present the numbers in Table~\ref{tab:doop-run-times}. We can observe that the number of mock invocations correlates with the run-time; taking a bit more effort to compute a better call graph may well pay off in terms of overall analysis time. We suspect that the interprocedural analysis is especially slow for mybatis because we also analyze its 50 dependencies; that count is at the high end among our benchmarks.


%% \subsection{Application}
%% \label{subsec:static}

%% There are more test cases holding interprocedural mocks (i.e., the mock object is created in a helper method and passed into the test case) in commons-collections and micro-benchmark. The interprocedural analysis is currently in development and will be discussed in Section~\ref{sec:discussion}.

%% The Procedure Summaries produced after the analysis has indicated that the tracing of "mockiness" of variables and containers is also correct through the whole program. 

%% The accuracy results in tracing intraprocedural mock objects or containers have indicated that \textsc{MockDetector} has the potential to be applied as a helper for existing static analysis tools. By adding proper adjustment, it could pass the mock information to the static analysis, so that the generated call graph may appropriately omit the methods invoked on mock objects, thus increasing its accuracy.

%% By running evaluation (also interprocedurally) on more benchmarks, our tool would have the potential to finding the scenario where developers prefer using mock objects for dependencies, and subsequently providing mock suggestions.


%% context-sensitivity for analysis


\section{Conclusion}
\label{sec:conclusion}

Our thesis is that mock objects are an important technique that developers use when creating test suites. However, common mock object libraries use reflection and cause developers to write tests that appear to have different behaviours than they actually do---method invocations on mocks look like normal method calls, but instead record behaviour. Because of the prevalence of mocks, tools that work with tests (including static analyses, IDEs, and automatic repair tools) need to correctly handle mock objects. 

We have described our \textsc{MockDetector} static analysis, which we intend to use for further static analyses of test cases. We have implemented \textsc{MockDetector} imperatively in Soot and declaratively in Doop, and characterized its performance and behaviour on a set of 8 open-source benchmarks. Our results show that mocks play an important role in achieving test coverage for some real-world benchmarks. We believe that mock analysis is a useful prerequisite for test case analysis and development, enabling numerous subsequent analyses.

%% Having described our imperative and declarative approaches to implementing mock analysis, we now comment on the strengths and weaknesses of these two approaches. We hope that our discussion will help future designers of source code analyses and frameworks.

%% \paragraph{Subsequent use of results} Doop is a standalone tool. It depends on other tools to provide input, but provides output in the form of \texttt{csv} files, whose content can be matched to the program source, if a subsequent analysis has the appropriate internal representation. On the other hand, Soot is a compiler framework. Thus, using the Soot analysis results in a subsequent compilation phase is quite easy. Doop works quite well for producing analysis results, and not quite as well for using these results in a compilation process. Our Soot analysis also doesn't need to process the whole program for itself to produce the analysis results that we're interested in here---our intraprocedural analysis can use the existing in-memory representation and pass it on to the next phase, while Doop reads the whole program, throws it away, and leaves nothing for the next compilation phase. 

%% \paragraph{Expressiveness vs concision} In~\cite{bravenboer09:_stric_declar_specif_sophis_point_analy}, Bravenboer and Smaragdakis point out that:
%% \begin{quote}
%% Even conceptually clean program analysis algorithms that
%% rely on mutually recursive definitions often get transformed
%% into complex imperative code for implementation purposes.
%% \end{quote}
%% The presentation of the declarative approach in Section~\ref{sec:technique} could meaningfully include direct excerpts from the Datalog; including Java code is rarely meaningful, as there is too much boilerplate in that language.

%% The declarative approach takes 237 non-comment lines, compared to about 533 non-comment lines for the main part of the imperative approach, which is a significant point in favour of Doop. A head-to-head comparison is tricky, as the imperative approach also uses pre-analyses which are not present in the declarative approach.

%% We comment on the reasons for using helper analyses in the imperative version and not the declarative version. Recall that the helper analyses pre-computed information about 1) mock annotations and 2) constructors and setup methods. The mock annotations are an inessential difference; they could be computed on the fly in the imperative version, as they are in the declarative version. As for the constructors: when thinking imperatively, it is more intuitive to explicitly order the computations for constructors before regular test methods. On the other hand, thinking declaratively, it is more natural to use mutual recursion to declare a dependency on the results of previous computations for fields (our relation \texttt{isCollectionFieldThatContainsMocks} in particular) than to declare an explicit ordering. There is a small semantic difference in the two implementations, as the declarative implementation does not require field writes to be confined to constructors and setup methods; in this particular case, we empirically verified that the imperative assumption was almost always satisfied.

%% We also contrast how we store the abstraction in the two versions. The imperative version uses a standard dataflow analysis abstraction (three bits per local variable/field reference), along with an explicitly specified merge operator, while the declarative version uses one relation for each of the three bits. Propagating and merging data happens automatically in Doop.
%% % There is something going on with doop and kills, but I don't know enough of what's happening to meaningfully comment on it. Intersection-based analyses seem to be possible, because there is e.g. IntraproceduralMustPointTo. There's also something to do with phi nodes and redefinitions, but I can't clearly express it.

%% Another difference between the declarative and imperative versions is in the support for interprocedural analysis. As stated earlier, in Section~\ref{sec:technique}, the declarative version implements a context-insensitive interprocedural analysis while the imperative version is intraprocedural. The choice of intraprocedural versus interprocedural depends strongly on the particular analysis being implemented. Implementing the interprocedural analysis declaratively was impressively easy, while it is significantly more challenging to implement an interprocedural analysis in Soot, requiring the use of Heros~\cite{soap12ifds}, an additional framework. On the other hand, the Heros implementation would be IFDS-based and be context-sensitive; it would be somewhat harder to upgrade our context-insensitive implementat to a context-sensitive Doop implementation.

%% It is easier to add instrumentation, e.g. timers, to the imperative version than the declarative version. Doop contains some built-in timers, but it is unclear how to add new ones.

%% \paragraph{Development velocity}
%% To help the reader calibrate our descriptions, we describe our experience levels with Soot and Doop. One of the two authors has extensive experience with the Soot framework, while the other author started with no experience using Soot. Neither author was familiar with Doop at the beginning of this project. The Soot implementation was developed by the author who was unfamiliar with Soot, while the Doop implementation was developed by the other author. 

%% Soot is a mature program analysis framework and many of the common sticking points have, over the years, been addressed by the developers. Nevertheless, it can be intimidating to start working with Soot. Our experience with Doop is that it is overall robust, yet still being actively developed (i.e. occasionally, at the start, some daily snapshots didn't work with some versions of the underlying Soufflé engine). There is more documentation for Soot than for Doop, although even for Doop, it is often possible to scrape together answers to one's questions from the source code and the online documentation. Finding the right API (or relation, in Doop) to use can be challenging for both Soot and Doop; it's impossible for us to fairly compare them, due to our different experiences with Soot and Doop.

%% Most of the time, adding a feature to the declarative version (e.g. field support) required an evening of work. This typically happened first; the declarative version is better for cleanly describing some approximation of the desired behaviour. Somewhat to our surprise, it was then possible to fast-follow with the imperative version, which ended up not taking much more than an evening to implement either. We believe that the existence of the declarative specification helped with designing the imperative version.

%% The declarative version was still subject to the combinatorial feature interaction problem; for instance, when we added support for fields and containers, we also specifically needed to add support for containers stored in fields.

%% Debugging is an inevitable part of any development process, including this one; declarative languages are no proof against debugging. Some Doop errors were just frustrating, e.g. hardcoding a syntactically incorrect method signature for a collection method. Other times, better type system enforcement in Datalog, and in particular, identifying relations that are unsatisfiable due to type conflicts, would help. Soot errors are typical programming errors.

%% Iteration speed can help with more effective debugging. On some benchmarks, Soot iterations could finish in under a minute, while Doop analysis-only iterations could finish in 10 seconds (but we didn't know that at the time). To expand on that: while developing our analysis, we ran our analysis together with the main analysis, and recomputed the main analysis every time we iterated. Yet, Doop supports running add-on analyses like ours, in isolation, after the main analysis terminates. If we were doing it again, we would develop our analysis as a run-after analysis. Running with the main analysis requires at least a 2.5-minute iteration time due to the necessity of re-compiling and re-running the entire analysis every time the analysis changes, while running an analysis after the main analysis can take 10 seconds, as mentioned above. Setting up the analysis to run after the main analysis is trickier and requires understanding of Doop which we did not have until late in the process. 

%% As stated above, instrumentation is easier in Soot than in Doop, and that extends to printing debug information and using traditional debugging tools, which works as well for Soot as traditional debugging does in general. To debug the Doop analysis, we resorted to outputting relevant relations after a Doop run and manually pinpointing which facts were missing or extraneous. Because Doop uses Soot to generate program facts, understanding Soot in particular and compilers in general was invaluable while developing the Doop implementation---we also looked at the Soot intermediate representation to understand what analysis information was flowing to which intermediate variables.

%% \paragraph{Summary} 
%We would conclude that, especially with the knowledge we have now gained about Doop, prototyping in Doop is easier than in Soot, but that it is no panacea; it remains subject to the feature interaction problem as well as debugging. Additionally, trying to add certain functional behaviours to the Doop implementation, such as timers, can be challenging.

%Finally, we have discussed our experience implementing this analysis twice, and pointed out the benefits and disadvantages of the imperative and declarative approaches for writing static analyses.



% cut here
%% Which base pointer analysis to use in the Doop? Assuming that we're talking about conservative call graphs (over-approximation), you would expect better call graphs to return fewer mocks. We performed experiments with the context-insensitive call graph versus the basic call graph and found XXX.

%% What we'll have for analysis results: context-insensitive intraproc; context-insensitive interproc; basic-only intraproc; basic-only interproc. The difference will be that the virtual method calls in the tests will be better resolved.

%% What we'll have for analysis timings: basic-only without mocks, basic-only with mocks, context-insensitive with mocks


%% \textsc{MockDetector} has demonstrated its capability of correctly identifying and tracing mock objects and containers containing them intra-procedurally in the test cases, in a suite of three benchmarks. It has the potential to be a helper static analysis tool, passing the mock information into existing static analysis frameworks for better call graph analysis. Future work of this project including adding the interprocedual analysis, and gathering results for more benchmarks. 

\section{Math Equations}
You may want to display math equations in three distinct styles:
inline, numbered or non-numbered display.  Each of the three are
discussed in the next sections.

\subsection{Inline (In-text) Equations}
A formula that appears in the running text is called an inline or
in-text formula.  It is produced by the \textbf{math} environment,
which can be invoked with the usual
\texttt{{\char'134}begin\,\ldots{\char'134}end} construction or with
the short form \texttt{\$\,\ldots\$}. You can use any of the symbols
and structures, from $\alpha$ to $\omega$, available in
\LaTeX~\cite{Lamport:LaTeX}; this section will simply show a few
examples of in-text equations in context. Notice how this equation:
\begin{math}
  \lim_{n\rightarrow \infty}x=0
\end{math},
set here in in-line math style, looks slightly different when
set in display style.  (See next section).

\subsection{Display Equations}
A numbered display equation---one set off by vertical space from the
text and centered horizontally---is produced by the \textbf{equation}
environment. An unnumbered display equation is produced by the
\textbf{displaymath} environment.

Again, in either environment, you can use any of the symbols and
structures available in \LaTeX\@; this section will just give a couple
of examples of display equations in context.  First, consider the
equation, shown as an inline equation above:
\begin{equation}
  \lim_{n\rightarrow \infty}x=0
\end{equation}
Notice how it is formatted somewhat differently in
the \textbf{displaymath}
environment.  Now, we'll enter an unnumbered equation:
\begin{displaymath}
  \sum_{i=0}^{\infty} x + 1
\end{displaymath}
and follow it with another numbered equation:
\begin{equation}
  \sum_{i=0}^{\infty}x_i=\int_{0}^{\pi+2} f
\end{equation}
just to demonstrate \LaTeX's able handling of numbering.

\section{Figures}
Your figures should contain a caption which describes the figure to
the reader.

Figure captions are placed {\itshape below} the figure.

Every figure should also have a figure description unless it is purely
decorative. These descriptions convey what’s in the image to someone
who cannot see it. They are also used by search engine crawlers for
indexing images, and when images cannot be loaded.

A figure description must be unformatted plain text less than 2000
characters long (including spaces).  {\bfseries Figure descriptions
  should not repeat the figure caption – their purpose is to capture
  important information that is not already provided in the caption or
  the main text of the paper.} For figures that convey important and
complex new information, a short text description may not be
adequate. More complex alternative descriptions can be placed in an
appendix and referenced in a short figure description. For example,
provide a data table capturing the information in a bar chart, or a
structured list representing a graph.  For additional information
regarding how best to write figure descriptions and why doing this is
so important, please see
\url{https://www.acm.org/publications/taps/describing-figures/}.

\subsection{The ``Teaser Figure''}

A ``teaser figure'' is an image, or set of images in one figure, that
are placed after all author and affiliation information, and before
the body of the article, spanning the page. If you wish to have such a
figure in your article, place the command immediately before the
\verb|\maketitle| command:
\begin{verbatim}
  \begin{teaserfigure}
    \includegraphics[width=\textwidth]{sampleteaser}
    \caption{figure caption}
    \Description{figure description}
  \end{teaserfigure}
\end{verbatim}

\section{Citations and Bibliographies}

The use of \BibTeX\ for the preparation and formatting of one's
references is strongly recommended. Authors' names should be complete
--- use full first names (``Donald E. Knuth'') not initials
(``D. E. Knuth'') --- and the salient identifying features of a
reference should be included: title, year, volume, number, pages,
article DOI, etc.

The bibliogrphy is included in your source document with these two
commands, placed just before the \verb|\end{document}| command:
\begin{verbatim}
  \bibliographystyle{ACM-Reference-Format}
  \bibliography{bibfile}
\end{verbatim}
where ``\verb|bibfile|'' is the name, without the ``\verb|.bib|''
suffix, of the \BibTeX\ file.

Citations and references are numbered by default. A small number of
ACM publications have citations and references formatted in the
``author year'' style; for these exceptions, please include this
command in the {\bfseries preamble} (before the command
``\verb|\begin{document}|'') of your \LaTeX\ source:
\begin{verbatim}
  \citestyle{acmauthoryear}
\end{verbatim}

  Some examples.  A paginated journal article \cite{Abril07}, an
  enumerated journal article \cite{Cohen07}, a reference to an entire
  issue \cite{JCohen96}, a monograph (whole book) \cite{Kosiur01}, a
  monograph/whole book in a series (see 2a in spec. document)
  \cite{Harel79}, a divisible-book such as an anthology or compilation
  \cite{Editor00} followed by the same example, however we only output
  the series if the volume number is given \cite{Editor00a} (so
  Editor00a's series should NOT be present since it has no vol. no.),
  a chapter in a divisible book \cite{Spector90}, a chapter in a
  divisible book in a series \cite{Douglass98}, a multi-volume work as
  book \cite{Knuth97}, a couple of articles in a proceedings (of a
  conference, symposium, workshop for example) (paginated proceedings
  article) \cite{Andler79, Hagerup1993}, a proceedings article with
  all possible elements \cite{Smith10}, an example of an enumerated
  proceedings article \cite{VanGundy07}, an informally published work
  \cite{Harel78}, a couple of preprints \cite{Bornmann2019,
    AnzarootPBM14}, a doctoral dissertation \cite{Clarkson85}, a
  master's thesis: \cite{anisi03}, an online document / world wide web
  resource \cite{Thornburg01, Ablamowicz07, Poker06}, a video game
  (Case 1) \cite{Obama08} and (Case 2) \cite{Novak03} and \cite{Lee05}
  and (Case 3) a patent \cite{JoeScientist001}, work accepted for
  publication \cite{rous08}, 'YYYYb'-test for prolific author
  \cite{SaeediMEJ10} and \cite{SaeediJETC10}. Other cites might
  contain 'duplicate' DOI and URLs (some SIAM articles)
  \cite{Kirschmer:2010:AEI:1958016.1958018}. Boris / Barbara Beeton:
  multi-volume works as books \cite{MR781536} and \cite{MR781537}. A
  couple of citations with DOIs:
  \cite{2004:ITE:1009386.1010128,Kirschmer:2010:AEI:1958016.1958018}. Online
  citations: \cite{TUGInstmem, Thornburg01, CTANacmart}. Artifacts:
  \cite{R} and \cite{UMassCitations}.

\section{Acknowledgments}

Identification of funding sources and other support, and thanks to
individuals and groups that assisted in the research and the
preparation of the work should be included in an acknowledgment
section, which is placed just before the reference section in your
document.

This section has a special environment:
\begin{verbatim}
  \begin{acks}
  ...
  \end{acks}
\end{verbatim}
so that the information contained therein can be more easily collected
during the article metadata extraction phase, and to ensure
consistency in the spelling of the section heading.

Authors should not prepare this section as a numbered or unnumbered {\verb|\section|}; please use the ``{\verb|acks|}'' environment.

\section{Appendices}

If your work needs an appendix, add it before the
``\verb|\end{document}|'' command at the conclusion of your source
document.

Start the appendix with the ``\verb|appendix|'' command:
\begin{verbatim}
  \appendix
\end{verbatim}
and note that in the appendix, sections are lettered, not
numbered. This document has two appendices, demonstrating the section
and subsection identification method.

\section{SIGCHI Extended Abstracts}

The ``\verb|sigchi-a|'' template style (available only in \LaTeX\ and
not in Word) produces a landscape-orientation formatted article, with
a wide left margin. Three environments are available for use with the
``\verb|sigchi-a|'' template style, and produce formatted output in
the margin:
\begin{itemize}
\item {\verb|sidebar|}:  Place formatted text in the margin.
\item {\verb|marginfigure|}: Place a figure in the margin.
\item {\verb|margintable|}: Place a table in the margin.
\end{itemize}

%%
%% The acknowledgments section is defined using the "acks" environment
%% (and NOT an unnumbered section). This ensures the proper
%% identification of the section in the article metadata, and the
%% consistent spelling of the heading.
\begin{acks}
To Robert, for the bagels and explaining CMYK and color spaces.
\end{acks}

%%
%% The next two lines define the bibliography style to be used, and
%% the bibliography file.
\bibliographystyle{ACM-Reference-Format}
\bibliography{bibliograph}

%%
%% If your work has an appendix, this is the place to put it.
\appendix

\section{Research Methods}

\subsection{Part One}

Lorem ipsum dolor sit amet, consectetur adipiscing elit. Morbi
malesuada, quam in pulvinar varius, metus nunc fermentum urna, id
sollicitudin purus odio sit amet enim. Aliquam ullamcorper eu ipsum
vel mollis. Curabitur quis dictum nisl. Phasellus vel semper risus, et
lacinia dolor. Integer ultricies commodo sem nec semper.

\subsection{Part Two}

Etiam commodo feugiat nisl pulvinar pellentesque. Etiam auctor sodales
ligula, non varius nibh pulvinar semper. Suspendisse nec lectus non
ipsum convallis congue hendrerit vitae sapien. Donec at laoreet
eros. Vivamus non purus placerat, scelerisque diam eu, cursus
ante. Etiam aliquam tortor auctor efficitur mattis.

\section{Online Resources}

Nam id fermentum dui. Suspendisse sagittis tortor a nulla mollis, in
pulvinar ex pretium. Sed interdum orci quis metus euismod, et sagittis
enim maximus. Vestibulum gravida massa ut felis suscipit
congue. Quisque mattis elit a risus ultrices commodo venenatis eget
dui. Etiam sagittis eleifend elementum.

Nam interdum magna at lectus dignissim, ac dignissim lorem
rhoncus. Maecenas eu arcu ac neque placerat aliquam. Nunc pulvinar
massa et mattis lacinia.

\end{document}
\endinput
%%
%% End of file `sample-lualatex.tex'.
