\documentclass[sigconf,review,anonymous]{acmart}
\acmConference[ISSTA 2022]{ACM SIGSOFT International Symposium on Software Testing and Analysis}{18-22 July, 2022}{Daejeon, South Korea}
\usepackage{mdframed}
\usepackage{todonotes}
\usepackage{array, booktabs, makecell} % For formal tables
\usepackage{algorithm,algpseudocode}
\usepackage{adjustbox}
\usepackage{graphicx}
\usepackage{textcomp}
\usepackage{xcolor}
\usepackage{listings}
\usepackage{hyperref}
\newcommand{\RomanNumeralCaps}[1]{\MakeUppercase{\romannumeral #1}}
\usepackage{url}
\usepackage{enumitem}
\usepackage{amsmath, amsthm, amsfonts}
\usepackage{tikz}
\usepackage{stfloats}
\usepackage{pifont}
\newcommand{\cmark}{\ding{51}}%
\newcommand{\xmark}{\ding{55}}

%%
%% \BibTeX command to typeset BibTeX logo in the docs
\AtBeginDocument{%
  \providecommand\BibTeX{{%
    \normalfont B\kern-0.5em{\scshape i\kern-0.25em b}\kern-0.8em\TeX}}}

%% Rights management information.  This information is sent to you
%% when you complete the rights form.  These commands have SAMPLE
%% values in them; it is your responsibility as an author to replace
%% the commands and values with those provided to you when you
%% complete the rights form.
%\setcopyright{acmcopyright}
%\copyrightyear{2018}
%\acmYear{2018}
%\acmDOI{10.1145/1122445.1122456}

%% These commands are for a PROCEEDINGS abstract or paper.
%\acmConference[Woodstock '18]{Woodstock '18: ACM Symposium on Neural
%  Gaze Detection}{June 03--05, 2018}{Woodstock, NY}
%\acmBooktitle{Woodstock '18: ACM Symposium on Neural Gaze Detection,
%  June 03--05, 2018, Woodstock, NY}
%\acmPrice{15.00}
%\acmISBN{978-1-4503-XXXX-X/18/06}


%%
%% Submission ID.
%% Use this when submitting an article to a sponsored event. You'll
%% receive a unique submission ID from the organizers
%% of the event, and this ID should be used as the parameter to this command.
%%\acmSubmissionID{123-A56-BU3}

%%
%% The majority of ACM publications use numbered citations and
%% references.  The command \citestyle{authoryear} switches to the
%% "author year" style.
%%
%% If you are preparing content for an event
%% sponsored by ACM SIGGRAPH, you must use the "author year" style of
%% citations and references.
%% Uncommenting
%% the next command will enable that style.
%%\citestyle{acmauthoryear}

%%
%% end of the preamble, start of the body of the document source.
\begin{document}

%%
%% The "title" command has an optional parameter,
%% allowing the author to define a "short title" to be used in page headers.
\title{MockDetector: A technique to identify mock objects created in unit tests}

%%
%% The "author" command and its associated commands are used to define
%% the authors and their affiliations.
%% Of note is the shared affiliation of the first two authors, and the
%% "authornote" and "authornotemark" commands
%% used to denote shared contribution to the research.
\author{Qian Liang}
\email{q8liang@uwaterloo.ca}
\author{Patrick Lam}
\email{patrick.lam@uwaterloo.ca}
\affiliation{%
  \institution{University of Waterloo}
  \city{Waterloo}
  \state{Ontario}
  \country{Canada}
}

%%
%% The abstract is a short summary of the work to be presented in the
%% article.
\begin{abstract}
Unit testing is a widely used tool in modern software development processes. A well-known issue in writing tests is handling dependencies: creating usable objects for dependencies is often complicated. Developers must therefore often introduce mock objects to stand in for dependencies during testing. 

We believe that the static analysis of test suites can enable developers to better understand and maintain existing test suites. However, because mock objects are created using reflection, they confound existing static analysis techniques. At present, it is impossible to statically distinguish methods invoked on mock objects from methods invoked on real objects. 

Researchers have started to build static analyses and developer tools for manipulating test cases. However, such tools currently cannot determine which dependencies' methods are actually tested, versus mock methods being called. As a specific example, removing confounding mock invocations from consideration as focal methods can improve the precision of analyses to detect focal methods under test, which is useful in itself and also a key prerequisite to further analysis of test cases.

In this paper, we introduce MockDetector, a technique to identify mock objects and pinpoint method invocations on mock objects. MockDetector locates common Java mock libraries' APIs for creating mock objects and propagates this information through test cases. Following our observations of tests in the wild, we have added special-case support for arrays and collections holding mock objects. We have built two implementations of MockDetector: a Soot-based imperative dataflow analysis implementation, as well as a Doop-based declarative analysis. On our suite of 8 open-source benchmarks, our imperative intraprocedural approach reported 2,095 invocations on mock objects, whereas our declarative interprocedural approach reported 5,315 invocations on mock objects (under context-insensitve base analyses), out of a total number of 63,017 method invocations in test suites; across benchmarks, mock invocations accounted for a range from 0.086\% to 31.8\% of the total invocations in tests.
	
\end{abstract}


%%
%% The code below is generated by the tool at http://dl.acm.org/ccs.cfm.
%% Please copy and paste the code instead of the example below.
%%
\begin{CCSXML}
<ccs2012>
<concept>
<concept_id>10011007.10011074.10011099.10011102.10011103</concept_id>
<concept_desc>Software and its engineering~Software testing and debugging</concept_desc>
<concept_significance>500</concept_significance>
</concept>
<concept>
<concept_id>10003752.10010124.10010138.10010143</concept_id>
<concept_desc>Theory of computation~Program analysis</concept_desc>
<concept_significance>500</concept_significance>
</concept>
</ccs2012>
\end{CCSXML}

\ccsdesc[500]{Software and its engineering~Software testing and debugging}
\ccsdesc[500]{Theory of computation~Program analysis}


%%
%% Keywords. The author(s) should pick words that accurately describe
%% the work being presented. Separate the keywords with commas.
\keywords{Static Analysis, Dataflow Analysis, Declarative Program Analysis, Mock Objects, Unit Tests}

%%
%% This command processes the author and affiliation and title
%% information and builds the first part of the formatted document.
\maketitle

\section{Introduction}
\label{sec:introduction}

Mock objects~\cite{beck02:_test_driven_devel} are commonly used in
unit tests for object-oriented systems.  They allow developers to test objects that 
rely on other objects, particularly ones that are hard 
to build (e.g. databases; or that come from different components).

While mock objects are an invaluable tool for developers, their use
complicates the static analysis and manipulation of test case source code. 
Such static analyses can help IDEs provide better
support to test case writers; enable better static estimation of test coverage
(avoiding mocks); and detect focal methods in test cases. While researchers have
proposed techniques for automatically generating mocks~\cite{alshahwan10:_autom,fazzini20:_framew_autom_test_mockin_mobil_apps}, our goal here is the opposite:
we detect mocks that already exist in test cases.

Ghafari et al discussed the notion of a focal method~\cite{ghafari15:_autom} for a test case---the method
whose behaviour is being tested---and presented a heuristic for determining focal methods.
By definition, the focal method's receiver object cannot be a mock object.
Ruling out mock invocations can thus improve the accuracy of focal method detection and
enable better understanding of a test case's behaviour.

Mock objects are difficult to analyze statically because, at the bytecode level,
a call to a mock object statically resembles a call to the real object (as
intended by the designers of mock libraries).
A naive static analysis attempting to be sound would have to include all of 
the possible behaviours of the actual object (rather than the mock) when analyzing such code. 
Such potential but unrealizable behaviours obscure the true behaviour 
of the test case.

We have designed a static analysis, \textsc{MockDetector}, which identifies
mock objects in JUnit\footnote{\url{https://junit.org}} test cases. It starts from a list of mock object creation sites
(our analyses include hardcoded APIs for common mocking libraries EasyMock\footnote{\url{https://easymock.org/}}, Mockito\footnote{\url{https://site.mockito.org/}}, and PowerMock\footnote{\url{https://github.com/powermock/powermock}}). 
It then propagates mockness
through the test and identifies invocation sites as (possibly) mock.
Given this analysis result, a subsequent analysis
can ask whether a given variable in a test case contains a mock or not, and
whether a given invocation site is a call to a mock object or not. We have
evaluated \textsc{MockDetector} on a suite of 8 benchmarks plus a microbenchmark. 
We have cross-checked results across the two implementations and manually inspected
the results on our microbenchmark, to ensure that the results are as expected.

Taking a broader view, we believe that helper static analyses like \textsc{MockDetector} 
can aid
in the development of more useful static analyses. These analyses can
encode useful domain properties; for instance, in our case, properties
of test cases. By taking a domain-specific approach, analyses can extract
useful facts about programs that would otherwise be difficult to establish.

We make the following contributions in this paper:
\begin{itemize}
	\item We designed and implemented two variants of a static mock detection algorithm, one as a dataflow analysis implemented imperatively (using Soot) and the other declaratively (using Doop).
	%\item We evaluate both the relative ease-of-implementation and precision of the imperative and declarative approaches, both intraprocedurally and interprocedurally (for Doop). % potentially intraprocedural as well
	\item We characterize our benchmark suite (8 open-source benchmarks, 383kLOC, 184 kLOC tests) with respect to their use of mock objects, finding that 1,084 out of 6,310 unit tests use intraprocedurally-detectable mocks, and that there are a total of 2,095 method invocations on mocks. %We further identify how powerful an analysis is required to identify mock object use---adding fields and collections adds X mock objects, while interprocedural techniques add Y mock objects.
	\item We present potential applications of mock analysis: detecting focal methods, helping to understand test cases, automated refactoring, and API usage extraction.
\end{itemize}
%% At a higher level, we see this paper as making both a contribution and a meta-contribution to
%% problems in source code analysis. The contribution, mock detection, enables more accurate analyses
%% of test cases, which account for a signficant fraction of modern codebases. The meta-contribution,
%% comparing analysis approaches, will help future researchers decide how to best solve their
%% source code analysis problems. In brief, the declarative approach allows users to quickly prototype, stating their properties
%% concisely, while the imperative approach is more amenable to use in program transformation; we return
%% to this question in Section~\ref{sec:discussion}.

We will make all of our artifacts and data publicly available.
%For the purposes of the not-very-well-defined
%required replication package for this submission, see: \url{https://anonymous.4open.science/r/MockAbstraction-B0F0/} or \url{https://figshare.com/s/285073ed5bf5b1e5e7a9}.
% also add somewhere:
% it's notoriously difficult to check static analyses but this is basically N-version programming and we can cross-check results.
% "The N-Version Approach to Fault-Tolerant Software"



\section{motivating-example}
\label{sec:motivating-example}

In this section, we illustrate how our \textsc{MockDetector} tool finds a mock object created within a unit test case. It considers for two scenarios: 1. there is a direct call to mock creation site; 2. the mock object is created through a def-use chain from the mock creation site.

Listing~\ref{lis:direct} shows the unit test case \textit{testSimpleResolution()} in the benchmark byte-buddy/byte-buddy-dep (version 1.7.10) where the mock object \textsc{TypeDescription} is created via a direct call on Java mocking library Mockito's \textit{mock(java.lang.class)}. In this example, our \textsc{MockDetector} tool would first take a list of statements in the unit test processed by Soot~\cite{Vallee-Rai:1999:SJB:781995.782008} and then locate the statements that are instances of Assignment Statement with an invoke expression at the right operand. It then checks if the method invoked matches with any Java mocking libraries' APIs creating a mock object, using the method's subsignature, consists of method name, parameter types, and the return type.

Meanwhile, Listing~\ref{lis:transitive} illustrates the unit test case \textit{testGetIterator()} in the benchmark commons-collections4 (version 4.3), where the array of \textsc{Node}, consists of mock objects created in the helper function \textit{createNodes()}, under this transitive call scenario. In this example with a def-use chain, our tool would first detect the Java mocking library that is in use within the benchmark, and retrieve the corresponding API creating a mock object from the detected Java mocking library. It then utilizes Soot's ReachableMethods with the input of a constructed call graph and the iterator consists of the specific, and checks if any of the statements in the unit test case's body, contains a method invocation that could eventually reach the API. 
 
We would also like to discuss an example where a method is invoked on a mocked object, differentiating it from a method invoked on an actual object in normal settings. In Listing~\ref{lis:mockCall}, the method \textit{addAll()} is invoked on the mocked object ...

\lstset{language=java,
	keywordstyle=\color{blue}\bfseries,
	commentstyle=\color{green},
	stringstyle=\ttfamily\color{red!50!brown},
	showstringspaces=false}‎
\lstset{literate=%
	*{0}{{{\color{red!20!violet}0}}}1
	{1}{{{\color{red!20!violet}1}}}1
	{2}{{{\color{red!20!violet}2}}}1
	{3}{{{\color{red!20!violet}3}}}1
	{4}{{{\color{red!20!violet}4}}}1
	{5}{{{\color{red!20!violet}5}}}1
	{6}{{{\color{red!20!violet}6}}}1
	{7}{{{\color{red!20!violet}7}}}1
	{8}{{{\color{red!20!violet}8}}}1
	{9}{{{\color{red!20!violet}9}}}1
}

\begin{lstlisting}[basicstyle=\ttfamily, caption={This example illustrates a direct call to Mockito's \textit{mock(java.lang.class)} function from test case \textit{testSimpleResolution()}.},
basicstyle=\scriptsize\ttfamily,language = Java, framesep=4.5mm,
framexleftmargin=1mm,,captionpos=b,label=lis:direct]

import static org.mockito.Mockito.mock;

...

@Test
public void testSimpleResolution() throws Exception {
	TypeDescription typeDescription = mock(TypeDescription.class);
	...
}

\end{lstlisting}

\begin{lstlisting}[basicstyle=\ttfamily, caption={This example illustrates a transitive call to EasyMock's \textit{CreateMock(java.lang.class)} function from test case \textit{testGetIterator()}.},
basicstyle=\scriptsize\ttfamily,language = Java, framesep=4.5mm,
framexleftmargin=1mm,,captionpos=b,label=lis:transitive]

private Node[] createNodes() {
	final Node node1 = createMock(Node.class);
	...
}

@Test
public void testGetIterator() {
	...
	final Node[] nodes = createNodes();
	...
}

\end{lstlisting}

\begin{lstlisting}[basicstyle=\ttfamily, caption={This code snippet illustrates an example where the method is invoked on a mocked object in unit test case \textit{addAllForIterable())}},
basicstyle=\scriptsize\ttfamily,language = Java, framesep=4.5mm,
framexleftmargin=1mm,,captionpos=b,label=lis:mockCall]

@Test
public void addAllForIterable() {
	...
	final Collection<Number> c = createMock(Collection.class);
	expect(c.addAll(inputCollection)).andReturn(true);
	...
}

\end{lstlisting}

\section{Research Questions}
\label{sec:rqs}

We now discuss our research questions and the motivation behind them. 

{\bf RQ1} How frequent is mock usage?

{\bf RQ2} To what extent does static analysis on test suites become imprecise due to mocks?

Both our research questions aim to show the importance of differentiating invocations on mock objects from real invocations.

\todo[inline]{describe}

\section{A Survey of Applications}
\label{sec:applications}

Our work supports test-to-code traceability. 
We now sketch several applications of our work to this important problem,
inspired by Ghafari et al~\cite{ghafari15:_autom} but specializing to
our analysis.

\paragraph{Test case comprehension} xUnit tests are snippets of arbitrary
code. In unpublished research, we have established that most tests are
simple straight-line code (although there are still nontrivial incidences of
conditionals and loops). However, this code by necessity interacts with
the system under test in potentially complicated ways. Hence,
understanding what a test case is doing can be difficult. Ghafari et
al described experiments where developers are asked to identify the
focal method, and finds that this is surprisingly difficult;
furthermore, Daka et al~\cite{daka15:_model_readab_improv_unit_tests}
state that reasoning about tests is difficult and propose a model that
enables the automatic generation of tests which are especially
readable.

Our mock analysis helps segregate tests into parts that are
mock-related and parts that are not mock-related. This is directly applicable
to identifying focal methods, as we'll discuss in Section~\ref{sec:focal}.
Also, it is fairly
obvious to a human reader that the part of a test that is calling mock
library methods such as \texttt{thenReturns()} is setting up mock
expectations, but we've seen cases where this is less
obvious. In particular, when recording behaviours with EasyMock,
the developer simply calls methods on the mock object, e.g. 
\texttt{mock.documentAdded("document");}.

Knowing what is a mock can help developers debugging test case
failures get into the right mindset, in two ways: 1) when looking at
mock calls, they can conclude that these mock calls are not
directly causing the test failures; but also, 2) if the mock calls
are now returning incorrect values (perhaps due to program evolution),
then it is likely appropriate to update the recorded values. Similar considerations
apply for automatic debugging and code repair.

\paragraph{Code recommendation}
To amplify the previous point, a key application of test-to-code
traceability is that when the code is updated, related tests may also
need to be updated. Integrated Development Environments can find
all tests referring to a particular fragment of code. Mock analysis
can augment the information available to the developer (or maintainer) by
giving them additional information about whether they are updating
tests that depend on the changed code as a mock or whether the tests
are in fact testing the changed code itself.

\paragraph{Automated refactoring/code generation}
We originally formulated the mock analysis problem as a side problem which needed
to be solved in the service of a deeper problem, which is still
ongoing work: automatically generating certain useful tests. In our context,
we needed to know which call was to the focal method of a test---we
wanted to create additional tests based on those focal methods, but
not based on mock objects.

\paragraph{Extracting API usage examples}
Finally, mock objects can serve as even more concise API usage
examples for the objects being mocked (compared to the test as a whole);
that information could be extracted from the calls to
the mock library. After all, mocks record invocations to methods the objects
being mocked, and the expected behaviours of those objects in response to
these invocations.

Our applications show that the treatment of mocks and non-mocks
when editing test cases (manually or automatically) should be different.
Furthermore, especially for automatic treatments of test cases, a static
analysis determining which invocations are mock invocations is key to effective
treatment.

%* code recommendation for editing test cases
%* automated debugging and repair that rely on failing tests to id repair subjects
%* automated refactoring
%* extracting API usage examples [Ghafari ICPC 2014]


\section{Application: Finding Focal Methods}
\label{sec:focal}

We now discuss another application in greater detail---the detection of focal methods under test
(F-MUTs), as Ghafari et
al~\cite{ghafari15:_autom} call them.
Our analysis facilitates this search.

Ghafari's approach is applicable to tests for classes that implement
stateful objects. It assumes that each test has at least one focal
method---a call to a mutator method for the class under test.
They declare the last such call to be the focal method. However, their approach 
only works on tests of mutator methods.  Purely-functional methods are beyond the scope of their approach.

Another approach uses name-based heuristics, 
as shown by the methods2test
dataset~\cite{tufano2020unit}.
%% This approach uses two heuristics: 1)
%% the name of the test matches the name of the focal method; or, 2) the
%% test calls a unique method in the focal class. 
Rompaey and
Demeyer~\cite{rompaey09:_estab_traceab_links_unit_test} explore the
name-based approach along with simply using the last method called---not necessarily a mutator---as the class containing the focal method.  Romaey and Demeyer also
explore other heuristics, as we discuss in Section~\ref{sec:related}. Ghafari et al point out
that these heuristics have significant limitations, e.g. when a
test case has sub-scenarios.

\paragraph{Example: mock objects and focal methods} We show how we rule out calls by revisiting the unit test case from Listing~\ref{lis:mockCall}. Figure~\ref{fig:mockExampleEvaluation} illustrates finding mock object \texttt{session} and mock invocation \texttt{getRequest()}. In this context, our analysis rules out \texttt{getRequest()} as a focal method. We judge \texttt{getToolchainsForType()} to be the focal method since we assume that every test has at least one focal method, and it is the only method invocation remaining after elimination. Return array \texttt{basics}'s \texttt{length} attribute gets checked in the assertion statement on Line~\ref{line:assert}, which means this test case indeed tests the behaviour of \texttt{getToolchainsForType()}. 

For this unit test, since Ghafari's algorithm does not consider accessors (``inspector methods'' in their paper) as focal methods, they will presumably report no focal method for this unit test case (or any other test cases with no assertions), thus losing recall.

%Incidentally, since Ghafari's heuristic requires tests to have at least one assertion statement, their algorithm will presumably fail to return any focal methods when analyzing unit test cases without assertions.

With the methods2test heuristics, \texttt{getToolChainsForType()} does not appear to match under its name-based heuristic. The other heuristic finds calls to the focal class, so methods2test should identify the call to \texttt{getToolChainsForType()} with that heuristic. Our mock analysis can make that heuristic more precise by removing mocks from the set of eligible focal classes.

%% \begin{table*}
%% 	\centering
%% 	\caption{Comparison of \% of test cases with reported focal methods by the two automated focal method detection algorithms.}
%% 	\vspace*{.5em}
%% 	\resizebox{\textwidth}{!}{\begin{tabular}{lrrrrrlrrrr} \toprule
%% 		\multicolumn{5}{c}{Ghafari's algorithm} & & \multicolumn{5}{c}{methods2test}\\
%% 		\cmidrule{1-5} \cmidrule{7-11}
%% 		\thead{Benchmark} & \thead{Source Code \\ KLoC} & \thead{Reported \\ Focal Methods} & \thead{Test Cases} & \thead{\% of test cases \\ with focal \\ methods detected} &  & \thead{Benchmark} & \thead{Source Code \\ KLoC} & \thead{Reported \\ Focal Methods} & \thead{Test Cases} & \thead{\% of test cases \\ with focal \\ methods detected} \\ 
%% 		\cmidrule{1-5} \cmidrule{7-11}
		
%% 		commons-email-1.3.3 & 8.78 & 90  & 130 &  69\%  &  &    goja-0.1.14/goja-core  & 11.52 & 27  & 80 &  34\% \\
%% 		PureMVC-1.0.8 & 19.46 & 34  & 43 &  79\%  &  &  mock-socket-0.9.0    & 1.09  & 4  & 34 &  12\%      \\
%% 		XStream-1.4.4 & 54.93 & 513  & 968 &  53\%   &  &  project-sunbird-4.3.0/sunbird-lms-service   & 45.36 & 310  & 984 &  31\%   \\
%% 		JGAP-3.4.4  & 73.96 & 1015  & 1390 &  73\%  &  &   optiq-0.8/core    & 93.94  & 26  & 1346 &  2\%   \\
%% 		\bottomrule
%% 		Geometric Mean &   &  &  &  68\% &   &  &  &  &  &  12\% 
%% 	\end{tabular}}
%% 	\label{tab:focal-method-algorithm-comparison}
%% \end{table*}


\begin{figure}[h]
	\begin{lstlisting}[
	numbers=left,numbersep=0pt,basicstyle=\ttfamily\scriptsize,language = Java, framesep=4.5mm, framexleftmargin=1.0mm, captionpos=b, escapechar={|}, mathescape=true, morekeywords={@Test}]
  @Test public void testMisconfiguredToolchain() throws Exception {
    //                mock:|\cmark|    mockAPI:|\cmark|
    MavenSession |\colorbox{olive}{session}| = |\colorbox{teal}{mock}| ( MavenSession.class );
    MavenExecutionRequest req =
        new DefaultMavenExecutionRequest();
    //     mock invocation:|\cmark| $\Rightarrow$ focal method:|\xmark|
    when( session.|\colorbox{red}{getRequest()}| ).thenReturn( req ); |\label{line:Fmock}|

    ToolchainPrivate[] basics =
      //                      focal method:|\cmark|
      toolchainManager.|\colorbox{gray}{getToolchainsForType}|("basic", session); |\label{line:Freal}|

    assertEquals( 0, basics.length );|\label{line:assert}|
  }
  \end{lstlisting}

  \caption{Example: removing mock invocation from focal method consideration.}
  \label{fig:mockExampleEvaluation}
\end{figure}

\paragraph{MockDetector and focal method detection}
We compare the static analysis and name-based approaches to detecting focal methods.
%% Table~\ref{tab:focal-method-algorithm-comparison} presents an approximation to recall---the percentage of test cases reported as having focal methods by each of the approaches. This assumes every test case has exactly one focal method. (A test case may have more than one focal method; we believe that a reasonable test has at least one focal method.) The four projects presented under Ghafari's algorithm are from their paper~\cite{ghafari15:_autom}, whereas we hand-picked four benchmarks from methods2test's dataset to match (as far as possible) those in Ghafari's work.
Data from~\cite{tufano2020unit} suggests that methods2test's name-based approach of focal method detection has low recall. This is not surprising---their heuristics have quite strict constraints, i.e. limited applicability. We reviewed their results and found that none of their reported focal method invocations were mocks, consistent with high precision.

Ghafari kindly shared with us recent focal method detection results. The percentage of test cases where they report focal methods (a proxy to recall) is reasonably high. Sometimes, their tool automatically reports a focal method that is actually a mock invocation from the developer's standpoint---\textsc{MockDetector} would help here. 
%On the other hand, their algorithm would return no focal method for test cases that come with no assertion statements. 
%Overall, Ghafari's algorithm would benefit from filters to increase precision, but 

In short, methods2test's approach finds focal methods with high precision but low recall, whereas Ghafari's algorithm has an acceptable recall but low precision. Our mock analysis can improve focal method detection---it can increase the precision of Ghafari-like methods by ruling out calls that are definitely not focal methods.



\section{Technique}
\label{sec:technique}

We implemented two complementary ways of statically computing mock information: 
a declarative implementation (using Doop~\cite{bravenboer09:_stric_declar_specif_sophis_point_analy}), along with
an imperative implementation of a dataflow analysis (using Soot~\cite{Vallee-Rai:1999:SJB:781995.782008}). While the core analysis implementations are similar, the Soot analysis is much easier for subsequent client analyses to build on. Because the Soot analysis is intraprocedural, it also runs much more quickly. On the other hand, it is easier to experiment with the Doop analysis. For instance, marking a field as mock-containing in Doop was quite easy (we added 3 rules), while the Soot version required us to write new code.
%We started this project with the usual imperative approach to implementing a static analysis---in our context, that meant using Soot. Then, when we wanted to experiment with adding more features to the analysis, we decided that this was a good opportunity to learn about Doop's declarative approach as well. We added new features to the Doop implementation and backported them to the Soot implementation. 
%Although the reason that two implementations exist is unimportant at this stage, the result is consequential:
The existence of these two implementations is a rare opportunity to cross-check static analysis results. We thus carefully compared our results to make sure that they matched.
%Since the focus of this paper is on mock analysis, we will not discuss specific implementation details in depth.
%While the core analysis is similar, the different implementation technologies have different affordances. For instance, it is easier for the Doop version to mark a field as mock-containing (we added 3 rules) than for the Soot version to do so. We start by describing each implementation in turn, and conclude this section with the commonalities between the two implementations. Section~\ref{sec:evaluation} then presents the results obtained using each technology and compares them.

\subsection{Declarative Doop Implementation}
\label{sec:dec-doop}
We next describe the declarative Doop-based technique that \textsc{MockDetector} uses. Both implementations propagate mockness from known mock sources, through the statements in the intermediate representation (IR), to potential mock invocation sites. Doop analyses specify propagation using Datalog rules. 

%We chose to implement a may-analysis rather than a must-analysis for two reasons: 1) we did not observe any cases where a value was assigned a mock on one branch and a real object on the other branch; 2) implementing a must-analysis would not help heuristics to find focal methods, as a must-analysis would rule out fewer mock invocations. 

Adding (context-insensitive) interprocedural support to a Doop analysis is almost trivial: we only needed to add two rules.
Both analyses support arrays, containers, and fields. Although declarative analyses are easier to define than their imperative counterparts, they are still subject to the feature interaction problem: for instance, they still must specifically handle the case where an array-typed field is assigned from a mock-containing array local.

% Mock Libraries discussed in Common Infrastructure section, perhaps refer to the paragraph in Common Infrastructure section?
The implementation declares facts for 9 mock source methods manually gleaned from the mock libraries' documentation, as specified through method signatures (e.g. 
\texttt{<org.mockito.\\Mockito: java.lang.Object mock(java.lang.Class)>}.)
It then declares that a variable {\tt v} satisfies \verb+isMockVar(v)+ if it is assigned from the return value of a mock source, or otherwise traverses the program's interprocedural control-flow graph, through assignments, which may possibly flow through fields, collections, or arrays. Finally, an invocation site is a mock invocation if the receiver object {\tt v} satisfies \verb+isMockVar(v)+.

\begin{lstlisting}[basicstyle=\ttfamily\small,numbers=none,label={lst:core},caption={Selected rules for Datalog mock analysis}]
// v = mock()
isMockVar(v) :-
  AssignReturnValue(mi, v),
  callsMockSource(mi).
// v = (type) from
isMockVar(v) :- isMockVar(from),
  AssignCast(_/* type */, from, v, _/* inmethod */).
// v = v1
isMockVar(v) :- isMockVar(v1),
  AssignLocal(v1, v, _).
\end{lstlisting}

The Doop-provided predicates \texttt{AssignReturnValue}, \\\texttt{AssignCast}, and \texttt{AssignLocal} resemble Java bytecode instructions; for instance, \texttt{AssignLocal(?from:Var, ?to:Var, \\?inmethod:Method)} is an assignment statement copying from variable \texttt{?from} to \texttt{?to} in method \texttt{?inmethod}. (It is Datalog convention to prefix parameters with \texttt{?}s).

We designed the analysis in a modular fashion, such that the interprocedural, collections, arrays, and fields support can all be disabled through the use of \verb+#ifdef+s, which can be specified on the Doop command-line.

\paragraph{Interprocedural support}
As mentioned above, it was quite easy to implement interprocedural analysis in our Doop implementation; we leveraged Doop's call graph and simply propagated mockness information along its call graph edges.

\begin{lstlisting}[basicstyle=\ttfamily\small,numbers=none,caption={Two rules give interprocedural analysis in Doop.}]
// v = callee(), where callee's return var is mock
isInterprocMockVar(v) :-
  AssignReturnValue(mi, v),
  mainAnalysis.CallGraphEdge(_, mi, _, callee),
  ReturnVar(v_callee, callee),
  isMockVar(v_callee).

// callee(v) results in formal
//   param of callee being mock
isInterprocMockVar(v_callee) :- isMockVar(v),
  ActualParam(n, mi, v),
  FormalParam(n, callee, v_callee),
  mainAnalysis.CallGraphEdge(_, mi, _, callee),
  Method_DeclaringType(callee, callee_class),
  ApplicationClass(callee_class).
\end{lstlisting}
We use Doop's call graph edges (\texttt{mainAnalysis.CallGraph\\Edge}) between the method invocation {\tt mi} and its callee {\tt callee}; the first rule propagates information from callees back to their callers, while the second rule propagates information from callers to callees through parameters. We restrict our analysis to so-called ``application classes'', excluding in particular the Java standard library. We chose to run our context-insensitive analysis on top of Doop's context-insensitive call graph. Mirroring Doop itself, it would also be possible to add context sensitivity to our analysis, potentially reducing the number of false positives.%, but our results suggest that this would not help much; we'll return to that point in Section~\ref{sec:evaluation}.

\paragraph{Arrays and Containers} We record local variables pointing to mock-containing arrays using  {\tt isArrayLocalThatContains\\Mocks}. This predicate is true whenever the program under analysis stores a mock variable into an array; we also transfer array-mockness through assignments and casts. When a local variable \texttt{v} is read from a mock-containing array \texttt{c}, then \texttt{v} is marked as a mock variable, as seen in the first rule below. A store of a mock variable \texttt{mv} into an array \texttt{c} causes that array to be \texttt{isArrayLocalThatContainsMocks}. Note that these predicates are mutually-recursive. 

\begin{lstlisting}[basicstyle=\ttfamily\small,numbers=none,caption={Rules for handling arrays.}]
// v = c[idx]
isMockVar(v) :-
  isArrayLocalThatContainsMocks(c),
  LoadArrayIndex(c, v, _ /* idx */).

// c[idx] = mv
isArrayLocalThatContainsMocks(c) :-
  StoreArrayIndex(mv, c, _ /* idx */),
  isMockVar(mv).
\end{lstlisting}

We treat collections analogously. However, while there is one API for arrays---bytecode array load and store instructions---Java's Collections APIs include, by our count, 60 relevant methods. We support iterators, collection copies via constructors, and add-all methods. We use our classification of collection methods to identify collection reads and writes and handle them as we do array reads and writes. We also handle {\tt Collection.toArray} by propagating mockness from the collection to the array.%, except that we say that it is a mock-containing collection, not a mock-containing array.

\paragraph{Fields} Apart from the obvious rule stating that a field which is assigned from a mock satisfies {\tt fieldContainsMock}, we also label fields that have the {\tt org.mockito.Mock} or \\{\tt org.easyMock.Mock} annotations as mock-containing. We declare that a given field \emph{signature} may contain a mock, i.e. the field with a given signature belonging to all objects of a given type. We also support containers stored in fields.

\paragraph{Arrays and Fields} We also support not just array locals but also array fields. That is, when an array-typed field is assigned from a mock-containing array local, then it is also a mock. And when an array-typed local variable is assigned from a mock-containing array field, then that array local is a mock-containing array.

\subsection{Imperative Soot Implementation}
\label{subsec:soot}
We now briefly describe our Soot-based imperative dataflow analysis to find mocks. Like the declarative analysis, we track information from the creation sites through the control-flow graph. Here, we implement a forward dataflow may-analysis---an object is declared a mock if there exists some execution path where it may receive a mock value. The treatment of arrays, fields, and containers mirrors that in the Doop implementation. The dataflow analysis maps values in the IR to three bits: one for the value being a mock, one for it being an array containing a mock, and one for it being a collection containing a mock. %At most one of the three bits may be true for any given value.
Space limitations prevent us from including full details of this implementation, but we will make all implementations available, and the analysis is fully described in the first author's master's thesis\footnote{Citation temporarily omitted due to double-blind review, but available on our institutional repository.}.

% Mock Libraries discussed in Common Infrastructure section
%\paragraph{Define Common Mocking Library APIs}
%\label{subsubsec:collection}

%Our tool stores a pool of common APIs, provided by the analysis designer, which are used to create mock objects when using popular Java mocking libraries, including Mockito, EasyMock, and PowerMock. These APIs are the possible mock creation sites, where the locals/variables holding mock objects are first created.

%Given a pool of possible APIs to search for, our tool may analyze tests for their usage of these APIs.% Facts for mock source methods is discussed in doop, maybe this part could be discussed once to avoid repetition?


%% \paragraph{Forward Dataflow May Analysis}
%% \label{subsubsec:forward}

%To solve the problem, our tool uses forward may analysis, where it analyzes statements from top to bottom, and to keep variables that are verified to be mocks on any possible path at merged points. \textsc{MockDetector} uses the abstraction 
%Not having a mapping in the abstraction is equivalent to mapping to a MockStatus having all three bits false. 
%Our tool would store a value with MockStatus holding three false bits into the abstraction if and only if the value was once a mock or mock-containing container, but later redefined to a non-mock object or container without mock objects.

%Our merge operation is therefore a fairly standard pointwise disjunction of the two incoming values in terms of values and in terms of the 3 bits of \texttt{MockStatus}.

%% \textsc{MockDetector} implements the "may" logic in the following manner: it checks the two in-flows 
%% of \begin{lstlisting}[basicstyle=\ttfamily\small,numbers=none]
%% Map<Value, MockStatus>
%% \end{lstlisting}
%% from two paths. For any variable that is only stored in one map, the key-value pair is directly passed to the out-flow map. For a variable that is a shared key of the two maps, the analysis would update the out-flow's MockStatus by applying the "OR" operation on the "May Mock", "Array Mock", and "Collection Mock" bits from the MockStatus value retrieved from both in-flow maps. 

%% For each statement in a forward flow analysis, we consider two sets: generated set and killed set. In this study, the first set contains the locals that are judged to become mocks, whereas the killed set containing locals that are determined to no longer to be mocks. Equations (1) and (2) illustrates how the inflow and outflow are defined and calculated for each unit: $In(u)$, representing a program point before executing $u$, is the intersection of all outflows after executing each element in immediate predecessor statements of $u$; $Out(u)$, on the other hand, is determined by first removing the killed set from $In(u)$, and union the result with generated set. 

%% \begin{equation}
%% \mathrm{In}(u) = \bigcap_{u' \in preds(u)} \mathrm{Out}(u') 
%% \end{equation}

%% \begin{equation}
%% \mathrm{Out}(u) = (\mathrm{In}(u) - \mathrm{Kill}(u)) \bigcup \mathrm{Gen}(u) 
%% \end{equation}

%Our dataflow analysis uses fairly standard gen and kill sets in the flow function. We set bits in \texttt{MockStatus} as follows:
%For the gen set, we consider for the Values in the following scenarios to be included in the abstraction, with mock bit set to 1:

%% First, the gen set includes pre-analyzed fields containing mock objects defined via annotation (e.g. \texttt{@Mock}), inside a constructor \texttt{<init>}, or in JUnit's \texttt{@Before}/\texttt{setUp()} methods. We discuss the pre-analysis below in Section~\ref{subsubsec:pre-analysis}. 

%% Second, it includes local variables assigned from mock-creation source methods:
%% \begin{lstlisting}[basicstyle=\ttfamily\small,numbers=none]
%%     X x = mock(X);
%% \end{lstlisting}

%% Third, it includes values assigned from return values of read methods from mock-containing collections or arrays:
%% \begin{lstlisting}[basicstyle=\ttfamily\small,numbers=none]
%%     // array read;
%%     // r1 is in the in-set as an array mock
%%     X x = r1[0];
%%     // collection read;
%%     // r2 is in the in-set as a collection mock
%%     X x = r2.get(0);
%% \end{lstlisting}

%% Fourth, if \texttt{x} is a mock and casted and assigned to \texttt{x\_cast}, then the gen set includes \texttt{x\_cast} (e.g. \texttt{r1} in Listing~\ref{lis:arrayIllustrationIR}):
%% \begin{lstlisting}[basicstyle=\ttfamily\small,numbers=none]
%%     // x is a mock in the in-set
%%     X x_cast = (X) x;
%% \end{lstlisting}

%% Finally, the gen set includes copies of already-flagged mocks:
%% \begin{lstlisting}[basicstyle=\ttfamily\small,numbers=none]
%%     // x is a mock in the in-set
%%     X y = x;
%% \end{lstlisting}
%% The copy-related rules also apply to mock-containing arrays and collections. We add some additional rules for generating mocks that the program reads from collections and arrays, as well as rules for marking arrays and collections as mock-containing. For instance, in the below array write, if the in set has \texttt{r2} as a mock, then the destination \texttt{r1} will be generated as a mock-containing array. Similarly, if \texttt{r3} is a known mock, then the collection \texttt{\$r4} to which it is added (the list of collection add methods is hardcoded) will be generated as a mock-containing collection.
%% \begin{lstlisting}[basicstyle=\ttfamily\small,numbers=none]
%%     // r2 is in the in set as a mock
%%     r1[0] = r2;
%%     // r3 is in the in set as a mock
%%     $r4.<java.util.ArrayList: boolean
%%              add(java.lang.Object)>(r3);
%% \end{lstlisting}

% nah, we say that above now.
%% In a similar fashion, the gen set will include values that traverse the program's control-flow graph via assignments, from a value that already has mock-containing array or mock-containing collection bit set to true.

%% \begin{lstlisting}[basicstyle=\ttfamily\small,numbers=none]
%% // arr2 is already in the gen set 
%% // with mock-containing array bit on.
%% arr1 = arr2
%% // arr2 is already in the gen set 
%% // with collection containing mock bit on.
%% collection1 = collection2
%% \end{lstlisting}

%% ---------

%% In our analysis, the generated set consists of two steps. Consider the statement: 
%% \begin{lstlisting}[basicstyle=\ttfamily\small,numbers=none]
%% Employee employee = mock(Employee.class);
%% \end{lstlisting}
%% The intermediate representation generated in Jimple format would be:
%% \begin{lstlisting}[basicstyle=\ttfamily\small,numbers=none]
%% $r1 = staticinvoke <org.mockito.Mockito: 
%% java.lang.Object mock(java.lang.Class)>
%% (class "Lca/liang/Employee;")

%% r2 = (ca.liang.Employee) $r1
%% \end{lstlisting}W

%% In this example, $\$r1$ is the immediate receiver from Mockito's mock creation site, whereas $r2$ is the casted expression that gets carried along in the subsequent program. Thus, our tool would include the immediate receivers, and the casted expressions of mock objects into the generation set, in two steps. 

%% \paragraph{Arrays and containers} At a read from an array into a local variable, where the source array is mock-containing, we declare that the local destination is a mock. At a write of a local variable into an array, where the local variable is mock-containing, we declare that the array is mock-containing.

\paragraph{Interprocedural support} Heros~\cite{bodden12:_inter_proced_data_flow_analy} implements IFDS/IDE for Soot. We explored implementing our analysis in Heros, but this would be more difficult than in Doop, where we simply added two rules to get the interprocedural analysis. We decided that, for our purposes, the extra effort required to formulate our analysis as IFDS would not give us any additional insights; comparing the Doop intraprocedural and interprocedural results sufficed. (Practically speaking, the increased runtime of an IFDS implementation also makes it less viable for subsequent client analyses to use.)

%In particular, Heros uses a different API in its implementation than Soot. Conceptually, though, it should be no harder to implement an interprocedural Heros analysis than an intraprocedural Soot dataflow analysis. 

%With some effort, it would be possible to rewrite our mock analysis with Heros; however, 

%%  Several test suites use arrays or collection objects to hold mock objects. In this scenario, our tool would consider that mockness propagates out to the container. For instance, say the flow function encounters an array write \begin{lstlisting}[basicstyle=\ttfamily\small,numbers=none]
%%     r1[0] = r2
%% \end{lstlisting} 
%% Then, the tool will look for \texttt{r2} in the abstraction. Once the abstraction pinpoints \texttt{r2} (with mock bit on), it will include \texttt{r1} with mock-containing array bit set to true. (THIS PART STILL FEELS REPETITIVE.)


%% Taking an array as an example, our tool would first look for an array reference in the executing statement, meaning there is a read or a write from an array. If the effect is a STORE to the array, \textsc{MockDetector} would look for variables stored into the array, and check whether any of the variables have been found to be mocks. If so, it would label the array as an array mock, and set the relevant bit in the abstraction to true. Reversely, if is a LOAD effect from the array, \textsc{MockDetector} would check if the array itself has mock-containing array bit on in the abstraction. If so, it will mark the value assigned from the array LOAD with mock bit on, and store it in the abstraction.

%% *** we say more about collections in the related work, we should probably move that to here.


%% As stated above, we hardcode all relevant methods from the Java Collections API. There are 60 such methods in total, which together account for about 1/6th of the total lines in our Doop analysis. An iterator can be treated as a copy of the container, with the request of an object from the iterator being tantamount to a container get. An add-all method copies the mock-containing collection bit.


% The main difference is Java's \texttt{Collection} interface has multiple implementations, which expose different APIs for objects. \textsc{MockDetector} resolves this problem with a manually-constructed pool of read and write method APIs associated with each sub-type of the interface \texttt{java.util.Collection}. It subsequently checks (using the hierarchy) whether collection classes appear in statements containing invoke expression. This is achieved by first determining the declaring class of the invoked method. If the declaring class is of an interface, \textsc{MockDetector} would check whether \texttt{java.util.Collection} is a super-interface for the declaring class. Otherwise, if the declaring class is of a class type, \textsc{MockDetector} would check whether \texttt{java.util.Collection} is a super-interface for any of declaring class's implemented interfaces. If a collection sub-type container is presented, \textsc{MockDetector} would then check if a STORE effect is applied to the container, indicating some object is to be stored in the container. Once the object is determined to be a mock, the collection container variable would immediately be labelled as a collection mock, setting the relevant bit in the abstraction to true. Reversely, if a LOAD effect is applied to the container labelled as a collection mock, the object retrieved from the container will be labelled a mock, and setting the mock bit in the abstraction to true.

\subsubsection{Pre-Analyses for Field Mocks Defined in Constructors and Before Methods}
\label{subsubsec:pre-analysis} 
A number of our benchmarks define fields as mock objects via EasyMock or Mockito \texttt{@Mock} annotations, or initialize these fields in the \texttt{<init>} constructor or \texttt{@Before} methods (\textit{setUp()} in JUnit 3), which test runners will execute before any test methods from those classes. These mock field or mock-containing container fields are then used in tests. (The Doop analysis explicitly handles annotations, and the interprocedural rules handle \texttt{@Before} methods). In the Soot implementation, we use two pre-analyses before running the main analysis, under the assumption that fields are rarely mutated in the test cases (and that it is incorrect to do so). We have validated our assumption on benchmarks. An empirical analysis of our benchmarks shows that fewer than 0.3\% of fields (29/9352) are mutated in tests.
%Table~\ref{tab:mutations} shows a preliminary analysis of field mutation frequency inside test cases---fewer than 0.3\% of fields are mutated in test cases.

%% The first pre-analysis handles annotated field mocks and field mocks defined in constructors (\texttt{<init>} methods), while the second handles \texttt{@Before} and \texttt{setUp()} methods. 

% obvious enough that we don't need to belabour this point
% Listing~\ref{lis:annotatedMock} illustrates a Mockito annotated field mock example taken from \texttt{DefaultToolchainManagerTest.java} class in maven-core.

Our analysis retrieves all fields in all test classes, and marks fields annotated {\tt @org.mockito.Mock} or {\tt @org.easymock.Mock} as mocks. Also, Listing~\ref{lis:fieldMock} depicts an example where instance fields are initialized using field initializers. Java copies such initializers into all constructors (\texttt{<init>}). To detect such mock-containing fields, our first pre-analysis applies the dataflow analysis to all constructors in the test classes prior to running the main analysis, using the same logic that we use to detect mock objects or mock-containing containers. The second pre-analysis handles field mocks defined in \texttt{@Before} methods just like the first pre-analysis handled constructors.

%% Listing~\ref{lis:fieldMock2} illustrates an example where fields are defined as mocks via mock-creation source methods inside the @Before method. Similarly, the field mocks initialized inside the @Before methods are determined by applying the forward dataflow analysis strictly on all @Before methods before the main analysis, where the values  


%% \begin{lstlisting}[basicstyle=\ttfamily, caption={Example for Annotated field mocks from \texttt{DefaultToolchainManagerTest.java} in maven-core.},
%% basicstyle=\scriptsize\ttfamily,language = Java, framesep=4.5mm,
%% framexleftmargin=1mm, captionpos=b, label=lis:annotatedMock]
%% public class DefaultToolchainManagerTest
%% {
%%     @Mock
%%     private Logger logger;
%%     @Mock
%%     private ToolchainFactory toolchainFactory_basicType;
%%     @Mock
%%     private ToolchainFactory toolchainFactory_rareType;

%%     @Before
%%     public void onSetup() throws Exception
%%     {    
%%         // ...
%%         MockitoAnnotations.initMocks( this );
%%         // ...
%%     }
%% }
%% \end{lstlisting}

\begin{lstlisting}[basicstyle=\ttfamily, caption={Field mocks defined by field initializations from \texttt{TypeRuleTest} in jsonschema2pojo.},
basicstyle=\scriptsize\ttfamily,language = Java, framesep=4.5mm,
framexleftmargin=1mm, captionpos=b, label=lis:fieldMock, numbers=none]
private GenerationConfig config = mock(GenerationConfig.class);
private RuleFactory ruleFactory = mock(RuleFactory.class);
\end{lstlisting}

% not necessary here. you can add it to your thesis.
%% \begin{lstlisting}[basicstyle=\ttfamily, caption={Example for field mocks defined in a \texttt{@Before} method.},
%% basicstyle=\scriptsize\ttfamily,language = Java, framesep=4.5mm,
%% framexleftmargin=1mm, captionpos=b, label=lis:fieldMock2]
%% public class PayRollMockTest {
%%     private EmployeeDB employeeDB;
%%     private BankService bankService;

%%     @Before
%%     public void init() {
%%         // ...
%%         employeeDB = mock(EmployeeDB.class);
%%         bankService = mock(BankService.class);
%%         // ...
%%     }
%% }
%% \end{lstlisting}

\subsection{Common Infrastructure}
\label{sec:common}
We have parameterized our technique with respect to mocking libraries and have instantiated it with respect to the popular Java libraries Mockito, EasyMock, and PowerMock. We also support JUnit 3 and 4+. %We discuss the parameterization in this subsection.

JUnit and mocking libraries all rely heavily on reflection, and would normally pose problems for static analyses: the set of reachable test methods is enumerated at run-time; mock libraries create mock objects reflectively. Fortunately, their use of reflection is stylized. We designed our analyses to soundly handle these libraries.

\paragraph{JUnit and Driver Generation}
JUnit tests are simply methods that developers write in test classes, appropriately annotated (in JUnit 3 by method name starting with ``test'', in 4+ by a \texttt{@Test} annotation). A JUnit test runner uses reflection to find tests. Out of the box, static analysis engines do not see tests as reachable code.

% what about hierarchical drivers?

Thus, to enable static analysis over a benchmark's test suite, our tool uses Soot to generate a driver class for each Java sub-package of the suite (e.g. \texttt{org.apache.ibatis.executor.statement}). In each of these sub-package driver classes, our tool creates a \textit{runall()} method, which invokes all methods within the sub-package that JUnit considers tests, as well as non-constructor test cases, all surrounded by calls to class-level init/setup and teardown methods. Concrete test methods are particularly easy to call from a driver, as they are specified to have no parameters and are not supposed to rely on any particular ordering. 
Our tool then creates a RootDriver class at the root package level, which invokes the \textit{runall()} method in each sub-package driver class, along with the \texttt{@Test}/\texttt{@Before}/\texttt{@After} methods found in classes located at the root. The drivers that we generate also contain code to catch all checked exceptions declared to be thrown by the unit tests. Both our Soot and Doop implementations use the generated driver classes.

All static frameworks must somehow approximate the set of entry points as appropriate for their targets. For instance, the Wala framework~\cite{wala19:_t} also creates synthetic entry points, but it does this to perform pointer analysis on a program's main code rather than to enumerate the program's test cases.

%% \paragraph{Intraprocedural Analysis} The Soot analysis is intraprocedural and the Doop analysis has an intraprocedural version. Intraprocedurally, we make the unsound (but reasonable for our anticipated use case) assumption that mockness can be dropped between callers and callees: at method entry points, no parameters are assumed to be mocks, and at method returns, the returned object is never a mock. Doop's interprocedural version drops this assumption and instead context-insensitively propagates information from callers to callees and back; we discuss the results in Section~\ref{sec:evaluation}.

%% \paragraph{Mock Libraries}
%% Our supported mock libraries take different approaches to instantiating mocks. All of the libraries have methods that generate mock objects; for instance, EasyMock contains the \texttt{createMock()} method. We consider return values from these methods to be mock objects. Additionally, Mockito contains a fluent \texttt{verify()} method which returns a mock object. Finally, Mockito and EasyMock also allow developers to mark fields as \texttt{@Mock}; we treat reads from such fields as mock objects. Both implementations start the analysis with these hard coded facts on mock source methods, as described in the mock libraries' documentation.

\subsection{Analysis Design Considerations}
\label{subsec:analysis-design}
\paragraph{Static vs Dynamic Analysis}
We chose to implement static analyses to best support integration with our envisioned downstream client analysis; the intraprocedural Soot implementation is most suitable for integration, even if it is less precise. For collecting information about tests, whose behaviour generally does not vary between executions, dynamic analysis would also be quite precise. A dynamic analysis would work well for reporting information to developers, but less well as a feeder analysis to subsequent ones.

\paragraph{Sources of Imprecision}
We explicitly list all known sources of imprecision in our analysis; we investigate their practical impact in Section~\ref{sec:evaluation}.
Approximations may result in false positives (threatening precision), while missed behaviours can cause false negatives (threatening soundness).
\begin{itemize}[noitemsep]
\item At control-flow merges, incoming branches (loops, conditionals) may bring information which must be approximated; our may-analysis could thus report false positives.
\item At method boundaries (formal parameters, return values), our intraprocedural implementations assume that incoming values are not mocks (unless a return value from a Collections API), while the Doop interprocedural implementation uses a context-insensitive approximation. The not-mock assumption may result in false negatives for Soot; extraneous call-graph edges may result in Doop interprocedural false positives. (Lambdas are implemented using method calls and result in intraprocedural false negatives.)
\item Arrays, fields, and collections are approximated such that if any mock is stored in the array, field, or collection, then all values coming from there are mocks; we are subject to false positives if arrays, fields or collections are heterogeneous over time or space.
\item Although we have manually categorized collection APIs and mock APIs, it is possible that our enumeration is incomplete, leading to false negatives.
  Soot also assumes that developers do not call inherited versions of mock creation sites (e.g. a wrapper of the mock source method). Doop's interprocedural analysis does detect such mock creations (fewer than 5).
  Our approach cannot detect when subjects build their own mocks; however, those are more likely to be stubs or fakes~\cite{fowler07:_mocks_arent_stubs} than true mocks. 
  \item As usual, our static analysis approach assumes that it has access to all of the program source; missing code (e.g. dynamically loaded) may result in false negatives.
\end{itemize}

%Call graphs are useful to our intraprocedural analysis because they help identify calls to mock source methods.

\paragraph{Mock Analysis is not just Reaching Definitions}
Our intraprocedural mock analysis is similar to reaching definitions. However, we point out some important differences which motivated our decision to design and implement a bespoke analysis. Even the base analysis requires slightly more than reaching definitions, and then we also support fields, containers and arrays. Our field and array analyses detected additional mocks in half of our benchmarks (Section~\ref{sec:evaluation}). It would not be obvious to support arrays in the reaching definition paradigm; they would require additional support.

For the base analysis, reaching definitions can sometimes indicate whether mock definitions reach uses. However, mock analysis also requires propagating information through casts, as seen in the \texttt{AssignCast} rule in the declarative analysis; at statement {\tt s: x = (X)y;}, then {\tt x} would be a mock iff {\tt y} is. Statement {\tt s} would be a definition for {\tt x}, and a further query would be needed to know whether {\tt y} was a mock or not.

\paragraph{Program Slicing}
Another alternative to reaching definitions for finding mocks is program slicing: to find out whether an invocation {\tt o.f()} is a mock invocation, compute a slice of statements that affect {\tt o.f()}. The usual issue with slicing is that too much of the program is flagged as potentially relevant, as pointed out by Sridharan et al~\cite{sridharan07:_thin_slicin}. They thus proposed thin slicing, which returns a much smaller fraction of the program. Such an approach could work for developers, generalizing the reaching definitions approach alluded to above. Note that the slicing would need to include results through arrays and fields. Even so, it would again be difficult to use slices as information about mocks in a subsequent analysis; our technique instead specifically identifies mock-containing variables and invocations.



\section{Evaluation}
\label{sec:evaluation}

We now quantitatively characterize the performance of our mock analysis implementations on a suite of popular open-source applications. We also provide evidence about the importance of mock invocations to test suites.

\begin{table*}
	\centering
	\caption{Our suite of 8 open-source benchmarks (8,000--117,000 LOC) plus our microbenchmark. Soot and Doop analysis run-times.}
	%	\begin{adjustbox}{width=0.1\textwidth}
	\resizebox{.95\textwidth}{!}{\begin{tabular}{lrrrrrrr}
			\toprule
			Benchmark & Total LOC & Test LOC & \thead{Soot intraproc \\ total (s)} & \thead{Doop intraproc \\ total (s)} & \thead{Soot intraproc \\ mock analysis (s)}  & \thead{Doop intraproc \\ mock analysis (s)} & \thead{Doop interproc \\ mock analysis (s)} \\
			\midrule
			bootique-2.0.B1-bootique         &  15530   & 8595   & 58  & 2810  &  0.276   & 19.93  & 24.90    \\
			commons-collections4-4.4         &  65273   & 36318  & 114 & 694   &  0.386   & 14.20  & 16.64     \\
			flink-core-1.13.0-rc1            &  117310  & 49730  & 341 & 1847  &  0.415   & 27.21  & 62.12      \\
			jsonschema2pojo-core-1.1.1       &  8233    & 2885   & 313 & 1005   &  0.282   & 29.33 & 41.05      \\
			maven-core-3.8.1   		         &  38866   & 11104  & 183 & 588   &  0.276   & 19.49  & 23.42     \\
			micro-benchmark         		 &  954     & 883	& 47  & 387   &  0.130   & 11.73   & 12.92     \\
			mybatis-3.5.6         		  	 &  68268   & 46334  & 500 & 4477  &  0.662   & 59.83  & 192.16      \\
			quartz-core-2.3.1        	  	 &  35355   & 8423   & 155 & 736   &  0.231   & 21.06  & 21.92   \\
			vraptor-core-3.5.5         	  	 &  34244   & 20133  & 371 & 1469  &  0.455   & 34.95  & 149.38    \\
			\bottomrule
			Total         	  				 &  384033  & 184405 & 2082 & 14013 &  3.123  & 237.73  & 544.51    \\
	\end{tabular}}
	%	\end{adjustbox}
	\label{tab:run-times}
\end{table*}

\begin{table*}
	\centering
	\caption{Counts of Test-Related (Test/Before/After) methods in public concrete test classes; counts of mocks, mock-containing arrays, mock-containing collections; and total number of field mock objects reported by Soot intraprocedural analysis.}
	%	\begin{adjustbox}{width=0.1\textwidth}
	\begin{tabular}{lrrrrr}
		\toprule
		Benchmark & \thead{\# of Test-Related \\ Methods} & \thead{\# of Test-Related \\ Methods with \\ mocks (intra)}  & \thead{\# of Test-Related \\ Methods with \\ mock-containing\\ arrays (intra)} & \thead{\# of Test-Related \\ Methods with \\ mock-containing\\ collections (intra)} & \thead{\# of Field \\ Mock Objects} \\
		\midrule
		bootique-2.0.B1-bootique           		&  420        &  32  & 7 & 0     &   8   \\
		commons-collections4-4.4          		&  1152       &  3   & 1 & 1     &   0   \\
		flink-core-1.13.0-rc1           		&  1091       &  4   & 0 & 0     &   0   \\
		jsonschema2pojo-core-1.1.1           	&  145        &  76  & 1 & 0     &   152  \\
		maven-core-3.8.1	           			&  337        &  24  & 0 & 0     &   8    \\
		micro-benchmark         		  		&  59         &  43  & 7 & 25    &   31   \\
		mybatis-3.5.6         		  			&  1769       &  330 & 3 & 0     &   41   \\	
		quartz-core-2.3.1         	  			&  218     	  &  7   & 0 & 0     &   0    \\
		vraptor-core-3.5.5         	  			&  1119       &  565 & 15 & 0    &   474  \\
		\bottomrule
		Total        	  						&  6310       &  1084  & 34 & 26  &  714   \\
	\end{tabular}
	%	\end{adjustbox}
	\label{tab:mocks}
\end{table*}

\begin{table*}
	\centering
	\caption{Total \# of InstanceInvokeExprs, and \# of InstanceInvokeExprs with Mock receivers found by Soot and Doop.}
	%	\begin{adjustbox}{width=0.1\textwidth}
	\begin{tabular}{lrrrr}
		\toprule
		Benchmark & \thead{Total Number \\ of Invocations} & \thead{Mock Invokes \\ intraproc (Soot)}  & \thead{Context-insensitive, \\ intraproc (Doop)} &\thead{Context-insensitive, \\ interproc (Doop)} \\
		\midrule
		bootique-2.0.B1-bootique           		&  3366     &  99  & 99   & 122    \\
		commons-collections4-4.4       			&  12753    &  11   &  3   & 23   \\
		flink-core-1.13.0-rc1           		&  11923    &  40   & 40   & 1389   \\
		jsonschema2pojo-core-1.1.1      	     	&  1896     &  276  & 282  & 604   \\
		maven-core-3.8.1           			&  4072     &  23   & 23   & 39  \\
		microbenchmark         		  		&  471      &  108  & 123  & 132   \\
		mybatis-3.5.6         		  		&  19232    &  575  & 577  & 1345     \\
		quartz-core-2.3.1       	  		&  3436     &  21   & 21   & 31    \\
		vraptor-core-3.5.51        	  		&  5868     &  942  & 962  & 1630   \\
		\bottomrule
		Total        	  				&  63017    & 2095  & 2130  & 5315   \\
	\end{tabular}
	%	\end{adjustbox}
	\label{tab:invokes}
\end{table*}

%\begin{table*}
%	\centering
%	\caption{Counts of Field Mock Objects defined via \protect \texttt{@Mock} annotation, \hspace{\textwidth} in the constructors, and in \texttt{@Before} methods, in each benchmark's test suite.}
%	%	\begin{adjustbox}{width=0.1\textwidth}
%	\vspace*{.5em}
%	\resizebox{1.3\columnwidth}{!}{
%		\begin{tabular}{lrrrr}
%			\toprule
%			Benchmark & \thead{\# of Annotated \\ Field Mock Objects} & \thead{\# of Field Mock Objects \\ defined in the \texttt{<init>} constructor}  & \thead{\# of Field Mock Objects \\ defined in @Before methods} \\
%			\midrule
%			bootique-2.0.B1-bootique           		&  0        &  0    & 8        \\
%			commons-collections4-4.4          		&  0        &  0    & 0        \\
%			flink-core-1.13.0-rc1           		&  0        &  0    & 0        \\
%			jsonschema2pojo-core-1.1.1           	&  26       &  126  & 0        \\
%			maven-core-3.8.1	           			&  7        &  0    & 1        \\
%			micro-benchmark         		  		&  2        &  0    & 29        \\
%			mybatis-3.5.6         		  			&  41       &  0    & 0        \\
%			quartz-core-2.3.1         	  			&  0     	&  0    & 0      \\
%			vraptor-core-3.5.5         	  			&  263      &  128  & 83       \\
%			\bottomrule
%		\end{tabular}
%	}
%	%	\end{adjustbox}
%	\label{tab:field-mocks}
%\end{table*}

%\begin{table*}
%	\centering
%	\caption{Doop analysis-only run-time after basic-only and context-insensitive base analyses. N/A = timed out after 90 minutes.}
%	%	\begin{adjustbox}{width=0.1\textwidth}
%	\begin{tabular}{lrrrrrr}
%		\toprule
%		Benchmark & \thead{Basic-only, \\ intraproc (s)} & \thead{Context-insensitive, \\ intraproc (s)} & \thead{Basic-only, \\ interproc (s)}  & \thead{Context-insensitive, \\ interproc (s)}  \\
%		\midrule
%		bootique-2.0.B1-bootique           		& 15.71  & 16.81 &  24.26    &  20.20     \\
%		commons-collections4-4.4           		& 17.42  & 12.26 &  21.79    &  15.36        \\
%		flink-core-1.13.0-rc1           		& 24.67  & 25.30 &  71.67    &  66.10         \\
%		jsonschema2pojo-core-1.1.1         		& 25.98  & 26.27 &  42.14    &  39.21         \\
%		maven-core-3.8.1   		        	& 18.01  & 16.34 &  25.49    &  22.09          \\
%		micro-benchmark         			& 10.97  & 10.50 &  12.51    &  12.53        \\
%		mybatis-3.5.6         		  		&  N/A   & 51.25 &   N/A     & 183.86          \\
%		quartz-core-2.3.1        	  		& 17.72  & 19.83 &  22.99    &  21.14        \\
%		vraptor-core-3.5.5         	  		& 22.10  & 23.81 &  66.73    & 146.09       \\
%		\bottomrule
%	\end{tabular}
%	%	\end{adjustbox}
%	\label{tab:doop-runtimes}
%\end{table*}



%\begin{table*}
%	\centering
%	\caption{Counts of mock invocations for Doop in basic-only and context-insensitive options, and for interprocedural and intraprocedural .}
%	%	\begin{adjustbox}{width=0.1\textwidth}
%	\begin{tabular}{lrrrr}
%		\toprule
%		Benchmark & \thead{Basic-only, \\ intraproc} & \thead{Context-insensitive, \\ intraproc} & \thead{Basic-only, \\ interproc} & \thead{Context-insensitive, \\ interproc} \\
%		\midrule
%		bootique-2.0.B1-bootique           		&    &    &   &       \\
%		commons-collections4-4.4           		&    &    &   &        \\
%		flink-core-1.13.0-rc1           		&    &    &   &         \\
%		jsonschema2pojo-core-1.1.1         		&    &    &   &          \\
%		maven-core-3.8.1   		           		&    &    &   &           \\
%		micro-benchmark         		  		&    & 	  &   &           \\
%		mybatis-3.5.6         		  			&    &    &   &          \\
%		quartz-core-2.3.1        	  			&    &    &   &         \\
%		vraptor-core-3.5.5         	  			&    &    &   &         \\
%		\bottomrule
%		Total         	  						&    &    &   &         \\
%	\end{tabular}
%	%	\end{adjustbox}
%	\label{tab:doop-mock-invokes}
%\end{table*}

%% The goal of our study is to correctly identify and trace mock objects as well as the method invocations in the test suite. To this end, we conduct quantitative and qualitative research focusing on two research questions:

%% \begin{quote}
%% 	\emph{RQ 1: Are the mocks correctly identified and traced for each test method?}
%% \end{quote}

%% \begin{quote}
%% 	\emph{RQ 2: Would this be helpful for existing static analysis tools?}
%% \end{quote}


\paragraph{Benchmark suite} We evaluated \textsc{MockDetector} on 8 open-source benchmarks, plus a micro-benchmark we developed to test our tool. We ran our experiments on a 32-core Intel(R) Xeon(R) CPU E5-4620 v2 at 2.60GHz with 128GB of RAM running Ubuntu 16.04.7 LTS.

Table~\ref{tab:run-times} presents summary information about our benchmarks and run-times---the LOC and Soot and Doop analysis run-times for each benchmark. The 9 benchmarks include over 383 kLOC, with 184 kLOC in the test suites, per SLOCCount\footnote{\url{https://dwheeler.com/sloccount/}}.  Our benchmarks are from different domains and created by different groups of developers. The Soot total time is the amount of time that it takes for Soot to analyze the benchmark and test suite in whole-program mode, including our analyses. The Soot intraprocedural analysis time is the sum of run-times for the main analysis plus two pre-analyses (Section~\ref{subsec:soot}). The reported Doop run-time is from the context-insensitive analysis, while the Doop analysis time for intraprocedural and interprocedural mock invocation analyses are for running the analysis alone based on recorded facts from the benchmark. The total Doop run-time is much slower than the total Soot run-time because Doop always computes a callgraph---an expensive operation. The Doop analysis-only time is also slower than the Soot time; Doop computes a solution over the entire program, while Soot works one method at a time.
%The major difference between the Doop's total run-time and the actual time spent on mock invocation analysis comes from the build of the complete graph. %Add reference for SLOCCount.

\paragraph{Mock usage in suite} Table~\ref{tab:mocks} presents the number of test-related (Test/Before/After) methods that contain local variables or that access fields that are mocks, mock-containing arrays, or mock-containing collections, as reported by our Soot-based intraprocedural analysis. Table~\ref{tab:mocks} also presents the total number of field mock objects created in each benchmark; 5 out of the 8 open-source benchmarks use field mock objects for ease of testing. Instead of creating the same mock objects in each test case requiring them, these benchmarks choose to create the field mock objects once and use them in all the test cases. The data suggest that our pre-analysis for field mocks is necessary to consequently analyzing for mock objects and mock invocations.

\begin{mdframed}[
	leftmargin=\parindent,
	rightmargin=\parindent,
	skipabove=\topsep,
	skipbelow=\topsep
	]
	{\bf Finding 1:} Across the 8 benchmarks, test-related methods containing local/field mocks or mock-containing containers accounted for 0.35\% to 51.8\% of the total number of test-related methods found in public concrete test classes.
\end{mdframed}

Of our benchmarks, 3 show almost no mock use, 2 show extensive use, and the remainder are in between. Benchmarks \textsc{vraptor-core} and \textsc{jsonschema2pojo-core} have more than half of their test-related methods containing mock objects (and mock-containing arrays); in both of these, most field mocks are created via annotations and reused in multiple test cases. The difference in mock usage reflects their different philosophies and constraints regarding the creation and usage of mock objects in tests.

\subsection{Mock Analysis Results}
Table~\ref{tab:invokes} is the core result about our mock analyses. It presents the detected number of method invocations on mocks. We include numbers from the imperative intraprocedural Soot implementation, as well as intraprocedural and interprocedural versions of the declarative Doop implementation.

\begin{mdframed}[
	leftmargin=\parindent,
	rightmargin=\parindent,
	skipabove=\topsep,
	skipbelow=\topsep
	]
	{\bf Finding 2:} Our intraprocedural analysis finds that method calls on mock objects account for 0.086\% to 16.4\% of the total number of method calls in tests. 
\end{mdframed}

%--- discuss the numbers for the interprocedural analysis.

In Section~\ref{sec:common} we discussed the implementation of our intraprocedural and interprocedural analyses. We can now discuss the effects of these implementation choices on the experimental results. Recall that we chose, unsoundly, to not propagate any information across method calls in the intraprocedural analysis. Thus, the intraproc columns in Table~\ref{tab:invokes} show smaller numbers than the interproc columns, as expected.

There is a sometimes drastic increase from the intraprocedural to the interprocedural result, e.g. from 40 to 1300 for \textsc{flink}. This is because mocks can propagate from tests to the methods that they call and further. Most of the increase is in main code.

To understand better, we manually investigated the 23 additional interprocedural mocks in \textsc{bootique}. Here, our intraprocedural analysis does not miss any mock invocations in tests. Listing~\ref{lis:bootiqueMockCall} illustrates. \texttt{mockParsed.valueOf()} is a mock invocation. \texttt{opts.optionStrings()} is not; there is a real \texttt{opts} object. But in \texttt{optionStrings()} there is a call to method \texttt{optionSet.values()}. Our interprocedural analysis marks \texttt{optionSet} as a mock field when it analyzed the constructor, and then flagged the call to \texttt{optionSet.values()} (in main code) as a mock, while the intraprocedural analysis assumed that the constructor got no mock parameters, and hence does not mark the field as mock.

\begin{lstlisting}[basicstyle=\ttfamily, caption={An interprocedural mock invocation from boutique's \texttt{JoptCliTest}.},
basicstyle=\scriptsize\ttfamily,language = Java, framesep=4.5mm, escapechar=|,
framexleftmargin=1.0mm, captionpos=b, label=lis:bootiqueMockCall, morekeywords={@Test}]
// JoptCliTest, modified to use @Mock vs @Before
@Mock private OptionSet mockParsed;

@Test public void testStringsFor_Missing() {
  when(mockParsed.valueOf(anyString())).
      thenReturn(Collections.emptyList());

  JoptCli opts = new JoptCli(mockParsed, "aname");
  assertNotNull(opts.optionStrings("no_such_opt"));
  /* ... */ }

// JoptCli
private OptionSet optionSet;
private String commandName;

public JoptCli(OptionSet parsed, String commandName) {
  this.optionSet = parsed; this.commandName = commandName; }

@Override public List<String> optionStrings(String name) {
  return /* mock */ optionSet.valuesOf(name).stream()
      .map(String::valueOf).collect(toList()); }
\end{lstlisting}

One false negative in the intraprocedural analysis that we specifically point out is due to lambda expressions. Mockness should propagate to lambdas, but that requires an interprocedural analysis. The Doop interprocedural implementation finds these mocks.

% TODO transitive mocks

\begin{mdframed}[
	leftmargin=\parindent,
	rightmargin=\parindent,
	skipabove=\topsep,
	skipbelow=\topsep
	]
	{\bf Finding 3:} Interprocedural analysis finds from 1.07$\times$ to 34$\times$ more mock invocations than intraprocedural analysis.
\end{mdframed}

We have successfully executed our Doop analysis with different base (pointer) analyses. We reported numbers for the context-insensitive base analysis here as it matches our own mock analysis. (It would also be possible, with much more effort, to adapt our analysis to carry around context.)

\paragraph{Imprecision and Unsoundness in Practice.} In Section~\ref{subsec:analysis-design} we discussed possible sources of imprecision for our static analysis; our analysis could be subject to false positives (loss of precision due to reported mocks that aren't) or false negatives (unsoundness due to unreported mocks). Cross-checking the results revealed that the two approaches have similar precision in detecting intraprocedural mock invokes. We observed some false negatives in the Soot implementation (fewer than 1\%, mostly due to missing support for \texttt{PriorityQueue} and \texttt{TreeSet}). The implementations also differ because Soot does not identify mock invocations in abstract test classes, constructors, or methods not labelled as tests. False negatives common to the two implementations are harder to detect.

After a manual inspection of our \textsc{bootique} benchmark showed zero false positives on that benchmark, we realized that dataflow analysis on test code is likely to have far fewer false positives than on arbitrary code, because it does not have many control-flow merges. We thus instrumented our Soot analysis implementations to count the number of merges where our static analysis must approximate. Among the 1,084 test-related methods in our benchmark suite that contain intraprocedural mocks, only 23 of them have merges (conditionals, loops or try-catch blocks) with different information on the two incoming branches.  {\bf An exhaustive manual inspection of the 23 methods showed that our analysis reports \emph{zero} false positive mock objects or mock invocations due to dataflow merges on our benchmark set.}

We have not quantified imprecision due to arrays, fields, and collections; however, we would be surprised if it were important to a client of our analysis that an object from an array, field, or collection could potentially be either mock or non-mock. Most of the time, such an object should be treated as a mock.


%\paragraph{Field Mocks Results} We perform an evaluation on the necessity of our pre-analyses finding field mocks. 
%Table~\ref{tab:field-mocks} displays the number of field mock objects that are defined via \texttt{@Mock} annotations, in the constructors, and in the \texttt{@Before}/\texttt{setUp()} methods, respectively. 
%
%We focus on the 5 benchmarks that have defined field mock objects. 
%%They are \textsc{bootique}, \textsc{maven-core}, \textsc{jsonschema2pojo-core}, \textsc{mybatis}, and \textsc{vraptor-core}. 
%Among these benchmarks, \textsc{jsonschema2pojo-core}, \textsc{mybatis}, and \textsc{vraptor-core} have a high number (565) or a high percentage (over 50\%) of test-related methods containing mock objects, with many intraprocedural mock invokes. From the results in Table~\ref{tab:field-mocks}, we can tell these benchmarks also define field mock objects instead of repetitively creating the same mock objects in each test, reducing the need for code maintenance. In addition, although \textsc{bootique} and \textsc{maven-core} have lower number of tests using mock objects, these benchmarks still define field mocks. 
%
%\begin{mdframed}[
%	leftmargin=\parindent,
%	rightmargin=\parindent,
%	skipabove=\topsep,
%	skipbelow=\topsep
%	]
%	{\bf Finding 4:} More than half (5/8) of our benchmarks define field mock objects in their tests.
%\end{mdframed}
%We can deduce that our pre-analysis for field mocks described in Section~\ref{subsubsec:pre-analysis} is required for analyzing mocks in  test suites.

\subsection{Application: Mocks contribute to coverage} Since our mock analysis identifies mock objects, we are in a position to empirically evaluate the importance of mock objects in our benchmarks' test suites. One measure of their importance is how much they contribute to code coverage. Using jacoco with the surefire Maven plugin, we measured branch and statement coverage for 4 of our benchmarks (the ones with nontrivial mock usage, excluding one benchmark with parametrized test cases), both with and without test cases that contain intraprocedural mock invocations. We excluded intraprocedural mocks (on a per-test-case granularity) by generating custom Maven test execution commandlines.

\begin{table*}
	\centering
	\caption{Comparison of Statement Coverage and Branch Coverage with all test cases, \hspace{\textwidth} and with only test cases that do not contain intraprocedural mock invocations.}
	\vspace*{.5em}
	\begin{tabular}{lrrrrr} \toprule
		& \multicolumn{2}{c}{Statement Coverage} & & \multicolumn{2}{c}{Branch Coverage} \\
		\cmidrule{2-3} \cmidrule{5-6}
		\thead{Benchmark} & \thead{All Test Cases} & \thead{Test Cases without \\ Intraproc Mocks} & & \thead{All Test Cases} & \thead{Test Cases without \\ Intraproc Mocks} \\ 
		\midrule
		
		jsonschema2pojo-core-1.1.1  & 37\%  & 24\% & & 33\%    &  19\%     \\
		maven-core-3.8.1   		    & 48\%  & 48\% & & 39\%    &  38\%       \\
		mybatis-3.5.6   		    & 85\%  & 81\% & &  82\%    &  76\%        \\
		vraptor-core-3.5.5         	& 87\%  & 59\% & & 81\%   &  56\%    \\
		\bottomrule
	\end{tabular}
	\label{tab:test-coverages}
\end{table*}

Table~\ref{tab:test-coverages} presents our results. We can see that for benchmarks \textsc{jsonschema2pojo-core} and \textsc{vraptor-core}, which have about 15\% mock invocations, statement coverage drops by over 13 to 28 percentage points when excluding intraprocedural mocks. The fact that a relationship exists may be unsurprising, but the magnitude of the 28 point gap of statement coverage for \textsc{vraptor-core} is striking, and points out that the mock objects are indispensable in achieving coverage. We conclude that mock objects play a key role in test suites. Correctly tracing mock objects and invocations could provide better insights about benchmarks relying on mock objects for testing. For instance, with mock objects correctly identified, developers could easily find out which lines of their code are only executed in mock-using test cases. Note that our analysis, or a dynamic version thereof, is required to collect these numbers.

\begin{mdframed}[
	leftmargin=\parindent,
	rightmargin=\parindent,
	skipabove=\topsep,
	skipbelow=\topsep
	]
	{\bf Finding 4:} Test suites use mocks to increase statement coverage by 13--28\% versus not using mocks.
\end{mdframed}

%% \begin{mdframed}[
%% 	leftmargin=\parindent,
%% 	rightmargin=\parindent,
%% 	skipabove=\topsep,
%% 	skipbelow=\topsep
%% 	]
%% 	{\bf Finding 5:} About 2\% of intraprocedural mock-containing test related methods uses branches. % ... to be added
%% \end{mdframed}

%--- , indicates the removal of mock invocations from call graph would improve the call graph's accuracy on method coverage for the benchmarks on the high end of the mock invocation percentage. 

%We explored the performance of our 4 declarative analysis variants based on recorded program facts, and present the numbers in Table~\ref{tab:doop-run-times}. We can observe that the number of mock invocations correlates with the run-time; taking a bit more effort to compute a better call graph may well pay off in terms of overall analysis time. We suspect that the interprocedural analysis is especially slow for mybatis because we also analyze its 50 dependencies; that count is at the high end among our benchmarks.


%% \subsection{Application}
%% \label{subsec:static}

%% There are more test cases holding interprocedural mocks (i.e., the mock object is created in a helper method and passed into the test case) in commons-collections and micro-benchmark. The interprocedural analysis is currently in development and will be discussed in Section~\ref{sec:discussion}.

%% The Procedure Summaries produced after the analysis has indicated that the tracing of "mockiness" of variables and containers is also correct through the whole program. 

%% The accuracy results in tracing intraprocedural mock objects or containers have indicated that \textsc{MockDetector} has the potential to be applied as a helper for existing static analysis tools. By adding proper adjustment, it could pass the mock information to the static analysis, so that the generated call graph may appropriately omit the methods invoked on mock objects, thus increasing its accuracy.

%% By running evaluation (also interprocedurally) on more benchmarks, our tool would have the potential to finding the scenario where developers prefer using mock objects for dependencies, and subsequently providing mock suggestions.


%% context-sensitivity for analysis


\section{Related Work}
\label{sec:related}

We discuss related work in the areas of focal method detection,
% declarative versus imperative static analysis,
treatment of containers, and taint analysis.

\paragraph{Focal methods and classes} To situate Ghafari et al's work~\cite{ghafari15:_autom}, a number of previous works have studied test-to-code traceability by identifying focal \emph{classes} for a test case---the classes which are tested by a test case. The focal \emph{methods} we discuss in this paper belong to focal classes. Ghafari et al were the first to extend the study of traceability to focal methods. Before that, Qusef et al proposed techniques which identify focal classes. In~\cite{DBLP:conf/icsm/QusefBOLB11}, they propose a two-stage approach relying on the assumption that the last assertion statement in a test case is the key assertion. The first stage uses dynamic slicing to find all classes that contribute to the values tested in the assertion (possibly including mocks), while the second stage filters classes and keeps only those textually closest to the test class. An additional mock object filter would help remove definitely-not-focal classes. Earlier work by Qusef et al~\cite{DBLP:conf/icsm/QusefOL10} uses dataflow analysis instead of dynamic slicing.

Rompaey and Demeyer~\cite{rompaey09:_estab_traceab_links_unit_test} evaluate six other heuristics for finding focal classes: naming conventions, types referred to in tests (``fixture element types''), the static call graph, the last call before the assert, lexical analysis of the code and the test, and co-evolution of the test and the main code. No heuristic dominates: different heuristics work for different codebases. Our approach adds another way to rule out unwanted focal method and focal class results.

Ying and Tarr~\cite{DBLP:conf/eclipse/YingT07} also propose heuristics to filter out unwanted methods during code inspection. Their heuristics are based on characteristics of the call graph, i.e. they filter out small methods and methods closer to the bottom of a call graph, depending on tuneable parameters. These heuristics empirically eliminate mock calls in their benchmarks, but there is no principled reason for that to be the case, and indeed, the static call graph that they depend on should not interact well with mock calls.

\paragraph{Treatment of containers} In this work, we use coarse-grained abstractions for containers, consistent with the approach from Chu et al~\cite{chu12:_collec_disjoin_analy}. We do not observe sophisticated container manipulations where it would be necessary to track exactly which elements of a container are mocks. Were such an analysis necessary, the fine-grained container client analysis by Dillig et al~\cite{dillig11:_precis_reason_progr_using_contain} would work.

\paragraph{Taint analysis} Like many other static analyses, our mock analysis can be seen as a variant of a static taint analysis: sources are mock creation methods, while sinks are method invocations. There are no sanitizers in our case. However, for a taint analysis, there is usually a small set of sink methods, while in our case, every method invocation in a test method is a potential sink. In some ways, our analysis resembles an information flow analysis like that by Clark et al~\cite{clark07:_static_analy_quant_infor_flow}. However, the goal of our analysis (detecting possible mocks) is different from taint and information flow analyses in that it is not security-sensitive, so the balance between false positives and false negatives is different---it is less critical to not miss any potential important mock invocations, whereas missing a whole class of tainted methods would often be unacceptable.


%% \paragraph{Imperative vs declarative}
%% Kildall contributed perhaps the first dataflow analysis~\cite{kildall73:_unified_approac_global_progr_optim} as the concept is understood today, describing an algorithm for intraprocedural constant propagation and common subexpression elimination. His algorithm, operating on the program graph, is described in quite imperative pseudocode (and proven to terminate). In some sense, implementing algorithms imperatively is the default, and doesn't need further discussion, except to point out that program analysis frameworks such as Soot~\cite{Vallee-Rai:1999:SJB:781995.782008} provide libraries that can ease the implementation burden.

%% To our knowledge, Corsini et al did some of the first work in declarative program analysis~\cite{corsini93:_effic}; however, that work performed abstract interpretation on (tiny) logic programs rather than imperative programs. Dawson et al~\cite{dawson96:_pract_progr_analy_using_gener} did similar work. Around the same time, Reps proposed~\cite{Reps1995} a declarative analysis to perform demand versions of interprocedural program analyses, which is similar to what we have here; however, we compute all of the analysis results rather than performing a demand analysis. CodeQuest by Hajijev et al~\cite{hajiyev06} also allows developers to perform AST-level code queries using a declarative query language. {\sc Dimple$^+$}~\cite{benton07:_inter_scalab_declar_progr_analy}\cite[Chapter 3]{benton08:_fast_effec_progr_analy_objec_level_paral} by Benton and Fischer may be closest to what we are advocating as the declarative analysis approach. While Benton's dissertation presents a simple {\sc Dimple$^+$} implementation of Andersen's points-to analysis, the {\sc Dimple$^+$} work does not have Doop's sophisticated pointer analysis available to it. Soufflé, by Scholz et al~\cite{scholz16:_fast_large_scale_progr_analy_datal}, advocates for declarative static analysis (but without comparing it directly to an imperative approach as we do here), and presents performance optimizations needed to achieve this goal.
%% Finally, Doop~\cite{bravenboer09:_stric_declar_specif_sophis_point_analy}, which is now primarily implemented with a Soufflé backend, is perhaps the most powerful extant declarative program analysis, and focusses on expressing sophisticated pointer analyses in Datalog. 



%% % implementation note: Reps's approach is much more complicated than what we have in Doop. Perhaps Doop's use of SSA and simulation of phi nodes allows it to use much simpler rules, or maybe it's the specific analyses that are being implemented. e.g. for Doop, which computes an overapproximation, merging the two branches using the virtual phi node (simulated as "x = phi(x1,x2) => x = x1; x = x2") works just fine.

%% In terms of comparing implementations, Prakash et al~\cite{prakash21:_effec_progr_repres_point_analy} compare pointer analysis as provided by Doop and Wala; in some sense, the present work is similar to that work in that both works compare two frameworks. However, that work compares empirical results from two families of pointer analysis implementations (and finds that the specific intermediate representation used doesn't change the results much), while we discuss the process of implementing a static analysis declaratively versus imperatively. Like us, they note that Doop is difficult to incorporate into a program transformation framework (it works better in standalone mode) while Wala's results are readily available; a similar result applies to any result that a Soot-based data flow analysis produces as compared to a Doop-based declarative analysis.




\section{Conclusion}
\label{sec:conclusion}

Our thesis is that mock objects are an important technique that developers use when creating test suites. However, common mock object libraries use reflection and cause developers to write tests that appear to have different behaviours than they actually do---method invocations on mocks look like normal method calls, but instead record behaviour. Because of the prevalence of mocks, tools that work with tests (including static analyses, IDEs, and automatic repair tools) need to correctly handle mock objects. 

We have described our \textsc{MockDetector} static analysis, which we intend to use for further static analyses of test cases. We have implemented \textsc{MockDetector} imperatively in Soot and declaratively in Doop, and characterized its performance and behaviour on a set of 8 open-source benchmarks. Our results show that mocks play an important role in achieving test coverage for some real-world benchmarks. We believe that mock analysis is a useful prerequisite for test case analysis and development, enabling numerous subsequent analyses.

%% Having described our imperative and declarative approaches to implementing mock analysis, we now comment on the strengths and weaknesses of these two approaches. We hope that our discussion will help future designers of source code analyses and frameworks.

%% \paragraph{Subsequent use of results} Doop is a standalone tool. It depends on other tools to provide input, but provides output in the form of \texttt{csv} files, whose content can be matched to the program source, if a subsequent analysis has the appropriate internal representation. On the other hand, Soot is a compiler framework. Thus, using the Soot analysis results in a subsequent compilation phase is quite easy. Doop works quite well for producing analysis results, and not quite as well for using these results in a compilation process. Our Soot analysis also doesn't need to process the whole program for itself to produce the analysis results that we're interested in here---our intraprocedural analysis can use the existing in-memory representation and pass it on to the next phase, while Doop reads the whole program, throws it away, and leaves nothing for the next compilation phase. 

%% \paragraph{Expressiveness vs concision} In~\cite{bravenboer09:_stric_declar_specif_sophis_point_analy}, Bravenboer and Smaragdakis point out that:
%% \begin{quote}
%% Even conceptually clean program analysis algorithms that
%% rely on mutually recursive definitions often get transformed
%% into complex imperative code for implementation purposes.
%% \end{quote}
%% The presentation of the declarative approach in Section~\ref{sec:technique} could meaningfully include direct excerpts from the Datalog; including Java code is rarely meaningful, as there is too much boilerplate in that language.

%% The declarative approach takes 237 non-comment lines, compared to about 533 non-comment lines for the main part of the imperative approach, which is a significant point in favour of Doop. A head-to-head comparison is tricky, as the imperative approach also uses pre-analyses which are not present in the declarative approach.

%% We comment on the reasons for using helper analyses in the imperative version and not the declarative version. Recall that the helper analyses pre-computed information about 1) mock annotations and 2) constructors and setup methods. The mock annotations are an inessential difference; they could be computed on the fly in the imperative version, as they are in the declarative version. As for the constructors: when thinking imperatively, it is more intuitive to explicitly order the computations for constructors before regular test methods. On the other hand, thinking declaratively, it is more natural to use mutual recursion to declare a dependency on the results of previous computations for fields (our relation \texttt{isCollectionFieldThatContainsMocks} in particular) than to declare an explicit ordering. There is a small semantic difference in the two implementations, as the declarative implementation does not require field writes to be confined to constructors and setup methods; in this particular case, we empirically verified that the imperative assumption was almost always satisfied.

%% We also contrast how we store the abstraction in the two versions. The imperative version uses a standard dataflow analysis abstraction (three bits per local variable/field reference), along with an explicitly specified merge operator, while the declarative version uses one relation for each of the three bits. Propagating and merging data happens automatically in Doop.
%% % There is something going on with doop and kills, but I don't know enough of what's happening to meaningfully comment on it. Intersection-based analyses seem to be possible, because there is e.g. IntraproceduralMustPointTo. There's also something to do with phi nodes and redefinitions, but I can't clearly express it.

%% Another difference between the declarative and imperative versions is in the support for interprocedural analysis. As stated earlier, in Section~\ref{sec:technique}, the declarative version implements a context-insensitive interprocedural analysis while the imperative version is intraprocedural. The choice of intraprocedural versus interprocedural depends strongly on the particular analysis being implemented. Implementing the interprocedural analysis declaratively was impressively easy, while it is significantly more challenging to implement an interprocedural analysis in Soot, requiring the use of Heros~\cite{soap12ifds}, an additional framework. On the other hand, the Heros implementation would be IFDS-based and be context-sensitive; it would be somewhat harder to upgrade our context-insensitive implementat to a context-sensitive Doop implementation.

%% It is easier to add instrumentation, e.g. timers, to the imperative version than the declarative version. Doop contains some built-in timers, but it is unclear how to add new ones.

%% \paragraph{Development velocity}
%% To help the reader calibrate our descriptions, we describe our experience levels with Soot and Doop. One of the two authors has extensive experience with the Soot framework, while the other author started with no experience using Soot. Neither author was familiar with Doop at the beginning of this project. The Soot implementation was developed by the author who was unfamiliar with Soot, while the Doop implementation was developed by the other author. 

%% Soot is a mature program analysis framework and many of the common sticking points have, over the years, been addressed by the developers. Nevertheless, it can be intimidating to start working with Soot. Our experience with Doop is that it is overall robust, yet still being actively developed (i.e. occasionally, at the start, some daily snapshots didn't work with some versions of the underlying Soufflé engine). There is more documentation for Soot than for Doop, although even for Doop, it is often possible to scrape together answers to one's questions from the source code and the online documentation. Finding the right API (or relation, in Doop) to use can be challenging for both Soot and Doop; it's impossible for us to fairly compare them, due to our different experiences with Soot and Doop.

%% Most of the time, adding a feature to the declarative version (e.g. field support) required an evening of work. This typically happened first; the declarative version is better for cleanly describing some approximation of the desired behaviour. Somewhat to our surprise, it was then possible to fast-follow with the imperative version, which ended up not taking much more than an evening to implement either. We believe that the existence of the declarative specification helped with designing the imperative version.

%% The declarative version was still subject to the combinatorial feature interaction problem; for instance, when we added support for fields and containers, we also specifically needed to add support for containers stored in fields.

%% Debugging is an inevitable part of any development process, including this one; declarative languages are no proof against debugging. Some Doop errors were just frustrating, e.g. hardcoding a syntactically incorrect method signature for a collection method. Other times, better type system enforcement in Datalog, and in particular, identifying relations that are unsatisfiable due to type conflicts, would help. Soot errors are typical programming errors.

%% Iteration speed can help with more effective debugging. On some benchmarks, Soot iterations could finish in under a minute, while Doop analysis-only iterations could finish in 10 seconds (but we didn't know that at the time). To expand on that: while developing our analysis, we ran our analysis together with the main analysis, and recomputed the main analysis every time we iterated. Yet, Doop supports running add-on analyses like ours, in isolation, after the main analysis terminates. If we were doing it again, we would develop our analysis as a run-after analysis. Running with the main analysis requires at least a 2.5-minute iteration time due to the necessity of re-compiling and re-running the entire analysis every time the analysis changes, while running an analysis after the main analysis can take 10 seconds, as mentioned above. Setting up the analysis to run after the main analysis is trickier and requires understanding of Doop which we did not have until late in the process. 

%% As stated above, instrumentation is easier in Soot than in Doop, and that extends to printing debug information and using traditional debugging tools, which works as well for Soot as traditional debugging does in general. To debug the Doop analysis, we resorted to outputting relevant relations after a Doop run and manually pinpointing which facts were missing or extraneous. Because Doop uses Soot to generate program facts, understanding Soot in particular and compilers in general was invaluable while developing the Doop implementation---we also looked at the Soot intermediate representation to understand what analysis information was flowing to which intermediate variables.

%% \paragraph{Summary} 
%We would conclude that, especially with the knowledge we have now gained about Doop, prototyping in Doop is easier than in Soot, but that it is no panacea; it remains subject to the feature interaction problem as well as debugging. Additionally, trying to add certain functional behaviours to the Doop implementation, such as timers, can be challenging.

%Finally, we have discussed our experience implementing this analysis twice, and pointed out the benefits and disadvantages of the imperative and declarative approaches for writing static analyses.



% cut here
%% Which base pointer analysis to use in the Doop? Assuming that we're talking about conservative call graphs (over-approximation), you would expect better call graphs to return fewer mocks. We performed experiments with the context-insensitive call graph versus the basic call graph and found XXX.

%% What we'll have for analysis results: context-insensitive intraproc; context-insensitive interproc; basic-only intraproc; basic-only interproc. The difference will be that the virtual method calls in the tests will be better resolved.

%% What we'll have for analysis timings: basic-only without mocks, basic-only with mocks, context-insensitive with mocks


%% \textsc{MockDetector} has demonstrated its capability of correctly identifying and tracing mock objects and containers containing them intra-procedurally in the test cases, in a suite of three benchmarks. It has the potential to be a helper static analysis tool, passing the mock information into existing static analysis frameworks for better call graph analysis. Future work of this project including adding the interprocedual analysis, and gathering results for more benchmarks. 

\begin{acks} 
This work was funded in part by Canada's Natural Science and Engineering Research Council.
We thank the Doop developers for their timely and helpful answers to our questions. %; developer or community support is necessary to successfully use Doop.
%(or any research-grade program analysis framework, including Soot.)
We also thank M. Ghafari for his timely response and for access to his focal method results. 
\end{acks}

\bibliographystyle{acm}
\bibliography{bibliography}

\end{document}
\endinput
%%
%% End of file `sample-lualatex.tex'.
