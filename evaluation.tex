\section{Evaluation}
\label{sec:evaluation}

We now quantitatively characterize the performance of our mock analysis implementations on a suite of popular open-source applications. We also provide evidence about the importance of mock invocations to test suites. Section~\ref{sec:focal} discusses an application of our analysis to focal method detection.

\begin{table*}
	\centering
	\caption{Our suite of 8 open-source benchmarks (8,000--117,000 LOC) plus our microbenchmark. Soot and Doop analysis run-times.}
	%	\begin{adjustbox}{width=0.1\textwidth}
	\resizebox{.95\textwidth}{!}{\begin{tabular}{lrrrrrrr}
			\toprule
			Benchmark & Total LOC & Test LOC & \thead{Soot intraproc \\ total (s)} & \thead{Doop intraproc \\ total (s)} & \thead{Soot intraproc \\ mock analysis (s)}  & \thead{Doop intraproc \\ mock analysis (s)} & \thead{Doop interproc \\ mock analysis (s)} \\
			\midrule
			bootique-2.0.B1-bootique         &  15530   & 8595   & 58  & 2810  &  0.276   & 19.93  & 24.90    \\
			commons-collections4-4.4         &  65273   & 36318  & 114 & 694   &  0.386   & 14.20  & 16.64     \\
			flink-core-1.13.0-rc1            &  117310  & 49730  & 341 & 1847  &  0.415   & 27.21  & 62.12      \\
			jsonschema2pojo-core-1.1.1       &  8233    & 2885   & 313 & 1005   &  0.282   & 29.33 & 41.05      \\
			maven-core-3.8.1   		         &  38866   & 11104  & 183 & 588   &  0.276   & 19.49  & 23.42     \\
			micro-benchmark         		 &  954     & 883	& 47  & 387   &  0.130   & 11.73   & 12.92     \\
			mybatis-3.5.6         		  	 &  68268   & 46334  & 500 & 4477  &  0.662   & 59.83  & 192.16      \\
			quartz-core-2.3.1        	  	 &  35355   & 8423   & 155 & 736   &  0.231   & 21.06  & 21.92   \\
			vraptor-core-3.5.5         	  	 &  34244   & 20133  & 371 & 1469  &  0.455   & 34.95  & 149.38    \\
			\bottomrule
			Total         	  				 &  384033  & 184405 & 2082 & 14013 &  3.123  & 237.73  & 544.51    \\
	\end{tabular}}
	%	\end{adjustbox}
	\label{tab:run-times}
\end{table*}

\begin{table*}
	\centering
	\caption{Counts of Test-Related (Test/Before/After) methods in public concrete test classes, along with counts of mocks, mock-containing arrays, mock-containing collections, and total number of field mock objects reported by Soot intraprocedural analysis.}
	%	\begin{adjustbox}{width=0.1\textwidth}
	\begin{tabular}{lrrrrr}
		\toprule
		Benchmark & \thead{\# of Test-Related \\ Methods} & \thead{\# of Test-Related \\ Methods with \\ mocks (intra)}  & \thead{\# of Test-Related \\ Methods with \\ mock-containing\\ arrays (intra)} & \thead{\# of Test-Related \\ Methods with \\ mock-containing\\ collections (intra)} & \thead{\# of Field \\ Mock Objects} \\
		\midrule
		bootique-2.0.B1-bootique           		&  420        &  32  & 7 & 0     &   8   \\
		commons-collections4-4.4          		&  1152       &  3   & 1 & 1     &   0   \\
		flink-core-1.13.0-rc1           		&  1091       &  4   & 0 & 0     &   0   \\
		jsonschema2pojo-core-1.1.1           	&  145        &  76  & 1 & 0     &   152  \\
		maven-core-3.8.1	           			&  337        &  24  & 0 & 0     &   8    \\
		micro-benchmark         		  		&  59         &  43  & 7 & 25    &   31   \\
		mybatis-3.5.6         		  			&  1769       &  330 & 3 & 0     &   41   \\	
		quartz-core-2.3.1         	  			&  218     	  &  7   & 0 & 0     &   0    \\
		vraptor-core-3.5.5         	  			&  1119       &  565 & 15 & 0    &   474  \\
		\bottomrule
		Total        	  						&  6310       &  1084  & 34 & 26  &  714   \\
	\end{tabular}
	%	\end{adjustbox}
	\label{tab:mocks}
\end{table*}

\begin{table*}
	\centering
	\caption{Comparison of Number of InstanceInvokeExprs on Mock objects analyzed by Soot and Doop, \hspace{\textwidth}and Total Number of InstanceInvokeExprs, in each benchmark's test suite.}
	%	\begin{adjustbox}{width=0.1\textwidth}
	\begin{tabular}{lrrrr}
		\toprule
		Benchmark & \thead{Total Number \\ of Invocations} & \thead{Mock Invokes \\ intraproc (Soot)}  & \thead{Context-insensitive, \\ intraproc (Doop)} &\thead{Context-insensitive, \\ interproc (Doop)} \\
		\midrule
		bootique-2.0.B1-bootique           		&  3366     &  99  & 99   & 122    \\
		commons-collections4-4.4       			&  12753    &  11   &  3   & 23   \\
		flink-core-1.13.0-rc1           		&  11923    &  40   & 40   & 1389   \\
		jsonschema2pojo-core-1.1.1      	     	&  1896     &  276  & 282  & 604   \\
		maven-core-3.8.1           			&  4072     &  23   & 23   & 39  \\
		microbenchmark         		  		&  471      &  108  & 123  & 132   \\
		mybatis-3.5.6         		  		&  19232    &  575  & 577  & 1345     \\
		quartz-core-2.3.1       	  		&  3436     &  21   & 21   & 31    \\
		vraptor-core-3.5.51        	  		&  5868     &  942  & 962  & 1630   \\
		\bottomrule
		Total        	  				&  63017    & 2095  & 2130  & 5315   \\
	\end{tabular}
	%	\end{adjustbox}
	\label{tab:invokes}
\end{table*}

%\begin{table*}
%	\centering
%	\caption{Counts of Field Mock Objects defined via \protect \texttt{@Mock} annotation, \hspace{\textwidth} in the constructors, and in \texttt{@Before} methods, in each benchmark's test suite.}
%	%	\begin{adjustbox}{width=0.1\textwidth}
%	\vspace*{.5em}
%	\resizebox{1.3\columnwidth}{!}{
%		\begin{tabular}{lrrrr}
%			\toprule
%			Benchmark & \thead{\# of Annotated \\ Field Mock Objects} & \thead{\# of Field Mock Objects \\ defined in the \texttt{<init>} constructor}  & \thead{\# of Field Mock Objects \\ defined in @Before methods} \\
%			\midrule
%			bootique-2.0.B1-bootique           		&  0        &  0    & 8        \\
%			commons-collections4-4.4          		&  0        &  0    & 0        \\
%			flink-core-1.13.0-rc1           		&  0        &  0    & 0        \\
%			jsonschema2pojo-core-1.1.1           	&  26       &  126  & 0        \\
%			maven-core-3.8.1	           			&  7        &  0    & 1        \\
%			micro-benchmark         		  		&  2        &  0    & 29        \\
%			mybatis-3.5.6         		  			&  41       &  0    & 0        \\
%			quartz-core-2.3.1         	  			&  0     	&  0    & 0      \\
%			vraptor-core-3.5.5         	  			&  263      &  128  & 83       \\
%			\bottomrule
%		\end{tabular}
%	}
%	%	\end{adjustbox}
%	\label{tab:field-mocks}
%\end{table*}

%\begin{table*}
%	\centering
%	\caption{Doop analysis-only run-time after basic-only and context-insensitive base analyses. N/A = timed out after 90 minutes.}
%	%	\begin{adjustbox}{width=0.1\textwidth}
%	\begin{tabular}{lrrrrrr}
%		\toprule
%		Benchmark & \thead{Basic-only, \\ intraproc (s)} & \thead{Context-insensitive, \\ intraproc (s)} & \thead{Basic-only, \\ interproc (s)}  & \thead{Context-insensitive, \\ interproc (s)}  \\
%		\midrule
%		bootique-2.0.B1-bootique           		& 15.71  & 16.81 &  24.26    &  20.20     \\
%		commons-collections4-4.4           		& 17.42  & 12.26 &  21.79    &  15.36        \\
%		flink-core-1.13.0-rc1           		& 24.67  & 25.30 &  71.67    &  66.10         \\
%		jsonschema2pojo-core-1.1.1         		& 25.98  & 26.27 &  42.14    &  39.21         \\
%		maven-core-3.8.1   		        	& 18.01  & 16.34 &  25.49    &  22.09          \\
%		micro-benchmark         			& 10.97  & 10.50 &  12.51    &  12.53        \\
%		mybatis-3.5.6         		  		&  N/A   & 51.25 &   N/A     & 183.86          \\
%		quartz-core-2.3.1        	  		& 17.72  & 19.83 &  22.99    &  21.14        \\
%		vraptor-core-3.5.5         	  		& 22.10  & 23.81 &  66.73    & 146.09       \\
%		\bottomrule
%	\end{tabular}
%	%	\end{adjustbox}
%	\label{tab:doop-runtimes}
%\end{table*}

\begin{table*}
	\centering
	\caption{Comparison of Statement Coverage and Branch Coverage with all test cases, \hspace{\textwidth} and with test cases excluding intraprocedural mock invocations.}
	\vspace*{.5em}
	\begin{tabular}{lrrrrr} \toprule
		& \multicolumn{2}{c}{Statement Coverage} & & \multicolumn{2}{c}{Branch Coverage} \\
		\cmidrule{2-3} \cmidrule{5-6}
		\thead{Benchmark} & \thead{All Test Cases} & \thead{Test Cases without \\ Intraproc Mocks} & & \thead{All Test Cases} & \thead{Test Cases without \\ Intraproc Mocks} \\ 
		\midrule
		
		jsonschema2pojo-core-1.1.1  & 37\%  & 24\% & & 33\%    &  19\%     \\
		maven-core-3.8.1   		    & 48\%  & 48\% & & 39\%    &  38\%       \\
		mybatis-3.5.6   		    & 85\%  & 81\% & &  82\%    &  76\%        \\
		vraptor-core-3.5.5         	& 87\%  & 59\% & & 81\%   &  56\%    \\
		\bottomrule
	\end{tabular}
	\label{tab:test-coverages}
\end{table*}


%\begin{table*}
%	\centering
%	\caption{Counts of mock invocations for Doop in basic-only and context-insensitive options, and for interprocedural and intraprocedural .}
%	%	\begin{adjustbox}{width=0.1\textwidth}
%	\begin{tabular}{lrrrr}
%		\toprule
%		Benchmark & \thead{Basic-only, \\ intraproc} & \thead{Context-insensitive, \\ intraproc} & \thead{Basic-only, \\ interproc} & \thead{Context-insensitive, \\ interproc} \\
%		\midrule
%		bootique-2.0.B1-bootique           		&    &    &   &       \\
%		commons-collections4-4.4           		&    &    &   &        \\
%		flink-core-1.13.0-rc1           		&    &    &   &         \\
%		jsonschema2pojo-core-1.1.1         		&    &    &   &          \\
%		maven-core-3.8.1   		           		&    &    &   &           \\
%		micro-benchmark         		  		&    & 	  &   &           \\
%		mybatis-3.5.6         		  			&    &    &   &          \\
%		quartz-core-2.3.1        	  			&    &    &   &         \\
%		vraptor-core-3.5.5         	  			&    &    &   &         \\
%		\bottomrule
%		Total         	  						&    &    &   &         \\
%	\end{tabular}
%	%	\end{adjustbox}
%	\label{tab:doop-mock-invokes}
%\end{table*}

%% The goal of our study is to correctly identify and trace mock objects as well as the method invocations in the test suite. To this end, we conduct quantitative and qualitative research focusing on two research questions:

%% \begin{quote}
%% 	\emph{RQ 1: Are the mocks correctly identified and traced for each test method?}
%% \end{quote}

%% \begin{quote}
%% 	\emph{RQ 2: Would this be helpful for existing static analysis tools?}
%% \end{quote}


\paragraph{Benchmark suite} We evaluated \textsc{MockDetector} on 8 open-source benchmarks, along with a micro-benchmark we developed to test our tool. We ran our experiments on a 32-core Intel(R) Xeon(R) CPU E5-4620 v2 at 2.60GHz with 128GB of RAM running Ubuntu 16.04.7 LTS.

Table~\ref{tab:run-times} presents summary information about our benchmarks and run-times---the LOC and Soot and Doop analysis run-times for each benchmark. The 9 benchmarks include over 383 kLOC, with 184 kLOC in the test suites, per SLOCCount\footnote{\url{https://dwheeler.com/sloccount/}}.  Our benchmarks are from different domains and created by different groups of developers. The Soot total time is the amount of time that it takes for Soot to analyze the benchmark and test suite in whole-program mode, including our analyses. The Soot intraprocedural analysis time is the sum of run-times for the main analysis plus two pre-analyses (Section~\ref{subsec:soot}). The reported Doop run-time is from the context-insensitive analysis, while the Doop analysis time for intraprocedural and interprocedural mock invocation analyses are for running the analysis alone based on recorded facts from the benchmark. The total Doop run-time is much slower than the total Soot run-time because Doop always computes a callgraph---an expensive operation. The Doop analysis-only time is also slower than the Soot time; Doop computes a solution over the entire program, while Soot works one method at a time.
%The major difference between the Doop's total run-time and the actual time spent on mock invocation analysis comes from the build of the complete graph. %Add reference for SLOCCount.

\paragraph{Mock usage in suite} Table~\ref{tab:mocks} presents the number of test-related (Test/Before/After) methods that contain local variables or that access fields that are mocks, mock-containing arrays, or mock-containing collections, as reported by our Soot-based intraprocedural analysis. Table~\ref{tab:mocks} also presents the total number of field mock objects created in each benchmark, where 5 out of the 8 open-source benchmarks use field mock objects for the ease of testing. Instead of creating the same mock objects in each test case requiring them, these benchmarks choose to create the field mock objects once and use them in all the test cases. The data suggest that our pre-analysis for field mocks is necessary to consequently analyzing for mock objects and mock invocations.

\begin{mdframed}[
	leftmargin=\parindent,
	rightmargin=\parindent,
	skipabove=\topsep,
	skipbelow=\topsep
	]
	{\bf Finding 1:} Across the 8 benchmarks, test-related methods containing local/field mocks or mock-containing containers accounted for 0.35\% to 51.8\% of the total number of test-related methods found in public concrete test classes.
\end{mdframed}

Of our benchmarks, 3 show almost no mock use, 2 show extensive use, and the remainder are in between. Benchmarks \textsc{vraptor-core} and \textsc{jsonschema2pojo-core} have more than half of their test-related methods containing mock objects (and mock-containing arrays); in both of these, most field mocks are created via annotations and reused in multiple test cases. The difference in mock usage reflects their different philosophies and constraints regarding the creation and usage of mock objects in tests.

\subsection{Mock Analysis Results}
Table~\ref{tab:invokes} is the core result about our mock analyses. It presents the detected number of method invocations on mocks. We include numbers from the imperative intraprocedural Soot implementation, as well as intraprocedural and interprocedural versions of the declarative Doop implementation.

\begin{mdframed}[
	leftmargin=\parindent,
	rightmargin=\parindent,
	skipabove=\topsep,
	skipbelow=\topsep
	]
	{\bf Finding 2:} Our intraprocedural analysis finds that method calls on mock objects account for 0.086\% to 16.4\% of the total number of calls. 
\end{mdframed}

%--- discuss the numbers for the interprocedural analysis.

In Section~\ref{sec:common} we discussed the implementation of our intraprocedural and interprocedural analyses. We can now discuss the effects of these implementation choices on the experimental results. Recall that we chose, unsoundly, to not propagate any information across method calls in the intraprocedural analysis. Thus, the intraproc columns in Table~\ref{tab:invokes} show smaller numbers than the interproc columns, as expected. Note also that there is a sometimes drastic increase from the intraprocedural to the interprocedural result, e.g. from 40 to 1300 for \textsc{flink}. This is because mocks can propagate from tests to the methods that they call and all around the main program code. It would be desirable to be able to differentiate test helper methods, which we want to propagate mocks to, from methods in the main program, which we generally do not want to propagate mocks to; however, Doop treats test and main code identically.

One type of false negative in the intraprocedural analysis that we specifically point out is due to lambda expressions. Mockness should propagate to lambda expressions but that requires an interproecdural analysis. The Doop interprocedural implementation finds these mocks.

\begin{mdframed}[
	leftmargin=\parindent,
	rightmargin=\parindent,
	skipabove=\topsep,
	skipbelow=\topsep
	]
	{\bf Finding 3:} Interprocedural analysis finds from 1.07$\times$ to 34$\times$ more mock invocations than intraprocedural analysis, but that number includes potential mock calls in the main program.
\end{mdframed}

We have successfully executed our Doop analysis with different base (pointer) analyses. We reported numbers for the context-insensitive base analysis here as it matches our own mock analysis. (It would also be possible, with much more effort, to adapt our analysis to carry around context.)

\paragraph{Validation} In Section~\ref{subsec:analysis-design} we discussed possible sources of imprecision for our static analysis; our analysis could be subject to false positives (loss of precision due to reported mocks that aren't) or false negatives (unsoundness due to unreported mocks). Cross-checking the results revealed that the two approaches have similar precision in detecting intraprocedural mock invokes. We observed some false negatives in the Soot implementation (fewer than 2\%, mostly due to missing support for certain array-/collection-related functions). The implementations also differ because Soot does not identify mock invocations in abstract test classes, constructors, or methods not labelled as tests. False negatives common to the two implementations are harder to detect.

After a manual inspection of our \textsc{bootique} benchmark showed zero false positives on that benchmark, we realized that dataflow analysis on test code is likely to have far fewer false positives than on arbitrary code, because it does not have many control-flow merges. We thus instrumented our Soot analysis implementations to count the number of merges where our static analysis must approximate. Among the 1,084 test-related methods in our benchmark suite that contain intraprocedural mocks, only 23 of them have merges (conditionals, loops or try-catch blocks) with different information on the two incoming branches.  An exhaustive manual inspection of the 23 methods showed that our analysis reports \emph{zero} false positive mock objects or mock invocations due to dataflow merges on our benchmark set.


%\paragraph{Field Mocks Results} We perform an evaluation on the necessity of our pre-analyses finding field mocks. 
%Table~\ref{tab:field-mocks} displays the number of field mock objects that are defined via \texttt{@Mock} annotations, in the constructors, and in the \texttt{@Before}/\texttt{setUp()} methods, respectively. 
%
%We focus on the 5 benchmarks that have defined field mock objects. 
%%They are \textsc{bootique}, \textsc{maven-core}, \textsc{jsonschema2pojo-core}, \textsc{mybatis}, and \textsc{vraptor-core}. 
%Among these benchmarks, \textsc{jsonschema2pojo-core}, \textsc{mybatis}, and \textsc{vraptor-core} have a high number (565) or a high percentage (over 50\%) of test-related methods containing mock objects, with many intraprocedural mock invokes. From the results in Table~\ref{tab:field-mocks}, we can tell these benchmarks also define field mock objects instead of repetitively creating the same mock objects in each test, reducing the need for code maintenance. In addition, although \textsc{bootique} and \textsc{maven-core} have lower number of tests using mock objects, these benchmarks still define field mocks. 
%
%\begin{mdframed}[
%	leftmargin=\parindent,
%	rightmargin=\parindent,
%	skipabove=\topsep,
%	skipbelow=\topsep
%	]
%	{\bf Finding 4:} More than half (5/8) of our benchmarks define field mock objects in their tests.
%\end{mdframed}
%We can deduce that our pre-analysis for field mocks described in Section~\ref{subsubsec:pre-analysis} is required for analyzing mocks in  test suites.

\subsection{Application: Mocks contribute to coverage} Since our mock analysis identifies mock objects, we are in a position to empirically evaluate the importance of mock objects in our benchmarks' test suites. One measure of their importance is how much they contribute to code coverage. Using jacoco with the surefire Maven plugin, we measured branch and statement coverage for 4 of our benchmarks, both with and without test cases that contain intraprocedural mock invocations. We excluded intraprocedural mocks by generating custom Maven test execution commandlines.

Table~\ref{tab:test-coverages} presents our results. We can see that for benchmarks \textsc{jsonschema2pojo-core} and \textsc{vraptor-core}, which have about 15\% mock invocations, statement coverage drops by over 13 to 28 percentage points when excluding intraprocedural mocks. The fact that a correlation exists may be unsurprising, but the magnitude of the 28 point gap of statement coverage for \textsc{vraptor-core} is striking, and points out that the mock objects are pulling more than their weight in terms of coverage. These coverage numbers suggest that mock objects may play a key role in test suites. Correctly tracing mock objects and invocations could provide better insights about benchmarks relying on mock objects for testing. For instance, with mock objects correctly identified, developers could easily find out which lines of their code are only executed in mock-using test cases. Note that our analysis, or a dynamic version thereof, is required to collect these numbers.

\begin{mdframed}[
	leftmargin=\parindent,
	rightmargin=\parindent,
	skipabove=\topsep,
	skipbelow=\topsep
	]
	{\bf Finding 4:} Test suites use mocks to increase statement coverage by 13--28\% versus not using mocks.
\end{mdframed}

%% \begin{mdframed}[
%% 	leftmargin=\parindent,
%% 	rightmargin=\parindent,
%% 	skipabove=\topsep,
%% 	skipbelow=\topsep
%% 	]
%% 	{\bf Finding 5:} About 2\% of intraprocedural mock-containing test related methods uses branches. % ... to be added
%% \end{mdframed}

%--- , indicates the removal of mock invocations from call graph would improve the call graph's accuracy on method coverage for the benchmarks on the high end of the mock invocation percentage. 

%We explored the performance of our 4 declarative analysis variants based on recorded program facts, and present the numbers in Table~\ref{tab:doop-run-times}. We can observe that the number of mock invocations correlates with the run-time; taking a bit more effort to compute a better call graph may well pay off in terms of overall analysis time. We suspect that the interprocedural analysis is especially slow for mybatis because we also analyze its 50 dependencies; that count is at the high end among our benchmarks.


%% \subsection{Application}
%% \label{subsec:static}

%% There are more test cases holding interprocedural mocks (i.e., the mock object is created in a helper method and passed into the test case) in commons-collections and micro-benchmark. The interprocedural analysis is currently in development and will be discussed in Section~\ref{sec:discussion}.

%% The Procedure Summaries produced after the analysis has indicated that the tracing of "mockiness" of variables and containers is also correct through the whole program. 

%% The accuracy results in tracing intraprocedural mock objects or containers have indicated that \textsc{MockDetector} has the potential to be applied as a helper for existing static analysis tools. By adding proper adjustment, it could pass the mock information to the static analysis, so that the generated call graph may appropriately omit the methods invoked on mock objects, thus increasing its accuracy.

%% By running evaluation (also interprocedurally) on more benchmarks, our tool would have the potential to finding the scenario where developers prefer using mock objects for dependencies, and subsequently providing mock suggestions.


%% context-sensitivity for analysis
