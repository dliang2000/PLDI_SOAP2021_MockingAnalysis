\section{Motivating Example}
\label{sec:motivating-example}

\textsc{MockDetector} finds variables containing mock objects, along with invocations on mock objects, in unit tests. It does so by identifying invocations on variables which have been assigned an object flowing from a mock creation site either using a forward dataflow may-analysis (Soot) or by solving declarative constraints (Doop).

\begin{lstlisting}[basicstyle=\ttfamily, caption={Code from maven-core; calls to focal method \texttt{getToolchainsForType()} and to mock \texttt{session}'s \texttt{getRequest()} method occur in \textit{testMisconfiguredToolchain()}.},
numbers=left,numbersep=2pt,basicstyle=\scriptsize\ttfamily,language = Java, framesep=4.5mm, escapechar=|,
framexleftmargin=1.0mm, captionpos=b, label=lis:mockCall, morekeywords={@Test}]
@Test public void testMisconfiguredToolchain() throws Exception {
  MavenSession session = mock( MavenSession.class );
  MavenExecutionRequest req = new DefaultMavenExecutionRequest();
  when( session.getRequest() ).thenReturn( req ); |\label{line:mock}|
  ToolchainPrivate[] basics =
    toolchainManager.getToolchainsForType("basic", session); |\label{line:real}|
  assertEquals( 0, basics.length );
}
\end{lstlisting}

To motivate our work, consider Listing~\ref{lis:mockCall}, which presents a unit test from the Maven project. Line~\ref{line:mock} calls \textit{getRequest()}, invoking it on mock object \texttt{session}. Line~\ref{line:real} then calls \textit{getToolchainsForType()}, which happens to be the focal method whose behaviour is being tested here. At the bytecode level, the two calls are indistinguishable with respect to mockness; the static call graph contains a call to the actual method in both cases, so that current static analysis tools cannot easily tell the difference between the method invocation on a mock object on line~\ref{line:mock} and the method invocation on a real object on line~\ref{line:real}. This uncertainty about mockness confounds a naive static analysis that attempts to identify focal methods. For instance, Ghafari et al~\cite{ghafari15:_autom}'s heuristic would fail on this test, as it returns the last mutator method in the object under test, and the focal method here is an accessor. Section~\ref{sec:focal} discusses applications of mock analysis to finding focal methods in more detail. Mockness information can also help IDEs provide better suggestions. 


\paragraph{Basic dataflow analysis} The dataflow analysis-based \textsc{MockDetector} implementation keeps an abstraction mapping values (local variables or field references) in Soot's Jimple intermediate representation (IR)~\cite{Vallee-Rai:1999:SJB:781995.782008} to \texttt{MockStatus}, which holds three bits monitoring each value's status (a mock, a mock-containing array, or a mock-containing collection). %The declarative version maintains relations which track the same bits.

\begin{figure}[H]
	\begin{lstlisting}[basicstyle=\ttfamily,
	numbers=left,numbersep=0pt,basicstyle=\scriptsize\ttfamily,language = Java, framesep=4.5mm, framexleftmargin=1.0mm, captionpos=b, escapechar=|, morekeywords={@Test}]
	//      mock: |\xmark~\,|  mockAPI: |\xmark|
	Object object1 = new Object();
	
	// mock: |\xmark|
	object1.foo();
	
	//      mock: |\cmark|   mockAPI: |\cmark|
	Object object2 = mock(Object.class);
	
	// mock: |\cmark|
	object2.foo();
	\end{lstlisting}
	%    \includegraphics[width=.25\textwidth]{Images/mockInvocationIllustration.png}	
	\caption{Our static analysis propagates mockness from sources (e.g. \texttt{mock(Object.class}) to invocations.}
	\label{fig:mockMethodIllustration}
	
\end{figure}

\begin{lstlisting}[basicstyle=\ttfamily, caption={Jimple Intermediate Representation for the code in Figure~\ref{fig:mockMethodIllustration}.},
numbers=left,numbersep=2pt,basicstyle=\scriptsize\ttfamily, captionpos=b, label=lis:mockMethodIllustrationIR, escapechar=|, morekeywords={@Test, specialinvoke, virtualinvoke, staticinvoke}]
java.lang.Object $r1, r2;

$r1 = new java.lang.Object; |\label{line:lis3line3}|
specialinvoke $r1.<java.lang.Object: void <init>()>(); |\label{line:lis3line4}|
virtualinvoke $r1.<java.lang.Object: void foo()>(); |\label{line:lis3line5}|
r2 = staticinvoke <org.mockito.Mockito: java.lang.Object
	mock(java.lang.Class)> (class "Ljava/lang/Object;"); |\label{line:lis3line6}|
virtualinvoke r2.<java.lang.Object: void foo()>(); |\label{line:lis3line9}|
\end{lstlisting}


Figure~\ref{fig:mockMethodIllustration} shows how our dataflow analysis works; Listing~\ref{lis:mockMethodIllustrationIR} shows the Jimple IR of the code in Figure~\ref{fig:mockMethodIllustration}. At the top of the IR, we begin with an empty abstraction (no mapping for any values, equivalent to all bits false everywhere) before line~\ref{line:lis3line3}. For the creation of \texttt{\$r1} on lines~\ref{line:lis3line3} and~\ref{line:lis3line4}, since the call to the no-arg \texttt{<init>} constructor is not one of our hardcoded mock APIs, our analysis does not declare \texttt{\$r1} to be a mock object. 
%In practice, our abstraction simply does not create an explicit binding for \texttt{\$r1}, instead leaving the mapping empty as it was prior to line~\ref{line:lis3line3}, tantamount to creating a new \texttt{MockStatus} with all bits false and binding it to \texttt{\$r1}. 
Thus, invocation \texttt{object1.foo()} on line 5 in Figure~\ref{fig:mockMethodIllustration} is not known to be a mock invocation. Tying back to our focal methods application, we would not exclude the call to \texttt{foo()} from being a possible focal method.

On the other hand, our imperative analysis sees a call to the mock creation API \texttt{<org.mockito.Mockito: java.lang.Object \\ mock(java.lang.Class)>} on line~\ref{line:lis3line6} in the Jimple IR. It thus adds a mapping from local variable \texttt{r2} to a new \texttt{MockStatus} with mock bit set to true. Because \texttt{r2} has a mapping in the abstraction with mock bit set, \textsc{MockDetector} will deduce that the call on line~\ref{line:lis3line9} is a mock invocation. This implies that the call to method \textit{foo()} on line 11 in Figure~\ref{fig:mockMethodIllustration} cannot be a focal method.


\paragraph{Basic declarative analysis} Again referring to IR Listing~\ref{lis:mockMethodIllustrationIR}, this time we ask whether the call on IR line~\ref{line:lis3line9} satisfies the predicate \texttt{isMockInvocation} (facts Listing~\ref{lis:facts}, line~\ref{line:facts-imi}), which we define to hold the analysis result (all mock invocation sites in the program). It does, because of facts lines~\ref{line:facts-vmi}--\ref{line:facts-imv}: IR line~\ref{line:lis3line9} contains a virtual method invocation, and receiver \texttt{r2} for the invocation on that line satisfies our predicate \texttt{isMockVar}, which holds all mock-containing variables in the program (Section~\ref{sec:dec-doop} provides more details). Predicate \texttt{isMockVar} holds because of lines~\ref{line:facts-arv}--\ref{line:facts-cms}: \texttt{r2} satisfies \texttt{isMockVar} because IR line~\ref{line:lis3line6} assigns \texttt{r2} the return value from mock source method \texttt{createMock} (facts line~\ref{line:facts-arv}), and the call to \texttt{createMock} satisfies predicate \texttt{callsMockSource} (facts line~\ref{line:facts-cms}), which requires that the call destination \texttt{createMock} be enumerated as a constant in our 1-ary relation \texttt{MockSourceMethod} (facts line~\ref{line:facts-msm}), and that there be a call graph edge between the method invocation at line~\ref{line:lis3line6} and the mock source method (facts line~\ref{line:facts-cge}).


\begin{lstlisting}[basicstyle=\ttfamily, caption={Facts about invocation \texttt{r2.foo()} in method \texttt{test}.},
numbers=left,numbersep=2pt,basicstyle=\scriptsize\ttfamily, framesep=4.5mm, framexleftmargin=1.0mm, captionpos=b, label=lis:facts, escapechar=!, morekeywords={@Test}]
isMockInvocation(<Object:void foo()>/test/0, 
                 <Object:void foo()>, test, _, r2). !\label{line:facts-imi}!
|VirtualMethodInvocation(<Object:void foo()>/test/0, !\label{line:facts-vmi}!
|                        <Object:void foo()>, test).
|VirtualMethodInvocation_Base(<Object:void foo()>/test/0, r2).
|isMockVar(r2). !\label{line:facts-imv}!
|-AssignReturnValue(<Mockito:Object mock(Class)>/test/0!\label{line:facts-arv}!, r2).
|-callsMockSource(<Mockito:Object mock(Class)>/test/0). !\label{line:facts-cms}!
|MockSourceMethod(<Mockito:Object mock(Class)>). !\label{line:facts-msm}!
|CallGraphEdge(_,<Mockito:Object mock(Class)>/test/0, _,!\label{line:facts-cge}!
|              <Mockito: Object mock(Class)>). 
\end{lstlisting}


\paragraph{Extensions: arrays and collections} While designing \textsc{MockDetector}, we observed that developers store mock objects in arrays and collections. Listing~\ref{lis:container} presents method \textit{setUp()} in class \texttt{NodeListIteratorTest} from commons-collections-4.4. Line \ref{line:storeMocksInArray} puts mock \texttt{Node} objects in array field \texttt{nodes}, later used in tests. When the dataflow analysis encounters an assignment statement containing an array read or write, it first looks for values (local variables or field reference sources) on the opposite side of the assignment (the statement's destination or source) in the IR. It then checks whether any of these locals or fields are mocks. If so, then the tool would mark the local or field reference as an array mock---it propagates mockness to the array container.

\begin{lstlisting}[basicstyle=\ttfamily, caption={This example illustrates a field array container holding mock objects from \textit{setUp()} in \texttt{NodeListIteratorTest}.},
numbers=left,numbersep=2pt,basicstyle=\scriptsize\ttfamily,language = Java, framesep=4.5mm, framexleftmargin=1.0mm, captionpos=b, label=lis:container, escapechar=|, morekeywords={@Test}]
// Node array to be filled with mock Node instances
private Node[] nodes;
@Test protected void setUp() throws Exception {
  // create mock Node Instances and fill Node[] to be used by tests
  final Node node1 = createMock(Element.class);
  final Node node2 = createMock(Element.class);
  final Node node3 = createMock(Text.class);
  final Node node4 = createMock(Element.class);
  nodes = new Node[] {node1, node2, node3, node4}; |\label{line:storeMocksInArray}|
  // ...
}
\end{lstlisting}

Figure~\ref{fig:arrayMockIllustration} illustrates a mock-containing array, and Listing~\ref{lis:arrayIllustrationIR} shows the corresponding IR. Our analysis reaches the mock API call on lines~\ref{line:lis4line4}--\ref{line:lis4line6}, where it records that \texttt{\$r2} is a mock object---it creates a MockStatus with mock bit set for \texttt{\$r2}. The tool would then handle the cast expression assigning to \texttt{r1} on line~\ref{line:lis4line7}, giving it the same MockStatus as \texttt{\$r2}. When the analysis reaches line~\ref{line:lis4line9}, it finds an array reference on the left hand side, along with \texttt{r1} stored in the array on the right hand side of the assignment statement. At that point, it has a MockStatus associated with \texttt{r1}, with the mock bit turned on. It can now deduce that \texttt{\$r3} on the LHS is an array container which may hold a mock. Therefore, \textsc{MockDetector}'s imperative analysis associates \texttt{\$r3} with a MockStatus with mock-containing array bit set. Collections work similarly---we hard-code methods from the Java Collections API.

\begin{lstlisting}[basicstyle=\ttfamily, caption={Jimple Intermediate Representation for the array in Figure~\ref{fig:arrayMockIllustration}.},
numbers=left,numbersep=2pt,basicstyle=\scriptsize\ttfamily, framesep=4.5mm, framexleftmargin=1.0mm, captionpos=b, label=lis:arrayIllustrationIR, escapechar=|, morekeywords={@Test, specialinvoke, virtualinvoke, staticinvoke, newarray}]
java.lang.Object r1, $r2;
java.lang.Object[] $r3;

$r2 = staticinvoke <org.easymock.EasyMock: |\label{line:lis4line4}|
java.lang.Object createMock(java.lang.Class)>
(class "java.lang.Object;"); |\label{line:lis4line6}|
r1 = (java.lang.Object) $r2; |\label{line:lis4line7}|
$r3 = newarray (java.lang.Object)[1]; |\label{line:lis4line8}|
$r3[0] = r1;  |\label{line:lis4line9}|
\end{lstlisting}

\begin{figure}[h]
	\begin{lstlisting}[
	numbers=left,numbersep=0pt,basicstyle=\scriptsize\ttfamily,language = Java, framesep=4.5mm, framexleftmargin=1.0mm, captionpos=b, escapechar=|, morekeywords={@Test}]
	//       mock: |\cmark|    mockAPI: |\cmark|
	Object object1 = createMock(Object.class);
	
	// arrayMock: |\cmark| |$\Leftarrow$| array-write    |~~|  mock: |\cmark|
	objects |~| = new Object[]  |~|  { object1 };
	\end{lstlisting}
	
	\caption{Our static analysis also finds array mocks.}
	\label{fig:arrayMockIllustration}
	
\end{figure}

The declarative analysis uses analogous reasoning, but again uses relations instead of bits in the MockStatus abstraction.
