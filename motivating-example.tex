\section{motivating-example}

\lstset{language=java,
	keywordstyle=\color{blue}\bfseries,
	commentstyle=\color{green},
	stringstyle=\ttfamily\color{red!50!brown},
	showstringspaces=false}‎
\lstset{literate=%
	*{0}{{{\color{red!20!violet}0}}}1
	{1}{{{\color{red!20!violet}1}}}1
	{2}{{{\color{red!20!violet}2}}}1
	{3}{{{\color{red!20!violet}3}}}1
	{4}{{{\color{red!20!violet}4}}}1
	{5}{{{\color{red!20!violet}5}}}1
	{6}{{{\color{red!20!violet}6}}}1
	{7}{{{\color{red!20!violet}7}}}1
	{8}{{{\color{red!20!violet}8}}}1
	{9}{{{\color{red!20!violet}9}}}1
}

\begin{lstlisting}[basicstyle=\ttfamily, caption={This example illustrates a transitive call to EasyMock's \textit{CreateMock(java.lang.class)} function from test case \textit{testGetIterator()}.},
basicstyle=\scriptsize\ttfamily,language = Java, framesep=4.5mm,
framexleftmargin=1mm,,captionpos=b,label=lis:withZone]

private Node[] createNodes() {
	final Node node1 = createMock(Node.class);
	final Node node2 = createMock(Node.class);
	final Node node3 = createMock(Node.class);
	final Node node4 = createMock(Node.class);
	...
}

@Test
public void testGetIterator() {
	...
	final Node[] nodes = createNodes();
	...
}



\end{lstlisting}