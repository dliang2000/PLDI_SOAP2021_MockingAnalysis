\section{Application: Finding Focal Methods}
\label{sec:focal}

Section~\ref{sec:applications} surveyed applications of mock analysis.
We now discuss, in more detail, the detection of focal methods under test
(F-MUTs), as Ghafari et
al~\cite{ghafari15:_autom} call them.
Our analysis facilitates this search.

Ghafari's approach is applicable to tests for classes that implement
stateful objects. It assumes that each test has at least one focal
method---a call to a mutator method for the class under test.
They declare the last such call to be the focal method. However, their approach 
only works on tests of mutator methods.  Purely-functional methods are beyond the scope of their approach.

Another approach uses name-based heuristics, 
as shown by the methods2test
dataset~\cite{tufano2020unit}.
%% This approach uses two heuristics: 1)
%% the name of the test matches the name of the focal method; or, 2) the
%% test calls a unique method in the focal class. 
Rompaey and
Demeyer~\cite{rompaey09:_estab_traceab_links_unit_test} explore the
name-based approach along with simply using the last method called---not necessarily a mutator---as the class containing the focal method.  Romaey and Demeyer also
explore other heuristics, as we discuss in Section~\ref{sec:related}. Ghafari et al point out
that these heuristics have significant limitations, e.g. when a
test case has sub-scenarios.

\paragraph{Example: mock objects and focal methods} We show how we rule out calls by revisiting the unit test case from Listing~\ref{lis:mockCall}. Figure~\ref{fig:mockExampleEvaluation} illustrates finding mock object \texttt{session} and mock invocation \texttt{getRequest()}. In this context, our analysis rules out \texttt{getRequest()} as a focal method. We judge \texttt{getToolchainsForType()} to be the focal method since we assume that every test has at least one focal method, and it is the only method invocation remaining after elimination. Return array \texttt{basics}'s \texttt{length} attribute gets checked in the assertion statement on Line~\ref{line:assert}, which means this test case indeed tests the behaviour of \texttt{getToolchainsForType()}. 

For this unit test, since Ghafari's algorithm does not consider accessors (``inspector methods'' in their paper) as focal methods, they will presumably report no focal method for this unit test case (or any other test cases with no assertions), thus losing recall.

%Incidentally, since Ghafari's heuristic requires tests to have at least one assertion statement, their algorithm will presumably fail to return any focal methods when analyzing unit test cases without assertions.

With the methods2test heuristics, \texttt{getToolChainsForType()} does not appear to match under its name-based heuristic. The other heuristic finds calls to the focal class, so methods2test should identify the call to \texttt{getToolChainsForType()} with that heuristic. Our mock analysis can make that heuristic more precise by removing mocks from the set of eligible focal classes.

%% \begin{table*}
%% 	\centering
%% 	\caption{Comparison of \% of test cases with reported focal methods by the two automated focal method detection algorithms.}
%% 	\vspace*{.5em}
%% 	\resizebox{\textwidth}{!}{\begin{tabular}{lrrrrrlrrrr} \toprule
%% 		\multicolumn{5}{c}{Ghafari's algorithm} & & \multicolumn{5}{c}{methods2test}\\
%% 		\cmidrule{1-5} \cmidrule{7-11}
%% 		\thead{Benchmark} & \thead{Source Code \\ KLoC} & \thead{Reported \\ Focal Methods} & \thead{Test Cases} & \thead{\% of test cases \\ with focal \\ methods detected} &  & \thead{Benchmark} & \thead{Source Code \\ KLoC} & \thead{Reported \\ Focal Methods} & \thead{Test Cases} & \thead{\% of test cases \\ with focal \\ methods detected} \\ 
%% 		\cmidrule{1-5} \cmidrule{7-11}
		
%% 		commons-email-1.3.3 & 8.78 & 90  & 130 &  69\%  &  &    goja-0.1.14/goja-core  & 11.52 & 27  & 80 &  34\% \\
%% 		PureMVC-1.0.8 & 19.46 & 34  & 43 &  79\%  &  &  mock-socket-0.9.0    & 1.09  & 4  & 34 &  12\%      \\
%% 		XStream-1.4.4 & 54.93 & 513  & 968 &  53\%   &  &  project-sunbird-4.3.0/sunbird-lms-service   & 45.36 & 310  & 984 &  31\%   \\
%% 		JGAP-3.4.4  & 73.96 & 1015  & 1390 &  73\%  &  &   optiq-0.8/core    & 93.94  & 26  & 1346 &  2\%   \\
%% 		\bottomrule
%% 		Geometric Mean &   &  &  &  68\% &   &  &  &  &  &  12\% 
%% 	\end{tabular}}
%% 	\label{tab:focal-method-algorithm-comparison}
%% \end{table*}


\begin{figure}[h]
	\begin{lstlisting}[
	basicstyle=\ttfamily\scriptsize,language = Java, framesep=4.5mm, framexleftmargin=1.0mm, captionpos=b, escapechar={|}, mathescape=true, morekeywords={@Test}]
  @Test public void testMisconfiguredToolchain() throws Exception {
    //                mock:|\cmark|    mockAPI:|\cmark|
    MavenSession |\colorbox{olive}{session}| = |\colorbox{teal}{mock}| ( MavenSession.class );
    MavenExecutionRequest req =
        new DefaultMavenExecutionRequest();
    //     mock invocation:|\cmark| $\Rightarrow$ focal method:|\xmark|
    when( session.|\colorbox{red}{getRequest()}| ).thenReturn( req ); |\label{line:Fmock}|

    ToolchainPrivate[] basics =
      //                      focal method:|\cmark|
      toolchainManager.|\colorbox{gray}{getToolchainsForType}|("basic", session); |\label{line:Freal}|

    assertEquals( 0, basics.length );|\label{line:assert}|
  }
  \end{lstlisting}

  \caption{Example: removing mock invocation from focal method consideration.}
  \label{fig:mockExampleEvaluation}
\end{figure}

\paragraph{MockDetector and focal method detection}
We compare the static analysis and name-based approaches to detecting focal methods.
%% Table~\ref{tab:focal-method-algorithm-comparison} presents an approximation to recall---the percentage of test cases reported as having focal methods by each of the approaches. This assumes every test case has exactly one focal method. (A test case may have more than one focal method; we believe that a reasonable test has at least one focal method.) The four projects presented under Ghafari's algorithm are from their paper~\cite{ghafari15:_autom}, whereas we hand-picked four benchmarks from methods2test's dataset to match (as far as possible) those in Ghafari's work.
Data from~\cite{tufano2020unit} suggests that methods2test's name-based approach of focal method detection has low recall. This is not surprising---their heuristics have quite strict constraints, i.e. limited applicability. We reviewed their results and found that none of their reported focal method invocations were mocks, consistent with high precision.

Ghafari kindly shared with us recent focal method detection results. The percentage of test cases where they report focal methods (a proxy to recall) is reasonably high. Sometimes, their tool automatically reports a focal method that is actually a mock invocation from the developer's standpoint---\textsc{MockDetector} would help here. 
%On the other hand, their algorithm would return no focal method for test cases that come with no assertion statements. 
%Overall, Ghafari's algorithm would benefit from filters to increase precision, but 

In short, methods2test's approach finds focal methods with high precision but low recall, whereas Ghafari's algorithm has an acceptable recall but low precision. Our mock analysis can improve focal method detection---it can increase the precision of Ghafari-like methods by ruling out calls that are definitely not focal methods.

