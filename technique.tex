\section{Technique}
\label{sec:technique}

We implemented two complementary ways of statically computing mock information: 
a declarative implementation (using Doop~\cite{bravenboer09:_stric_declar_specif_sophis_point_analy}), along with
an imperative implementation of a dataflow analysis (using Soot~\cite{Vallee-Rai:1999:SJB:781995.782008}). While the core analysis implementations are similar, the Soot analysis is much easier for subsequent client analyses to build on. Because the Soot analysis is intraprocedural, it also runs much more quickly. On the other hand, it is easier to experiment with the Doop analysis. For instance, marking a field as mock-containing in Doop was quite easy (we added 3 rules), while the Soot version required us to write new code.
%We started this project with the usual imperative approach to implementing a static analysis---in our context, that meant using Soot. Then, when we wanted to experiment with adding more features to the analysis, we decided that this was a good opportunity to learn about Doop's declarative approach as well. We added new features to the Doop implementation and backported them to the Soot implementation. 
%Although the reason that two implementations exist is unimportant at this stage, the result is consequential:
The existence of these two implementations is a rare opportunity to cross-check static analysis results. We thus carefully compared our results to make sure that they matched.
%Since the focus of this paper is on mock analysis, we will not discuss specific implementation details in depth.
%While the core analysis is similar, the different implementation technologies have different affordances. For instance, it is easier for the Doop version to mark a field as mock-containing (we added 3 rules) than for the Soot version to do so. We start by describing each implementation in turn, and conclude this section with the commonalities between the two implementations. Section~\ref{sec:evaluation} then presents the results obtained using each technology and compares them.

\subsection{Declarative Doop Implementation}
\label{sec:dec-doop}
We next describe the declarative Doop-based technique that \textsc{MockDetector} uses. Both implementations propagate mockness from known mock sources, through the statements in the intermediate representation (IR), to potential mock invocation sites. Doop analyses specify propagation using Datalog rules. 

%We chose to implement a may-analysis rather than a must-analysis for two reasons: 1) we did not observe any cases where a value was assigned a mock on one branch and a real object on the other branch; 2) implementing a must-analysis would not help heuristics to find focal methods, as a must-analysis would rule out fewer mock invocations. 

Adding (context-insensitive) interprocedural support to a Doop analysis is almost trivial: we only needed to add two rules.
Both analyses support arrays, containers, and fields. Although declarative analyses are easier to define than their imperative counterparts, they are still subject to the feature interaction problem: for instance, they still must specifically handle the case where an array-typed field is assigned from a mock-containing array local.

% Mock Libraries discussed in Common Infrastructure section, perhaps refer to the paragraph in Common Infrastructure section?
The implementation declares facts for 9 mock source methods manually gleaned from the mock libraries' documentation, as specified through method signatures (e.g. 
\texttt{<org.mockito.\\Mockito: java.lang.Object mock(java.lang.Class)>}.)
It then declares that a variable {\tt v} satisfies \verb+isMockVar(v)+ if it is assigned from the return value of a mock source, or otherwise traverses the program's interprocedural control-flow graph, through assignments, which may possibly flow through fields, collections, or arrays. Finally, an invocation site is a mock invocation if the receiver object {\tt v} satisfies \verb+isMockVar(v)+.

\begin{lstlisting}[basicstyle=\ttfamily\small,numbers=none,label={lst:core},caption={Selected rules for Datalog mock analysis},
basicstyle=\scriptsize\ttfamily, framesep=4.5mm, framexleftmargin=1.0mm, captionpos=b, label=lis:datalog-rule, escapechar=!]
// v = mock()
isMockVar(v) :-
  AssignReturnValue(mi, v),
  callsMockSource(mi).
// v = (type) from
isMockVar(v) :- isMockVar(from),
  AssignCast(_/* type */, from, v, _/* inmethod */).
// v = v1
isMockVar(v) :- isMockVar(v1),
  AssignLocal(v1, v, _).
\end{lstlisting}

The Doop-provided predicates \texttt{AssignReturnValue}, \\\texttt{AssignCast}, and \texttt{AssignLocal} resemble Java bytecode instructions; for instance, \texttt{AssignLocal(?from:Var, ?to:Var, \\?inmethod:Method)} is an assignment statement copying from variable \texttt{?from} to \texttt{?to} in method \texttt{?inmethod}. (It is Datalog convention to prefix parameters with \texttt{?}s).

We designed the analysis in a modular fashion, such that the interprocedural, collections, arrays, and fields support can all be disabled through the use of \verb+#ifdef+s, which can be specified on the Doop command-line.

\paragraph{Interprocedural support}
As mentioned above, it was quite easy to implement interprocedural analysis in our Doop implementation; we leveraged Doop's call graph and simply propagated mockness information along its call graph edges.

\begin{lstlisting}[basicstyle=\ttfamily\small,numbers=none,caption={Two rules give interprocedural analysis in Doop.},
basicstyle=\scriptsize\ttfamily, framesep=4.5mm, framexleftmargin=1.0mm, captionpos=b, label=lis:interproc-rule, escapechar=!]
// v = callee(), where callee's return var is mock
isInterprocMockVar(v) :-
  AssignReturnValue(mi, v),
  mainAnalysis.CallGraphEdge(_, mi, _, callee),
  ReturnVar(v_callee, callee),
  isMockVar(v_callee).

// callee(v) results in formal
//   param of callee being mock
isInterprocMockVar(v_callee) :- isMockVar(v),
  ActualParam(n, mi, v),
  FormalParam(n, callee, v_callee),
  mainAnalysis.CallGraphEdge(_, mi, _, callee),
  Method_DeclaringType(callee, callee_class),
  ApplicationClass(callee_class).
\end{lstlisting}
We use Doop's call graph edges (\texttt{mainAnalysis.CallGraph\\Edge}) between the method invocation {\tt mi} and its callee {\tt callee}; the first rule propagates information from callees back to their callers, while the second rule propagates information from callers to callees through parameters. We restrict our analysis to so-called ``application classes'', excluding in particular the Java standard library. We chose to run our context-insensitive analysis on top of Doop's context-insensitive call graph. Mirroring Doop itself, it would also be possible to add context sensitivity to our analysis, potentially reducing the number of false positives.%, but our results suggest that this would not help much; we'll return to that point in Section~\ref{sec:evaluation}.

\paragraph{Arrays and Containers} We record local variables pointing to mock-containing arrays using  {\tt isArrayLocalThatContains\\Mocks}. This predicate is true whenever the program under analysis stores a mock variable into an array; we also transfer array-mockness through assignments and casts. When a local variable \texttt{v} is read from a mock-containing array \texttt{c}, then \texttt{v} is marked as a mock variable, as seen in the first rule below. A store of a mock variable \texttt{mv} into an array \texttt{c} causes that array to be \texttt{isArrayLocalThatContainsMocks}. Note that these predicates are mutually-recursive. 

\begin{lstlisting}[basicstyle=\ttfamily\small,numbers=none,caption={Rules for handling arrays.},
basicstyle=\scriptsize\ttfamily, framesep=4.5mm, framexleftmargin=1.0mm, captionpos=b, label=lis:array-rule, escapechar=!]
// v = c[idx]
isMockVar(v) :-
  isArrayLocalThatContainsMocks(c),
  LoadArrayIndex(c, v, _ /* idx */).

// c[idx] = mv
isArrayLocalThatContainsMocks(c) :-
  StoreArrayIndex(mv, c, _ /* idx */),
  isMockVar(mv).
\end{lstlisting}

We treat collections analogously. However, while there is one API for arrays---bytecode array load and store instructions---Java's Collections APIs include, by our count, 60 relevant methods. We support iterators, collection copies via constructors, and add-all methods. We use our classification of collection methods to identify collection reads and writes and handle them as we do array reads and writes. We also handle {\tt Collection.toArray} by propagating mockness from the collection to the array.%, except that we say that it is a mock-containing collection, not a mock-containing array.

\paragraph{Fields} Apart from the obvious rule stating that a field which is assigned from a mock satisfies {\tt fieldContainsMock}, we also label fields that have the {\tt org.mockito.Mock} or \\{\tt org.easyMock.Mock} annotations as mock-containing. We declare that a given field \emph{signature} may contain a mock, i.e. the field with a given signature belonging to all objects of a given type. We also support containers stored in fields.

\paragraph{Arrays and Fields} We also support not just array locals but also array fields. That is, when an array-typed field is assigned from a mock-containing array local, then it is also a mock. And when an array-typed local variable is assigned from a mock-containing array field, then that array local is a mock-containing array.

\subsection{Imperative Soot Implementation}
\label{subsec:soot}
We now briefly describe our Soot-based imperative dataflow analysis to find mocks. Like the declarative analysis, we track information from the creation sites through the control-flow graph. Here, we implement a forward dataflow may-analysis---an object is declared a mock if there exists some execution path where it may receive a mock value. The treatment of arrays, fields, and containers mirrors that in the Doop implementation. The dataflow analysis maps values in the IR to three bits: one for the value being a mock, one for it being an array containing a mock, and one for it being a collection containing a mock. %At most one of the three bits may be true for any given value.
Space limitations prevent us from including full details of this implementation, but we will make all implementations available, and the analysis is fully described in the first author's master's thesis\footnote{Citation temporarily omitted due to double-blind review, but available on our institutional repository.}.

% Mock Libraries discussed in Common Infrastructure section
%\paragraph{Define Common Mocking Library APIs}
%\label{subsubsec:collection}

%Our tool stores a pool of common APIs, provided by the analysis designer, which are used to create mock objects when using popular Java mocking libraries, including Mockito, EasyMock, and PowerMock. These APIs are the possible mock creation sites, where the locals/variables holding mock objects are first created.

%Given a pool of possible APIs to search for, our tool may analyze tests for their usage of these APIs.% Facts for mock source methods is discussed in doop, maybe this part could be discussed once to avoid repetition?


%% \paragraph{Forward Dataflow May Analysis}
%% \label{subsubsec:forward}

%To solve the problem, our tool uses forward may analysis, where it analyzes statements from top to bottom, and to keep variables that are verified to be mocks on any possible path at merged points. \textsc{MockDetector} uses the abstraction 
%Not having a mapping in the abstraction is equivalent to mapping to a MockStatus having all three bits false. 
%Our tool would store a value with MockStatus holding three false bits into the abstraction if and only if the value was once a mock or mock-containing container, but later redefined to a non-mock object or container without mock objects.

%Our merge operation is therefore a fairly standard pointwise disjunction of the two incoming values in terms of values and in terms of the 3 bits of \texttt{MockStatus}.

%% \textsc{MockDetector} implements the "may" logic in the following manner: it checks the two in-flows 
%% of \begin{lstlisting}[basicstyle=\ttfamily\small,numbers=none]
%% Map<Value, MockStatus>
%% \end{lstlisting}
%% from two paths. For any variable that is only stored in one map, the key-value pair is directly passed to the out-flow map. For a variable that is a shared key of the two maps, the analysis would update the out-flow's MockStatus by applying the "OR" operation on the "May Mock", "Array Mock", and "Collection Mock" bits from the MockStatus value retrieved from both in-flow maps. 

%% For each statement in a forward flow analysis, we consider two sets: generated set and killed set. In this study, the first set contains the locals that are judged to become mocks, whereas the killed set containing locals that are determined to no longer to be mocks. Equations (1) and (2) illustrates how the inflow and outflow are defined and calculated for each unit: $In(u)$, representing a program point before executing $u$, is the intersection of all outflows after executing each element in immediate predecessor statements of $u$; $Out(u)$, on the other hand, is determined by first removing the killed set from $In(u)$, and union the result with generated set. 

%% \begin{equation}
%% \mathrm{In}(u) = \bigcap_{u' \in preds(u)} \mathrm{Out}(u') 
%% \end{equation}

%% \begin{equation}
%% \mathrm{Out}(u) = (\mathrm{In}(u) - \mathrm{Kill}(u)) \bigcup \mathrm{Gen}(u) 
%% \end{equation}

%Our dataflow analysis uses fairly standard gen and kill sets in the flow function. We set bits in \texttt{MockStatus} as follows:
%For the gen set, we consider for the Values in the following scenarios to be included in the abstraction, with mock bit set to 1:

%% First, the gen set includes pre-analyzed fields containing mock objects defined via annotation (e.g. \texttt{@Mock}), inside a constructor \texttt{<init>}, or in JUnit's \texttt{@Before}/\texttt{setUp()} methods. We discuss the pre-analysis below in Section~\ref{subsubsec:pre-analysis}. 

%% Second, it includes local variables assigned from mock-creation source methods:
%% \begin{lstlisting}[basicstyle=\ttfamily\small,numbers=none]
%%     X x = mock(X);
%% \end{lstlisting}

%% Third, it includes values assigned from return values of read methods from mock-containing collections or arrays:
%% \begin{lstlisting}[basicstyle=\ttfamily\small,numbers=none]
%%     // array read;
%%     // r1 is in the in-set as an array mock
%%     X x = r1[0];
%%     // collection read;
%%     // r2 is in the in-set as a collection mock
%%     X x = r2.get(0);
%% \end{lstlisting}

%% Fourth, if \texttt{x} is a mock and casted and assigned to \texttt{x\_cast}, then the gen set includes \texttt{x\_cast} (e.g. \texttt{r1} in Listing~\ref{lis:arrayIllustrationIR}):
%% \begin{lstlisting}[basicstyle=\ttfamily\small,numbers=none]
%%     // x is a mock in the in-set
%%     X x_cast = (X) x;
%% \end{lstlisting}

%% Finally, the gen set includes copies of already-flagged mocks:
%% \begin{lstlisting}[basicstyle=\ttfamily\small,numbers=none]
%%     // x is a mock in the in-set
%%     X y = x;
%% \end{lstlisting}
%% The copy-related rules also apply to mock-containing arrays and collections. We add some additional rules for generating mocks that the program reads from collections and arrays, as well as rules for marking arrays and collections as mock-containing. For instance, in the below array write, if the in set has \texttt{r2} as a mock, then the destination \texttt{r1} will be generated as a mock-containing array. Similarly, if \texttt{r3} is a known mock, then the collection \texttt{\$r4} to which it is added (the list of collection add methods is hardcoded) will be generated as a mock-containing collection.
%% \begin{lstlisting}[basicstyle=\ttfamily\small,numbers=none]
%%     // r2 is in the in set as a mock
%%     r1[0] = r2;
%%     // r3 is in the in set as a mock
%%     $r4.<java.util.ArrayList: boolean
%%              add(java.lang.Object)>(r3);
%% \end{lstlisting}

% nah, we say that above now.
%% In a similar fashion, the gen set will include values that traverse the program's control-flow graph via assignments, from a value that already has mock-containing array or mock-containing collection bit set to true.

%% \begin{lstlisting}[basicstyle=\ttfamily\small,numbers=none]
%% // arr2 is already in the gen set 
%% // with mock-containing array bit on.
%% arr1 = arr2
%% // arr2 is already in the gen set 
%% // with collection containing mock bit on.
%% collection1 = collection2
%% \end{lstlisting}

%% ---------

%% In our analysis, the generated set consists of two steps. Consider the statement: 
%% \begin{lstlisting}[basicstyle=\ttfamily\small,numbers=none]
%% Employee employee = mock(Employee.class);
%% \end{lstlisting}
%% The intermediate representation generated in Jimple format would be:
%% \begin{lstlisting}[basicstyle=\ttfamily\small,numbers=none]
%% $r1 = staticinvoke <org.mockito.Mockito: 
%% java.lang.Object mock(java.lang.Class)>
%% (class "Lca/liang/Employee;")

%% r2 = (ca.liang.Employee) $r1
%% \end{lstlisting}W

%% In this example, $\$r1$ is the immediate receiver from Mockito's mock creation site, whereas $r2$ is the casted expression that gets carried along in the subsequent program. Thus, our tool would include the immediate receivers, and the casted expressions of mock objects into the generation set, in two steps. 

%% \paragraph{Arrays and containers} At a read from an array into a local variable, where the source array is mock-containing, we declare that the local destination is a mock. At a write of a local variable into an array, where the local variable is mock-containing, we declare that the array is mock-containing.

\paragraph{Interprocedural support} Heros~\cite{bodden12:_inter_proced_data_flow_analy} implements IFDS/IDE for Soot. We explored implementing our analysis in Heros, but this would be more difficult than in Doop, where we simply added two rules to get the interprocedural analysis. We decided that, for our purposes, the extra effort required to formulate our analysis as IFDS would not give us any additional insights; comparing the Doop intraprocedural and interprocedural results sufficed. (Practically speaking, the increased runtime of an IFDS implementation also makes it less viable for subsequent client analyses to use.)

%In particular, Heros uses a different API in its implementation than Soot. Conceptually, though, it should be no harder to implement an interprocedural Heros analysis than an intraprocedural Soot dataflow analysis. 

%With some effort, it would be possible to rewrite our mock analysis with Heros; however, 

%%  Several test suites use arrays or collection objects to hold mock objects. In this scenario, our tool would consider that mockness propagates out to the container. For instance, say the flow function encounters an array write \begin{lstlisting}[basicstyle=\ttfamily\small,numbers=none]
%%     r1[0] = r2
%% \end{lstlisting} 
%% Then, the tool will look for \texttt{r2} in the abstraction. Once the abstraction pinpoints \texttt{r2} (with mock bit on), it will include \texttt{r1} with mock-containing array bit set to true. (THIS PART STILL FEELS REPETITIVE.)


%% Taking an array as an example, our tool would first look for an array reference in the executing statement, meaning there is a read or a write from an array. If the effect is a STORE to the array, \textsc{MockDetector} would look for variables stored into the array, and check whether any of the variables have been found to be mocks. If so, it would label the array as an array mock, and set the relevant bit in the abstraction to true. Reversely, if is a LOAD effect from the array, \textsc{MockDetector} would check if the array itself has mock-containing array bit on in the abstraction. If so, it will mark the value assigned from the array LOAD with mock bit on, and store it in the abstraction.

%% *** we say more about collections in the related work, we should probably move that to here.


%% As stated above, we hardcode all relevant methods from the Java Collections API. There are 60 such methods in total, which together account for about 1/6th of the total lines in our Doop analysis. An iterator can be treated as a copy of the container, with the request of an object from the iterator being tantamount to a container get. An add-all method copies the mock-containing collection bit.


% The main difference is Java's \texttt{Collection} interface has multiple implementations, which expose different APIs for objects. \textsc{MockDetector} resolves this problem with a manually-constructed pool of read and write method APIs associated with each sub-type of the interface \texttt{java.util.Collection}. It subsequently checks (using the hierarchy) whether collection classes appear in statements containing invoke expression. This is achieved by first determining the declaring class of the invoked method. If the declaring class is of an interface, \textsc{MockDetector} would check whether \texttt{java.util.Collection} is a super-interface for the declaring class. Otherwise, if the declaring class is of a class type, \textsc{MockDetector} would check whether \texttt{java.util.Collection} is a super-interface for any of declaring class's implemented interfaces. If a collection sub-type container is presented, \textsc{MockDetector} would then check if a STORE effect is applied to the container, indicating some object is to be stored in the container. Once the object is determined to be a mock, the collection container variable would immediately be labelled as a collection mock, setting the relevant bit in the abstraction to true. Reversely, if a LOAD effect is applied to the container labelled as a collection mock, the object retrieved from the container will be labelled a mock, and setting the mock bit in the abstraction to true.

\subsubsection{Pre-Analyses for Field Mocks Defined in Constructors and Before Methods}
\label{subsubsec:pre-analysis} 
A number of our benchmarks define fields as mock objects via EasyMock or Mockito \texttt{@Mock} annotations, or initialize these fields in the \texttt{<init>} constructor or \texttt{@Before} methods (\textit{setUp()} in JUnit 3), which test runners will execute before any test methods from those classes. These mock field or mock-containing container fields are then used in tests. (The Doop analysis explicitly handles annotations, and the interprocedural rules handle \texttt{@Before} methods). In the Soot implementation, we use two pre-analyses before running the main analysis, under the assumption that fields are rarely mutated in the test cases (and that it is incorrect to do so). We have validated our assumption on benchmarks. An empirical analysis of our benchmarks shows that fewer than 0.3\% of fields (29/9352) are mutated in tests.
%Table~\ref{tab:mutations} shows a preliminary analysis of field mutation frequency inside test cases---fewer than 0.3\% of fields are mutated in test cases.

%% The first pre-analysis handles annotated field mocks and field mocks defined in constructors (\texttt{<init>} methods), while the second handles \texttt{@Before} and \texttt{setUp()} methods. 

% obvious enough that we don't need to belabour this point
% Listing~\ref{lis:annotatedMock} illustrates a Mockito annotated field mock example taken from \texttt{DefaultToolchainManagerTest.java} class in maven-core.

Our analysis retrieves all fields in all test classes, and marks fields annotated {\tt @org.mockito.Mock} or {\tt @org.easymock.Mock} as mocks. Also, Listing~\ref{lis:fieldMock} depicts an example where instance fields are initialized using field initializers. Java copies such initializers into all constructors (\texttt{<init>}). To detect such mock-containing fields, our first pre-analysis applies the dataflow analysis to all constructors in the test classes prior to running the main analysis, using the same logic that we use to detect mock objects or mock-containing containers. The second pre-analysis handles field mocks defined in \texttt{@Before} methods just like the first pre-analysis handled constructors.

%% Listing~\ref{lis:fieldMock2} illustrates an example where fields are defined as mocks via mock-creation source methods inside the @Before method. Similarly, the field mocks initialized inside the @Before methods are determined by applying the forward dataflow analysis strictly on all @Before methods before the main analysis, where the values  


%% \begin{lstlisting}[basicstyle=\ttfamily, caption={Example for Annotated field mocks from \texttt{DefaultToolchainManagerTest.java} in maven-core.},
%% basicstyle=\scriptsize\ttfamily,language = Java, framesep=4.5mm,
%% framexleftmargin=1mm, captionpos=b, label=lis:annotatedMock]
%% public class DefaultToolchainManagerTest
%% {
%%     @Mock
%%     private Logger logger;
%%     @Mock
%%     private ToolchainFactory toolchainFactory_basicType;
%%     @Mock
%%     private ToolchainFactory toolchainFactory_rareType;

%%     @Before
%%     public void onSetup() throws Exception
%%     {    
%%         // ...
%%         MockitoAnnotations.initMocks( this );
%%         // ...
%%     }
%% }
%% \end{lstlisting}

\begin{lstlisting}[basicstyle=\ttfamily, caption={Field mocks defined by field initializations from \texttt{TypeRuleTest} in jsonschema2pojo.},
basicstyle=\scriptsize\ttfamily,language = Java, framesep=4.5mm,
framexleftmargin=1mm, captionpos=b, label=lis:fieldMock, numbers=none]
private GenerationConfig config = mock(GenerationConfig.class);
private RuleFactory ruleFactory = mock(RuleFactory.class);
\end{lstlisting}

% not necessary here. you can add it to your thesis.
%% \begin{lstlisting}[basicstyle=\ttfamily, caption={Example for field mocks defined in a \texttt{@Before} method.},
%% basicstyle=\scriptsize\ttfamily,language = Java, framesep=4.5mm,
%% framexleftmargin=1mm, captionpos=b, label=lis:fieldMock2]
%% public class PayRollMockTest {
%%     private EmployeeDB employeeDB;
%%     private BankService bankService;

%%     @Before
%%     public void init() {
%%         // ...
%%         employeeDB = mock(EmployeeDB.class);
%%         bankService = mock(BankService.class);
%%         // ...
%%     }
%% }
%% \end{lstlisting}

\subsection{Common Infrastructure}
\label{sec:common}
We have parameterized our technique with respect to mocking libraries and have instantiated it with respect to the popular Java libraries Mockito, EasyMock, and PowerMock. We also support JUnit 3 and 4+. %We discuss the parameterization in this subsection.

JUnit and mocking libraries all rely heavily on reflection, and would normally pose problems for static analyses: the set of reachable test methods is enumerated at run-time; mock libraries create mock objects reflectively. Fortunately, their use of reflection is stylized. We designed our analyses to soundly handle these libraries.

\paragraph{JUnit and Driver Generation}
JUnit tests are simply methods that developers write in test classes, appropriately annotated (in JUnit 3 by method name starting with ``test'', in 4+ by a \texttt{@Test} annotation). A JUnit test runner uses reflection to find tests. Out of the box, static analysis engines do not see tests as reachable code.

% what about hierarchical drivers?

Thus, to enable static analysis over a benchmark's test suite, our tool uses Soot to generate a driver class for each Java sub-package of the suite (e.g. \texttt{org.apache.ibatis.executor.statement}). In each of these sub-package driver classes, our tool creates a \textit{runall()} method, which invokes all methods within the sub-package that JUnit considers tests, as well as non-constructor test cases, all surrounded by calls to class-level init/setup and teardown methods. Concrete test methods are particularly easy to call from a driver, as they are specified to have no parameters and are not supposed to rely on any particular ordering. 
Our tool then creates a RootDriver class at the root package level, which invokes the \textit{runall()} method in each sub-package driver class, along with the \texttt{@Test}/\texttt{@Before}/\texttt{@After} methods found in classes located at the root. The drivers that we generate also contain code to catch all checked exceptions declared to be thrown by the unit tests. Both our Soot and Doop implementations use the generated driver classes.

All static frameworks must somehow approximate the set of entry points as appropriate for their targets. For instance, the Wala framework~\cite{wala19:_t} also creates synthetic entry points, but it does this to perform pointer analysis on a program's main code rather than to enumerate the program's test cases.

%% \paragraph{Intraprocedural Analysis} The Soot analysis is intraprocedural and the Doop analysis has an intraprocedural version. Intraprocedurally, we make the unsound (but reasonable for our anticipated use case) assumption that mockness can be dropped between callers and callees: at method entry points, no parameters are assumed to be mocks, and at method returns, the returned object is never a mock. Doop's interprocedural version drops this assumption and instead context-insensitively propagates information from callers to callees and back; we discuss the results in Section~\ref{sec:evaluation}.

%% \paragraph{Mock Libraries}
%% Our supported mock libraries take different approaches to instantiating mocks. All of the libraries have methods that generate mock objects; for instance, EasyMock contains the \texttt{createMock()} method. We consider return values from these methods to be mock objects. Additionally, Mockito contains a fluent \texttt{verify()} method which returns a mock object. Finally, Mockito and EasyMock also allow developers to mark fields as \texttt{@Mock}; we treat reads from such fields as mock objects. Both implementations start the analysis with these hard coded facts on mock source methods, as described in the mock libraries' documentation.
\subsection{Analysis Design Considerations}
\label{subsec:analysis-design}
\paragraph{Static vs Dynamic Analysis}
We chose to implement static analyses to best support integration with our envisioned downstream client analysis; the intraprocedural Soot implementation is most suitable for integration, even if it is less precise. For collecting information about tests, whose behaviour generally does not vary between executions, dynamic analysis would also be quite precise. A dynamic analysis would work well for reporting information to developers, but less well as a feeder analysis to subsequent ones.

\paragraph{Sources of Imprecision}
We explicitly list all known sources of imprecision in our analysis; we investigate their practical impact in Section~\ref{sec:evaluation}.
Approximations may result in false positives (threatening precision), while missed behaviours can cause false negatives (threatening soundness).
\begin{itemize}[noitemsep]
\item At control-flow merges, incoming branches (loops, conditionals) may bring information which must be approximated; our may-analysis could thus report false positives.
\item At method boundaries (formal parameters, return values), our intraprocedural implementations assume that incoming values are not mocks (unless a return value from a Collections API), while the Doop interprocedural implementation uses a context-insensitive approximation. The not-mock assumption may result in false negatives for Soot; extraneous call-graph edges may result in Doop interprocedural false positives. (Lambdas are implemented using method calls and result in intraprocedural false negatives.)
\item Arrays, fields, and collections are approximated such that if any mock is stored in the array, field, or collection, then all values coming from there are mocks; we are subject to false positives if arrays, fields or collections are heterogeneous over time or space.
\item Although we have manually categorized collection APIs and mock APIs, it is possible that our enumeration is incomplete, leading to false negatives.
  Soot also assumes that developers do not call inherited versions of mock creation sites (e.g. a wrapper of the mock source method). Doop's interprocedural analysis does detect such mock creations (fewer than 5).
  Our approach cannot detect when subjects build their own mocks; however, those are more likely to be stubs or fakes~\cite{fowler07:_mocks_arent_stubs} than true mocks. 
  \item As usual, our static analysis approach assumes that it has access to all of the program source; missing code (e.g. dynamically loaded) may result in false negatives.
\end{itemize}

%Call graphs are useful to our intraprocedural analysis because they help identify calls to mock source methods.

\paragraph{Mock Analysis is not just Reaching Definitions}
Our intraprocedural mock analysis is similar to reaching definitions. However, we point out some important differences which motivated our decision to design and implement a bespoke analysis. Even the base analysis requires slightly more than reaching definitions, and then we also support fields, containers and arrays. Our field and array analyses detected additional mocks in half of our benchmarks (Section~\ref{sec:evaluation}). It would not be obvious to support arrays in the reaching definition paradigm; they would require additional support.

For the base analysis, reaching definitions can sometimes indicate whether mock definitions reach uses. However, mock analysis also requires propagating information through casts, as seen in the \texttt{AssignCast} rule in the declarative analysis; at statement {\tt s: x = (X)y;}, then {\tt x} would be a mock iff {\tt y} is. Statement {\tt s} would be a definition for {\tt x}, and a further query would be needed to know whether {\tt y} was a mock or not.

\paragraph{Program Slicing}
Another alternative to reaching definitions for finding mocks is program slicing: to find out whether an invocation {\tt o.f()} is a mock invocation, compute a slice of statements that affect {\tt o.f()}. The usual issue with slicing is that too much of the program is flagged as potentially relevant, as pointed out by Sridharan et al~\cite{sridharan07:_thin_slicin}. They thus proposed thin slicing, which returns a much smaller fraction of the program. Such an approach could work for developers, generalizing the reaching definitions approach alluded to above. Note that the slicing would need to include results through arrays and fields. Even so, it would again be difficult to use slices as information about mocks in a subsequent analysis; our technique instead specifically identifies mock-containing variables and invocations.

